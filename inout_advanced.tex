\documentclass{article}
\usepackage{fancyvrb}
\usepackage{color}
\usepackage[utf8]{inputenc}
\usepackage[a4paper, total={6.2in, 10in}]{geometry}
\usepackage[many]{tcolorbox}
\usepackage[T1]{fontenc}
\usepackage[slovene]{babel}

\usetikzlibrary{calc}

\setlength\parindent{0pt}

\makeatletter
\def\PY@reset{\let\PY@it=\relax \let\PY@bf=\relax%
    \let\PY@ul=\relax \let\PY@tc=\relax%
    \let\PY@bc=\relax \let\PY@ff=\relax}
\def\PY@tok#1{\csname PY@tok@#1\endcsname}
\def\PY@toks#1+{\ifx\relax#1\empty\else%
    \PY@tok{#1}\expandafter\PY@toks\fi}
\def\PY@do#1{\PY@bc{\PY@tc{\PY@ul{%
    \PY@it{\PY@bf{\PY@ff{#1}}}}}}}
\def\PY#1#2{\PY@reset\PY@toks#1+\relax+\PY@do{#2}}

\@namedef{PY@tok@c}{\def\PY@tc##1{\textcolor[rgb]{0.63,0.63,0.88}{##1}}}
\@namedef{PY@tok@cp}{\def\PY@tc##1{\textcolor[rgb]{0.19,0.51,0.25}{##1}}}
\@namedef{PY@tok@k}{\let\PY@bf=\textbf\def\PY@tc##1{\textcolor[rgb]{0.00,0.00,0.67}{##1}}}
\@namedef{PY@tok@o}{\def\PY@tc##1{\textcolor[rgb]{1.00,0.02,0.02}{##1}}}
\@namedef{PY@tok@p}{\def\PY@tc##1{\textcolor[rgb]{1.00,0.02,0.02}{##1}}}
\@namedef{PY@tok@n}{\def\PY@tc##1{\textcolor[rgb]{0.05,0.05,0.05}{##1}}}
\@namedef{PY@tok@s}{\def\PY@tc##1{\textcolor[rgb]{0.16,0.16,1.00}{##1}}}
\@namedef{PY@tok@m}{\def\PY@tc##1{\textcolor[rgb]{0.94,0.03,0.94}{##1}}}
\@namedef{PY@tok@kc}{\let\PY@bf=\textbf\def\PY@tc##1{\textcolor[rgb]{0.00,0.00,0.67}{##1}}}
\@namedef{PY@tok@kd}{\let\PY@bf=\textbf\def\PY@tc##1{\textcolor[rgb]{0.00,0.00,0.67}{##1}}}
\@namedef{PY@tok@kn}{\let\PY@bf=\textbf\def\PY@tc##1{\textcolor[rgb]{0.00,0.00,0.67}{##1}}}
\@namedef{PY@tok@kp}{\let\PY@bf=\textbf\def\PY@tc##1{\textcolor[rgb]{0.00,0.00,0.67}{##1}}}
\@namedef{PY@tok@kr}{\let\PY@bf=\textbf\def\PY@tc##1{\textcolor[rgb]{0.00,0.00,0.67}{##1}}}
\@namedef{PY@tok@kt}{\let\PY@bf=\textbf\def\PY@tc##1{\textcolor[rgb]{0.00,0.00,0.67}{##1}}}
\@namedef{PY@tok@na}{\def\PY@tc##1{\textcolor[rgb]{0.05,0.05,0.05}{##1}}}
\@namedef{PY@tok@nb}{\def\PY@tc##1{\textcolor[rgb]{0.05,0.05,0.05}{##1}}}
\@namedef{PY@tok@bp}{\def\PY@tc##1{\textcolor[rgb]{0.05,0.05,0.05}{##1}}}
\@namedef{PY@tok@nc}{\def\PY@tc##1{\textcolor[rgb]{0.05,0.05,0.05}{##1}}}
\@namedef{PY@tok@no}{\def\PY@tc##1{\textcolor[rgb]{0.05,0.05,0.05}{##1}}}
\@namedef{PY@tok@nd}{\def\PY@tc##1{\textcolor[rgb]{0.05,0.05,0.05}{##1}}}
\@namedef{PY@tok@ni}{\def\PY@tc##1{\textcolor[rgb]{0.05,0.05,0.05}{##1}}}
\@namedef{PY@tok@ne}{\def\PY@tc##1{\textcolor[rgb]{0.05,0.05,0.05}{##1}}}
\@namedef{PY@tok@nf}{\def\PY@tc##1{\textcolor[rgb]{0.05,0.05,0.05}{##1}}}
\@namedef{PY@tok@fm}{\def\PY@tc##1{\textcolor[rgb]{0.05,0.05,0.05}{##1}}}
\@namedef{PY@tok@py}{\def\PY@tc##1{\textcolor[rgb]{0.05,0.05,0.05}{##1}}}
\@namedef{PY@tok@nl}{\def\PY@tc##1{\textcolor[rgb]{0.05,0.05,0.05}{##1}}}
\@namedef{PY@tok@nn}{\def\PY@tc##1{\textcolor[rgb]{0.05,0.05,0.05}{##1}}}
\@namedef{PY@tok@nx}{\def\PY@tc##1{\textcolor[rgb]{0.05,0.05,0.05}{##1}}}
\@namedef{PY@tok@nt}{\def\PY@tc##1{\textcolor[rgb]{0.05,0.05,0.05}{##1}}}
\@namedef{PY@tok@nv}{\def\PY@tc##1{\textcolor[rgb]{0.05,0.05,0.05}{##1}}}
\@namedef{PY@tok@vc}{\def\PY@tc##1{\textcolor[rgb]{0.05,0.05,0.05}{##1}}}
\@namedef{PY@tok@vg}{\def\PY@tc##1{\textcolor[rgb]{0.05,0.05,0.05}{##1}}}
\@namedef{PY@tok@vi}{\def\PY@tc##1{\textcolor[rgb]{0.05,0.05,0.05}{##1}}}
\@namedef{PY@tok@vm}{\def\PY@tc##1{\textcolor[rgb]{0.05,0.05,0.05}{##1}}}
\@namedef{PY@tok@sa}{\def\PY@tc##1{\textcolor[rgb]{0.16,0.16,1.00}{##1}}}
\@namedef{PY@tok@sb}{\def\PY@tc##1{\textcolor[rgb]{0.16,0.16,1.00}{##1}}}
\@namedef{PY@tok@sc}{\def\PY@tc##1{\textcolor[rgb]{0.16,0.16,1.00}{##1}}}
\@namedef{PY@tok@dl}{\def\PY@tc##1{\textcolor[rgb]{0.16,0.16,1.00}{##1}}}
\@namedef{PY@tok@sd}{\def\PY@tc##1{\textcolor[rgb]{0.16,0.16,1.00}{##1}}}
\@namedef{PY@tok@s2}{\def\PY@tc##1{\textcolor[rgb]{0.16,0.16,1.00}{##1}}}
\@namedef{PY@tok@se}{\def\PY@tc##1{\textcolor[rgb]{0.16,0.16,1.00}{##1}}}
\@namedef{PY@tok@sh}{\def\PY@tc##1{\textcolor[rgb]{0.16,0.16,1.00}{##1}}}
\@namedef{PY@tok@si}{\def\PY@tc##1{\textcolor[rgb]{0.16,0.16,1.00}{##1}}}
\@namedef{PY@tok@sx}{\def\PY@tc##1{\textcolor[rgb]{0.16,0.16,1.00}{##1}}}
\@namedef{PY@tok@sr}{\def\PY@tc##1{\textcolor[rgb]{0.16,0.16,1.00}{##1}}}
\@namedef{PY@tok@s1}{\def\PY@tc##1{\textcolor[rgb]{0.16,0.16,1.00}{##1}}}
\@namedef{PY@tok@ss}{\def\PY@tc##1{\textcolor[rgb]{0.16,0.16,1.00}{##1}}}
\@namedef{PY@tok@mb}{\def\PY@tc##1{\textcolor[rgb]{0.94,0.03,0.94}{##1}}}
\@namedef{PY@tok@mf}{\def\PY@tc##1{\textcolor[rgb]{0.94,0.03,0.94}{##1}}}
\@namedef{PY@tok@mh}{\def\PY@tc##1{\textcolor[rgb]{0.94,0.03,0.94}{##1}}}
\@namedef{PY@tok@mi}{\def\PY@tc##1{\textcolor[rgb]{0.94,0.03,0.94}{##1}}}
\@namedef{PY@tok@il}{\def\PY@tc##1{\textcolor[rgb]{0.94,0.03,0.94}{##1}}}
\@namedef{PY@tok@mo}{\def\PY@tc##1{\textcolor[rgb]{0.94,0.03,0.94}{##1}}}
\@namedef{PY@tok@ow}{\def\PY@tc##1{\textcolor[rgb]{1.00,0.02,0.02}{##1}}}
\@namedef{PY@tok@ch}{\def\PY@tc##1{\textcolor[rgb]{0.63,0.63,0.88}{##1}}}
\@namedef{PY@tok@cm}{\def\PY@tc##1{\textcolor[rgb]{0.63,0.63,0.88}{##1}}}
\@namedef{PY@tok@cpf}{\def\PY@tc##1{\textcolor[rgb]{0.63,0.63,0.88}{##1}}}
\@namedef{PY@tok@c1}{\def\PY@tc##1{\textcolor[rgb]{0.63,0.63,0.88}{##1}}}
\@namedef{PY@tok@cs}{\def\PY@tc##1{\textcolor[rgb]{0.63,0.63,0.88}{##1}}}

\def\PYZbs{\char`\\}
\def\PYZus{\char`\_}
\def\PYZob{\char`\{}
\def\PYZcb{\char`\}}
\def\PYZca{\char`\^}
\def\PYZam{\char`\&}
\def\PYZlt{\char`\<}
\def\PYZgt{\char`\>}
\def\PYZsh{\char`\#}
\def\PYZpc{\char`\%}
\def\PYZdl{\char`\$}
\def\PYZhy{\char`\-}
\def\PYZsq{\char`\'}
\def\PYZdq{\char`\"}
\def\PYZti{\char`\~}
% for compatibility with earlier versions
\def\PYZat{@}
\def\PYZlb{[}
\def\PYZrb{]}
\makeatother

\definecolor{myblue}{RGB}{0,163,243}
\definecolor{myred}{RGB}{243, 10, 25}
\definecolor{mygreen}{RGB}{50, 205, 50}

\newcommand{\fon}[1]{\fontfamily{#1}\selectfont}


\tcbset{examplestyle/.style={
  enhanced,
  outer arc=4pt,
  arc=4pt,
  colframe=myblue,
  colback=myblue!20,
  attach boxed title to top left,
  boxed title style={
    colback=myblue,
    outer arc=4pt,
    arc=4pt,
    top=3pt,
    bottom=3pt,
    },
  fonttitle=\sffamily
  }
}

\tcbset{inoutstyle/.style={
  enhanced,
  outer arc=4pt,
  arc=4pt,
  colframe=mygreen,
  colback=mygreen!20,
  attach boxed title to top left,
  boxed title style={
    colback=mygreen,
    outer arc=4pt,
    arc=4pt,
    top=3pt,
    bottom=3pt,
    },
  fonttitle=\sffamily,
  fontupper=\ttfamily,
  fontlower=\ttfamily,
  }
}

\tcbset{errorstyle/.style={
  enhanced,
  outer arc=4pt,
  arc=4pt,
  colframe=myred,
  colback=myred!20,
  attach boxed title to top left,
  boxed title style={
    colback=myred,
    outer arc=4pt,
    arc=4pt,
    top=3pt,
    bottom=3pt,
    },
  fonttitle=\sffamily
  }
}

\newtcolorbox[auto counter,number within=section]{examples}[1][]{
  examplestyle,
  colback=white,
  title=Primer,
  overlay unbroken and first={
      \path
        let
        \p1=(title.north east),
        \p2=(frame.north east)
        in
        node[anchor=west,font=\sffamily,color=myblue,text width=\x2-\x1]
        at (title.east) {#1};
  }
}
\newtcolorbox[auto counter]{errors}[1][]{
  errorstyle,
  colback=white,
  title=Pogoste napake,
  overlay unbroken and first={
      \path
        let
        \p1=(title.north east),
        \p2=(frame.north east)
        in
        node[anchor=west,font=\sffamily,color=myblue,text width=\x2-\x1]
        at (title.east) {#1};
  }
}

\newtcolorbox[auto counter]{inout}[1][]{
  inoutstyle,
  colback=white,
  title=Primer vhoda in izhoda,
  overlay unbroken and first={
      \path
        let
        \p1=(title.north east),
        \p2=(frame.north east)
        in
        node[anchor=west,font=\sffamily,color=myblue,text width=\x2-\x1]
        at (title.east) {#1};
  }
}
\title{Branje in pisanje 2}
\date{}

\begin{document}
\maketitle

\section{Scanf}

Funkcija \verb+scanf+ lahko bere različne vrste podatkov:
\begin{itemize}
	\item \verb+%s+ - beseda (vsi znaki razen praznih znakov (glej spodaj)) (\verb+char+)
	\item \verb+%c+ - en (poljuben) znak (\verb+char+)
	\item \verb+%d+ - število med $-2^{31}$ in $2^{31}-1$ (\verb+int+)
	\item \verb+%lld+ - število med $-2^{63}$ in $2^{63}-1$ (\verb+long long+)
	\item ...
\end{itemize}

\noindent Prazni znaki (whitespace) so znaki, ki jih ne vidimo:
\begin{itemize}
	\item \verb+\n+ nova vrstica
	\item \verb+\t+ zamik
	\item " " presledek
\end{itemize}

S formatnikom \verb+%s+ funkcija \verb+scanf+ bere do prvega takšnega znaka. Prebere tudi vse take znake, ki sledijo, a jih ne shrani. Naslednjič, ko jo pokličemo, bere od prvega nepraznega znaka naprej.

\begin{examples}
\begin{Verbatim}[commandchars=\\\{\}]
\PY{c+cp}{\PYZsh{}}\PY{c+cp}{include}\PY{c+cpf}{\PYZlt{}stdio.h\PYZgt{}}

\PY{k+kt}{int}\PY{+w}{ }\PY{n+nf}{main}\PY{p}{(}\PY{p}{)}\PY{p}{\PYZob{}}
\PY{+w}{	}\PY{k+kt}{char}\PY{+w}{ }\PY{n}{a}\PY{p}{[}\PY{l+m+mi}{50}\PY{p}{]}\PY{p}{,}\PY{+w}{ }\PY{n}{b}\PY{p}{[}\PY{l+m+mi}{50}\PY{p}{]}\PY{p}{;}
\PY{+w}{	}\PY{n}{scanf}\PY{p}{(}\PY{l+s}{\PYZdq{}}\PY{l+s}{\PYZpc{}s}\PY{l+s}{\PYZdq{}}\PY{p}{,}\PY{+w}{ }\PY{n}{a}\PY{p}{)}\PY{p}{;}
\PY{+w}{	}\PY{n}{scanf}\PY{p}{(}\PY{l+s}{\PYZdq{}}\PY{l+s}{\PYZpc{}s}\PY{l+s}{\PYZdq{}}\PY{p}{,}\PY{+w}{ }\PY{n}{b}\PY{p}{)}\PY{p}{;}
\PY{+w}{	}\PY{n}{printf}\PY{p}{(}\PY{l+s}{\PYZdq{}}\PY{l+s}{\PYZpc{}s \PYZpc{}s}\PY{l+s+se}{\PYZbs{}n}\PY{l+s}{\PYZdq{}}\PY{p}{,}\PY{+w}{ }\PY{n}{a}\PY{p}{,}\PY{+w}{ }\PY{n}{b}\PY{p}{)}\PY{p}{;}
\PY{+w}{	}\PY{k}{return}\PY{+w}{ }\PY{l+m+mi}{0}\PY{p}{;}
\PY{p}{\PYZcb{}}
\end{Verbatim}

\begin{inout}
Hello \\ \\ World!
\tcblower
Hello World!
\end{inout}


\end{examples}

\pagebreak
\section{Branje do konca vrstice}

Lahko določimo, da se \verb+scanf+ ne bo ustavil pri prvem praznem znaku, temveč šele pri koncu vrstice (ali kje drugje).

\begin{examples}
\begin{Verbatim}[commandchars=\\\{\}]
\PY{c+cp}{\PYZsh{}}\PY{c+cp}{include}\PY{c+cpf}{\PYZlt{}stdio.h\PYZgt{}}

\PY{k+kt}{int}\PY{+w}{ }\PY{n+nf}{main}\PY{p}{(}\PY{p}{)}\PY{p}{\PYZob{}}
\PY{+w}{	}\PY{k+kt}{char}\PY{+w}{ }\PY{n}{a}\PY{p}{[}\PY{l+m+mi}{50}\PY{p}{]}\PY{p}{;}
\PY{+w}{	}\PY{n}{scanf}\PY{p}{(}\PY{l+s}{\PYZdq{}}\PY{l+s}{\PYZpc{}[\PYZca{}}\PY{l+s+se}{\PYZbs{}n}\PY{l+s}{]}\PY{l+s}{\PYZdq{}}\PY{p}{,}\PY{+w}{ }\PY{n}{a}\PY{p}{)}\PY{p}{;}\PY{+w}{ }\PY{c+c1}{//ali scanf(\PYZdq{}\PYZpc{}[\PYZca{}\PYZbs{}n]\PYZpc{}*c\PYZdq{}) in brez getchar()}
\PY{+w}{	}\PY{n}{getchar}\PY{p}{(}\PY{p}{)}\PY{p}{;}
\PY{+w}{	}\PY{n}{printf}\PY{p}{(}\PY{l+s}{\PYZdq{}}\PY{l+s}{\PYZpc{}s}\PY{l+s+se}{\PYZbs{}n}\PY{l+s}{\PYZdq{}}\PY{p}{,}\PY{+w}{ }\PY{n}{a}\PY{p}{)}\PY{p}{;}
\PY{+w}{	}\PY{k}{return}\PY{+w}{ }\PY{l+m+mi}{0}\PY{p}{;}
\PY{p}{\PYZcb{}}
\end{Verbatim}

\begin{inout}
Beremo do konca vrstice.
\tcblower
Beremo do konca vrstice.
\end{inout}

\end{examples}

\verb+[...]+ - med oklepaje pišemo navodila, kakšne znake lahko bere
\begin{itemize}
	\item \verb+[a-z]+ - beri male črke angleške abecede
	\item \verb+[a-zA-Z0-9]+ - beri male in velike črke in številke
\end{itemize}

\verb+[^...]+ - beri vse do znakov, ki sledijo strešici
\begin{itemize}
	\item \verb+[^\n]+ - beri vse do \verb+\n+
\end{itemize}

%CHECK
\begin{errors}
Za razliko od \verb+%s+ funkcija \verb+scanf+ s \verb+[^\n]+ prebere vse do \verb+\n+, tega pa ne prebere in se ustavi pred njim. Ko funkcijo pokličemo naslednjič, začne tam, kjer je nazadnje ostala, kar je v tem primeru točno pred znakom \verb+\n+. Če jo torej ponovno pokličemo s parametrom \verb+[^\n]+, se ne bo nikamor premaknila, saj je pred njo znak za novo vrstico. \\
\end{errors}

%CHECK
\verb+%c+ - prebere en znak (\verb+char+), ki je lahko karkoli - črka, številka, prazen znak... \\
\verb+%*c+ - prebere en znak, a ga ne shrani \\
Funkcija \verb+getchar+ dela podobno, vzame en znak in ga ne shrani. \\

Za razliko od tega s formatnikom \verb+%s+ preberemo vse prazne znake med dvema
nepraznima nizoma in jih ne shranimo. Tudi, če bodo prazni znaki pred besedo,
jih bo program ignoriral in poiskal prvi neprazen znak.

\pagebreak
\section{Branje do konca vhoda}

Če vemo točno, koliko besed/številk/vrstic bomo imeli na vhodu, jih lahko preberemo s for zanko. Kako preberemo neznano količino podatkov na vhodu, tako da preberemo vse? \\
Če napišemo \verb+while(true)+ ali \verb+while(1)+, bomo sicer prebrali vse, a se program ne bo nikoli ustavil.
Namesto tega lahko napišemo:

\begin{examples}
\begin{Verbatim}[commandchars=\\\{\}]
\PY{c+cp}{\PYZsh{}}\PY{c+cp}{include}\PY{c+cpf}{\PYZlt{}stdio.h\PYZgt{}}

\PY{k+kt}{int}\PY{+w}{ }\PY{n+nf}{main}\PY{p}{(}\PY{p}{)}\PY{p}{\PYZob{}}
\PY{+w}{	}\PY{k+kt}{int}\PY{+w}{ }\PY{n}{a}\PY{p}{;}
\PY{+w}{	}\PY{k}{while}\PY{p}{(}\PY{n}{scanf}\PY{p}{(}\PY{l+s}{\PYZdq{}}\PY{l+s}{\PYZpc{}d}\PY{l+s}{\PYZdq{}}\PY{p}{,}\PY{+w}{ }\PY{o}{\PYZam{}}\PY{n}{a}\PY{p}{)}\PY{+w}{ }\PY{o}{!}\PY{o}{=}\PY{+w}{ }\PY{n}{EOF}\PY{p}{)}\PY{p}{\PYZob{}}
\PY{+w}{		}\PY{n}{printf}\PY{p}{(}\PY{l+s}{\PYZdq{}}\PY{l+s}{\PYZpc{}d}\PY{l+s+se}{\PYZbs{}n}\PY{l+s}{\PYZdq{}}\PY{p}{,}\PY{+w}{ }\PY{n}{a}\PY{o}{*}\PY{n}{a}\PY{p}{)}\PY{p}{;}\PY{+w}{ }\PY{c+c1}{//izpisujemo kvadrate prebranega števila}
\PY{+w}{	}\PY{p}{\PYZcb{}}
\PY{+w}{	}\PY{k}{return}\PY{+w}{ }\PY{l+m+mi}{0}\PY{p}{;}
\PY{p}{\PYZcb{}}
\end{Verbatim}

\begin{inout}
1 2 3 4 5
\tcblower
1\\4\\9\\16\\25
\end{inout}
\end{examples}

\vskip 0.15in
%CHECK
\noindent Funkcija \verb+scanf+ vrne število formatnikov, ki jih je uspešno
prebrala (če ji kot parameter podamo samo \verb+%d+, bo vrnila $1$, če je
uspešno prebrala število, sicer pa 0). \verb+EOF+ (End of File) vrne, če na
vhodu ni ničesar več za prebrati. Tedaj se bo zanka ustavila. Če programu
vhodne podatke podajamo iz datoteke, se to zgodi avtomatsko ob koncu datoteke,
če pa mu podatke podajamo na roko, konec vhoda sporočimo s
\verb-Ctrl+D (Linux in MacOS)- ali \verb-Ctrl+Z- (Windows).

\begin{examples}
\begin{Verbatim}[commandchars=\\\{\}]
\PY{c+cp}{\PYZsh{}}\PY{c+cp}{include}\PY{c+cpf}{\PYZlt{}stdio.h\PYZgt{}}

\PY{k+kt}{int}\PY{+w}{ }\PY{n+nf}{main}\PY{p}{(}\PY{p}{)}\PY{+w}{ }\PY{p}{\PYZob{}}
\PY{+w}{    }\PY{k+kt}{int}\PY{+w}{ }\PY{n}{a}\PY{p}{,}\PY{+w}{ }\PY{n}{b}\PY{p}{,}\PY{+w}{ }\PY{n}{c}\PY{p}{;}
\PY{+w}{    }\PY{k+kt}{int}\PY{+w}{ }\PY{n}{r}\PY{+w}{ }\PY{o}{=}\PY{+w}{ }\PY{n}{scanf}\PY{p}{(}\PY{l+s}{\PYZdq{}}\PY{l+s}{\PYZpc{}d\PYZpc{}d\PYZpc{}d}\PY{l+s}{\PYZdq{}}\PY{p}{,}\PY{+w}{ }\PY{o}{\PYZam{}}\PY{n}{a}\PY{p}{,}\PY{+w}{ }\PY{o}{\PYZam{}}\PY{n}{b}\PY{p}{,}\PY{+w}{ }\PY{o}{\PYZam{}}\PY{n}{c}\PY{p}{)}\PY{p}{;}
\PY{+w}{    }\PY{n}{printf}\PY{p}{(}\PY{l+s}{\PYZdq{}}\PY{l+s}{\PYZpc{}d}\PY{l+s+se}{\PYZbs{}n}\PY{l+s}{\PYZdq{}}\PY{p}{,}\PY{+w}{ }\PY{n}{r}\PY{p}{)}\PY{p}{;}
\PY{+w}{    }\PY{k}{return}\PY{+w}{ }\PY{l+m+mi}{0}\PY{p}{;}
\PY{p}{\PYZcb{}}\PY{p}{;}
\end{Verbatim}

\begin{inout}
	5 8 100
	\tcblower
	3
\end{inout}
\begin{inout}
	5 8 miha
	\tcblower
	2
\end{inout}


\end{examples}

\pagebreak
\section{Branje in pisanje v in iz niza}

\verb+scanf+ uporabljamo za branje s standardnega vhoda, \verb+printf+ pa za pisanje na standardni izhod.
Namesto tega lahko beremo in pišemo tudi drugače, npr. v in iz nizov.
Za to uporabljamo funkciji \verb+sscanf+ in \verb+sprintf+, ki delata podobno kot \verb+scanf+ in \verb+printf+.

\begin{examples}
\begin{Verbatim}[commandchars=\\\{\}]
\PY{c+cp}{\PYZsh{}}\PY{c+cp}{include}\PY{c+cpf}{\PYZlt{}stdio.h\PYZgt{}}

\PY{k+kt}{int}\PY{+w}{ }\PY{n+nf}{main}\PY{p}{(}\PY{p}{)}\PY{p}{\PYZob{}}
\PY{+w}{	}\PY{k+kt}{int}\PY{+w}{ }\PY{n}{n}\PY{p}{;}
\PY{+w}{	}\PY{k+kt}{char}\PY{+w}{ }\PY{n}{a}\PY{p}{[}\PY{l+m+mi}{10}\PY{p}{]}\PY{p}{,}\PY{+w}{ }\PY{n}{b}\PY{p}{;}
\PY{+w}{	}\PY{k+kt}{char}\PY{+w}{ }\PY{n}{text}\PY{p}{[}\PY{p}{]}\PY{o}{=}\PY{l+s}{\PYZdq{}}\PY{l+s}{Slovenska 157 a}\PY{l+s}{\PYZdq{}}\PY{p}{;}
\PY{+w}{	}\PY{n}{sscanf}\PY{p}{(}\PY{n}{text}\PY{p}{,}\PY{+w}{ }\PY{l+s}{\PYZdq{}}\PY{l+s}{\PYZpc{}s\PYZpc{}d \PYZpc{}c}\PY{l+s}{\PYZdq{}}\PY{p}{,}\PY{+w}{ }\PY{n}{a}\PY{p}{,}\PY{+w}{ }\PY{o}{\PYZam{}}\PY{n}{n}\PY{p}{,}\PY{+w}{ }\PY{o}{\PYZam{}}\PY{n}{b}\PY{p}{)}\PY{p}{;}
\PY{+w}{	}\PY{n}{sprintf}\PY{p}{(}\PY{n}{text}\PY{p}{,}\PY{+w}{ }\PY{l+s}{\PYZdq{}}\PY{l+s}{\PYZpc{}c \PYZpc{}d \PYZpc{}s}\PY{l+s}{\PYZdq{}}\PY{p}{,}\PY{+w}{ }\PY{n}{b}\PY{p}{,}\PY{+w}{ }\PY{n}{n}\PY{p}{,}\PY{+w}{ }\PY{n}{a}\PY{p}{)}\PY{p}{;}
\PY{+w}{	}\PY{n}{printf}\PY{p}{(}\PY{l+s}{\PYZdq{}}\PY{l+s}{\PYZpc{}s}\PY{l+s+se}{\PYZbs{}n}\PY{l+s}{\PYZdq{}}\PY{p}{,}\PY{+w}{ }\PY{n}{text}\PY{p}{)}\PY{p}{;}
\PY{+w}{	}\PY{k}{return}\PY{+w}{ }\PY{l+m+mi}{0}\PY{p}{;}
\PY{p}{\PYZcb{}}
\end{Verbatim}

\begin{inout}

\tcblower
a 157 Slovenska
\end{inout}

\end{examples}

\begin{errors}
Medtem ko \verb+%s+ praznih znakov ne shrani, jih \verb+%c+ obravnava tako kot
vse ostale. Če pogeldamo prejšnji primer vidimo, da je v funkciji
\verb+scanf+ pred \verb+%c+ presledek. Ker so med posameznimi podatki, ki
jih želimo prebrati, presledki, bi \verb+%c+ pobral presledek, ne pa črke,
ki mu sledi. Če med posamezne formatnike postavimo presledek, ta načeloma
pobere prazne znake do naslednjega drugačnega znaka, vendar to v splošnem
ni dobra praksa.
\end{errors}

\noindent Funkciji \verb+sscanf+ podamo tri parametre (ali več):
\begin{enumerate}
	\item ime niza, iz katerega naj bere
	\item tip podatka, ki naj ga prebere (niz, število...)
	\item kam naj ta podatek zapiše (ime spremenljivke)
\end{enumerate}
	
\noindent Funkciji \verb+sprintf+ prav tako podamo tri parametre (ali več):
\begin{enumerate}
	\item ime niza, kamor naj piše
	\item tip podatka, ki naj ga zapiše
	\item kaj naj zapiše (ime spremenljivke ali podatek sam)
\end{enumerate}

\pagebreak
\section{Branje in pisanje v in iz datoteke}

Podobno kot v niz lahko pišemo in beremo tudi v in iz datotek s funkcijama \verb+fscanf+ in \verb+fprintf+.

\begin{examples}
\begin{Verbatim}[commandchars=\\\{\}]
\PY{c+cp}{\PYZsh{}}\PY{c+cp}{include}\PY{c+cpf}{\PYZlt{}stdio.h\PYZgt{}}

\PY{k+kt}{int}\PY{+w}{ }\PY{n+nf}{main}\PY{p}{(}\PY{p}{)}\PY{p}{\PYZob{}}
\PY{+w}{	}\PY{k+kt}{int}\PY{+w}{ }\PY{n}{a}\PY{p}{;}
\PY{+w}{	}\PY{k+kt}{FILE}\PY{+w}{ }\PY{o}{*}\PY{n}{fr}\PY{p}{,}\PY{+w}{ }\PY{o}{*}\PY{n}{fw}\PY{p}{;}
\PY{+w}{	}\PY{n}{fr}\PY{+w}{ }\PY{o}{=}\PY{+w}{ }\PY{n}{fopen}\PY{p}{(}\PY{l+s}{\PYZdq{}}\PY{l+s}{in.txt}\PY{l+s}{\PYZdq{}}\PY{p}{,}\PY{+w}{ }\PY{l+s}{\PYZdq{}}\PY{l+s}{r}\PY{l+s}{\PYZdq{}}\PY{p}{)}\PY{p}{;}
\PY{+w}{	}\PY{n}{fw}\PY{+w}{ }\PY{o}{=}\PY{+w}{ }\PY{n}{fopen}\PY{p}{(}\PY{l+s}{\PYZdq{}}\PY{l+s}{out.txt}\PY{l+s}{\PYZdq{}}\PY{p}{,}\PY{+w}{ }\PY{l+s}{\PYZdq{}}\PY{l+s}{w}\PY{l+s}{\PYZdq{}}\PY{p}{)}\PY{p}{;}
\PY{+w}{	}\PY{k}{while}\PY{p}{(}\PY{n}{fscanf}\PY{p}{(}\PY{n}{fr}\PY{p}{,}\PY{+w}{ }\PY{l+s}{\PYZdq{}}\PY{l+s}{\PYZpc{}d}\PY{l+s}{\PYZdq{}}\PY{p}{,}\PY{+w}{ }\PY{o}{\PYZam{}}\PY{n}{a}\PY{p}{)}\PY{+w}{ }\PY{o}{!}\PY{o}{=}\PY{+w}{ }\PY{n}{EOF}\PY{p}{)}\PY{p}{\PYZob{}}
\PY{+w}{		}\PY{n}{fprintf}\PY{p}{(}\PY{n}{fw}\PY{p}{,}\PY{+w}{ }\PY{l+s}{\PYZdq{}}\PY{l+s}{\PYZpc{}d}\PY{l+s+se}{\PYZbs{}n}\PY{l+s}{\PYZdq{}}\PY{p}{,}\PY{+w}{ }\PY{n}{a}\PY{o}{*}\PY{n}{a}\PY{p}{)}\PY{p}{;}
\PY{+w}{	}\PY{p}{\PYZcb{}}\PY{+w}{ }\PY{c+c1}{//iz datoteke fr preberemo vsa števila in v fw izpišemo njihove kvadrate}
\PY{+w}{	}\PY{n}{fclose}\PY{p}{(}\PY{n}{fr}\PY{p}{)}\PY{p}{;}
\PY{+w}{	}\PY{n}{fclose}\PY{p}{(}\PY{n}{fw}\PY{p}{)}\PY{p}{;}
\PY{+w}{	}\PY{k}{return}\PY{+w}{ }\PY{l+m+mi}{0}\PY{p}{;}
\PY{p}{\PYZcb{}}
\end{Verbatim}

\begin{inout}
{\color{blue} \bf in.txt:} \\
1 2 3 4 5 6 7
\tcblower
{\color{blue} \bf out.txt:} \\
1\\
4\\
9\\
16\\
25\\
36\\
49
\end{inout}

(Ta program ne bere s standardega vhoda in ne piše ne standardni izhod.)
\end{examples}

\verb+FILE+ - tip podatka
\verb+fopen+ - funkcija, s katero odpremo datoteko, podamo ji ime datoteke, ki
jo želimo odpreti in način \verb+"r"+ (\emph{read} - za branje) ali \verb+"w"+
(\emph{write} - za pisanje).\\ Datoteke, ki jih odpiramo za branje, morajo že
prej obstajati, sicer se bo program sesul. \\

Za datoteke, v katere pišemo, ni nujno, da že obstajajo. Če še ne obstajajo, bo
program ustvaril novo datoteko s tem imenom. Če že obstajajo, program ne bo
pisal na konec te datoteke, temveč bo pobrisal vso prejšnjo vsebino.
Če želimo obstoječi datoteki dodajati vsebino, moramo uporabiti način
\verb+"a"+ (\emph{append} - pripenjanje).

\verb+fclose+ - funkcija, ki zapre odprto datoteko, podamo ji ime datoteke

\end{document}
