\documentclass{article}
\usepackage{fancyvrb}
\usepackage{color}
\usepackage[utf8]{inputenc}
\usepackage[a4paper, total={6.2in, 9in}]{geometry}
\usepackage[many]{tcolorbox}
\usepackage[T1]{fontenc}
\usepackage[slovene]{babel}

\usetikzlibrary{calc}

\setlength\parindent{0pt}

\makeatletter
\def\PY@reset{\let\PY@it=\relax \let\PY@bf=\relax%
    \let\PY@ul=\relax \let\PY@tc=\relax%
    \let\PY@bc=\relax \let\PY@ff=\relax}
\def\PY@tok#1{\csname PY@tok@#1\endcsname}
\def\PY@toks#1+{\ifx\relax#1\empty\else%
    \PY@tok{#1}\expandafter\PY@toks\fi}
\def\PY@do#1{\PY@bc{\PY@tc{\PY@ul{%
    \PY@it{\PY@bf{\PY@ff{#1}}}}}}}
\def\PY#1#2{\PY@reset\PY@toks#1+\relax+\PY@do{#2}}

\@namedef{PY@tok@c}{\def\PY@tc##1{\textcolor[rgb]{0.63,0.63,0.88}{##1}}}
\@namedef{PY@tok@cp}{\def\PY@tc##1{\textcolor[rgb]{0.19,0.51,0.25}{##1}}}
\@namedef{PY@tok@k}{\let\PY@bf=\textbf\def\PY@tc##1{\textcolor[rgb]{0.00,0.00,0.67}{##1}}}
\@namedef{PY@tok@o}{\def\PY@tc##1{\textcolor[rgb]{1.00,0.02,0.02}{##1}}}
\@namedef{PY@tok@p}{\def\PY@tc##1{\textcolor[rgb]{1.00,0.02,0.02}{##1}}}
\@namedef{PY@tok@n}{\def\PY@tc##1{\textcolor[rgb]{0.05,0.05,0.05}{##1}}}
\@namedef{PY@tok@s}{\def\PY@tc##1{\textcolor[rgb]{0.16,0.16,1.00}{##1}}}
\@namedef{PY@tok@m}{\def\PY@tc##1{\textcolor[rgb]{0.94,0.03,0.94}{##1}}}
\@namedef{PY@tok@kc}{\let\PY@bf=\textbf\def\PY@tc##1{\textcolor[rgb]{0.00,0.00,0.67}{##1}}}
\@namedef{PY@tok@kd}{\let\PY@bf=\textbf\def\PY@tc##1{\textcolor[rgb]{0.00,0.00,0.67}{##1}}}
\@namedef{PY@tok@kn}{\let\PY@bf=\textbf\def\PY@tc##1{\textcolor[rgb]{0.00,0.00,0.67}{##1}}}
\@namedef{PY@tok@kp}{\let\PY@bf=\textbf\def\PY@tc##1{\textcolor[rgb]{0.00,0.00,0.67}{##1}}}
\@namedef{PY@tok@kr}{\let\PY@bf=\textbf\def\PY@tc##1{\textcolor[rgb]{0.00,0.00,0.67}{##1}}}
\@namedef{PY@tok@kt}{\let\PY@bf=\textbf\def\PY@tc##1{\textcolor[rgb]{0.00,0.00,0.67}{##1}}}
\@namedef{PY@tok@na}{\def\PY@tc##1{\textcolor[rgb]{0.05,0.05,0.05}{##1}}}
\@namedef{PY@tok@nb}{\def\PY@tc##1{\textcolor[rgb]{0.05,0.05,0.05}{##1}}}
\@namedef{PY@tok@bp}{\def\PY@tc##1{\textcolor[rgb]{0.05,0.05,0.05}{##1}}}
\@namedef{PY@tok@nc}{\def\PY@tc##1{\textcolor[rgb]{0.05,0.05,0.05}{##1}}}
\@namedef{PY@tok@no}{\def\PY@tc##1{\textcolor[rgb]{0.05,0.05,0.05}{##1}}}
\@namedef{PY@tok@nd}{\def\PY@tc##1{\textcolor[rgb]{0.05,0.05,0.05}{##1}}}
\@namedef{PY@tok@ni}{\def\PY@tc##1{\textcolor[rgb]{0.05,0.05,0.05}{##1}}}
\@namedef{PY@tok@ne}{\def\PY@tc##1{\textcolor[rgb]{0.05,0.05,0.05}{##1}}}
\@namedef{PY@tok@nf}{\def\PY@tc##1{\textcolor[rgb]{0.05,0.05,0.05}{##1}}}
\@namedef{PY@tok@fm}{\def\PY@tc##1{\textcolor[rgb]{0.05,0.05,0.05}{##1}}}
\@namedef{PY@tok@py}{\def\PY@tc##1{\textcolor[rgb]{0.05,0.05,0.05}{##1}}}
\@namedef{PY@tok@nl}{\def\PY@tc##1{\textcolor[rgb]{0.05,0.05,0.05}{##1}}}
\@namedef{PY@tok@nn}{\def\PY@tc##1{\textcolor[rgb]{0.05,0.05,0.05}{##1}}}
\@namedef{PY@tok@nx}{\def\PY@tc##1{\textcolor[rgb]{0.05,0.05,0.05}{##1}}}
\@namedef{PY@tok@nt}{\def\PY@tc##1{\textcolor[rgb]{0.05,0.05,0.05}{##1}}}
\@namedef{PY@tok@nv}{\def\PY@tc##1{\textcolor[rgb]{0.05,0.05,0.05}{##1}}}
\@namedef{PY@tok@vc}{\def\PY@tc##1{\textcolor[rgb]{0.05,0.05,0.05}{##1}}}
\@namedef{PY@tok@vg}{\def\PY@tc##1{\textcolor[rgb]{0.05,0.05,0.05}{##1}}}
\@namedef{PY@tok@vi}{\def\PY@tc##1{\textcolor[rgb]{0.05,0.05,0.05}{##1}}}
\@namedef{PY@tok@vm}{\def\PY@tc##1{\textcolor[rgb]{0.05,0.05,0.05}{##1}}}
\@namedef{PY@tok@sa}{\def\PY@tc##1{\textcolor[rgb]{0.16,0.16,1.00}{##1}}}
\@namedef{PY@tok@sb}{\def\PY@tc##1{\textcolor[rgb]{0.16,0.16,1.00}{##1}}}
\@namedef{PY@tok@sc}{\def\PY@tc##1{\textcolor[rgb]{0.16,0.16,1.00}{##1}}}
\@namedef{PY@tok@dl}{\def\PY@tc##1{\textcolor[rgb]{0.16,0.16,1.00}{##1}}}
\@namedef{PY@tok@sd}{\def\PY@tc##1{\textcolor[rgb]{0.16,0.16,1.00}{##1}}}
\@namedef{PY@tok@s2}{\def\PY@tc##1{\textcolor[rgb]{0.16,0.16,1.00}{##1}}}
\@namedef{PY@tok@se}{\def\PY@tc##1{\textcolor[rgb]{0.16,0.16,1.00}{##1}}}
\@namedef{PY@tok@sh}{\def\PY@tc##1{\textcolor[rgb]{0.16,0.16,1.00}{##1}}}
\@namedef{PY@tok@si}{\def\PY@tc##1{\textcolor[rgb]{0.16,0.16,1.00}{##1}}}
\@namedef{PY@tok@sx}{\def\PY@tc##1{\textcolor[rgb]{0.16,0.16,1.00}{##1}}}
\@namedef{PY@tok@sr}{\def\PY@tc##1{\textcolor[rgb]{0.16,0.16,1.00}{##1}}}
\@namedef{PY@tok@s1}{\def\PY@tc##1{\textcolor[rgb]{0.16,0.16,1.00}{##1}}}
\@namedef{PY@tok@ss}{\def\PY@tc##1{\textcolor[rgb]{0.16,0.16,1.00}{##1}}}
\@namedef{PY@tok@mb}{\def\PY@tc##1{\textcolor[rgb]{0.94,0.03,0.94}{##1}}}
\@namedef{PY@tok@mf}{\def\PY@tc##1{\textcolor[rgb]{0.94,0.03,0.94}{##1}}}
\@namedef{PY@tok@mh}{\def\PY@tc##1{\textcolor[rgb]{0.94,0.03,0.94}{##1}}}
\@namedef{PY@tok@mi}{\def\PY@tc##1{\textcolor[rgb]{0.94,0.03,0.94}{##1}}}
\@namedef{PY@tok@il}{\def\PY@tc##1{\textcolor[rgb]{0.94,0.03,0.94}{##1}}}
\@namedef{PY@tok@mo}{\def\PY@tc##1{\textcolor[rgb]{0.94,0.03,0.94}{##1}}}
\@namedef{PY@tok@ow}{\def\PY@tc##1{\textcolor[rgb]{1.00,0.02,0.02}{##1}}}
\@namedef{PY@tok@ch}{\def\PY@tc##1{\textcolor[rgb]{0.63,0.63,0.88}{##1}}}
\@namedef{PY@tok@cm}{\def\PY@tc##1{\textcolor[rgb]{0.63,0.63,0.88}{##1}}}
\@namedef{PY@tok@cpf}{\def\PY@tc##1{\textcolor[rgb]{0.63,0.63,0.88}{##1}}}
\@namedef{PY@tok@c1}{\def\PY@tc##1{\textcolor[rgb]{0.63,0.63,0.88}{##1}}}
\@namedef{PY@tok@cs}{\def\PY@tc##1{\textcolor[rgb]{0.63,0.63,0.88}{##1}}}

\def\PYZbs{\char`\\}
\def\PYZus{\char`\_}
\def\PYZob{\char`\{}
\def\PYZcb{\char`\}}
\def\PYZca{\char`\^}
\def\PYZam{\char`\&}
\def\PYZlt{\char`\<}
\def\PYZgt{\char`\>}
\def\PYZsh{\char`\#}
\def\PYZpc{\char`\%}
\def\PYZdl{\char`\$}
\def\PYZhy{\char`\-}
\def\PYZsq{\char`\'}
\def\PYZdq{\char`\"}
\def\PYZti{\char`\~}
% for compatibility with earlier versions
\def\PYZat{@}
\def\PYZlb{[}
\def\PYZrb{]}
\makeatother

\definecolor{myblue}{RGB}{0,163,243}
\definecolor{myred}{RGB}{243, 10, 25}
\definecolor{mygreen}{RGB}{50, 205, 50}

\newcommand{\fon}[1]{\fontfamily{#1}\selectfont}


\tcbset{examplestyle/.style={
  enhanced,
  outer arc=4pt,
  arc=4pt,
  colframe=myblue,
  colback=myblue!20,
  attach boxed title to top left,
  boxed title style={
    colback=myblue,
    outer arc=4pt,
    arc=4pt,
    top=3pt,
    bottom=3pt,
    },
  fonttitle=\sffamily
  }
}

\tcbset{inoutstyle/.style={
  enhanced,
  outer arc=4pt,
  arc=4pt,
  colframe=mygreen,
  colback=mygreen!20,
  attach boxed title to top left,
  boxed title style={
    colback=mygreen,
    outer arc=4pt,
    arc=4pt,
    top=3pt,
    bottom=3pt,
    },
  fonttitle=\sffamily,
  fontupper=\ttfamily,
  fontlower=\ttfamily,
  }
}

\tcbset{errorstyle/.style={
  enhanced,
  outer arc=4pt,
  arc=4pt,
  colframe=myred,
  colback=myred!20,
  attach boxed title to top left,
  boxed title style={
    colback=myred,
    outer arc=4pt,
    arc=4pt,
    top=3pt,
    bottom=3pt,
    },
  fonttitle=\sffamily
  }
}

\newtcolorbox[auto counter,number within=section]{examples}[1][]{
  examplestyle,
  colback=white,
  title=Primer,
  overlay unbroken and first={
      \path
        let
        \p1=(title.north east),
        \p2=(frame.north east)
        in
        node[anchor=west,font=\sffamily,color=myblue,text width=\x2-\x1]
        at (title.east) {#1};
  }
}
\newtcolorbox[auto counter]{errors}[1][]{
  errorstyle,
  colback=white,
  title=Pogoste napake,
  overlay unbroken and first={
      \path
        let
        \p1=(title.north east),
        \p2=(frame.north east)
        in
        node[anchor=west,font=\sffamily,color=myblue,text width=\x2-\x1]
        at (title.east) {#1};
  }
}

\newtcolorbox[auto counter]{inout}[1][]{
  inoutstyle,
  colback=white,
  title=Primer vhoda in izhoda,
  overlay unbroken and first={
      \path
        let
        \p1=(title.north east),
        \p2=(frame.north east)
        in
        node[anchor=west,font=\sffamily,color=myblue,text width=\x2-\x1]
        at (title.east) {#1};
  }
}
\title{Branje in pisanje}
\date{}

\begin{document}
\maketitle

\section{Vhod in izhod}
Programi za svoje delovanje potrebujejo način za komunikacijo z
uporabnikom. Kompleksnejši programi v ta namen uporabljajo ekran, miško
in tipkovnico. Pri tekmovalnem programiranju pa najpogosteje uporabljamo najpreprostejši način za komunikacijo: pisanje in branje s \emph{standardnega vhoda in izhoda}. Običajno to pomeni, da se nam ob zagonu programa odpre okno, kamor lahko pišemo programu in kamor program izpisuje stvari.\\
Ko želimo, da naš program kaj izpiše, uporabimo \emph{funkcijo} \verb+printf+.
\begin{examples}
\begin{Verbatim}[commandchars=\\\{\}]
\PY{c+cp}{\PYZsh{}}\PY{c+cp}{include}\PY{c+cpf}{\PYZlt{}stdio.h\PYZgt{}}

\PY{k+kt}{int}\PY{+w}{ }\PY{n+nf}{main}\PY{p}{(}\PY{p}{)}\PY{p}{\PYZob{}}
\PY{+w}{	}\PY{n}{printf}\PY{p}{(}\PY{l+s}{\PYZdq{}}\PY{l+s}{Hello World!}\PY{l+s+se}{\PYZbs{}n}\PY{l+s}{\PYZdq{}}\PY{p}{)}\PY{p}{;}
\PY{+w}{	}\PY{k}{return}\PY{+w}{ }\PY{l+m+mi}{0}\PY{p}{;}
\PY{p}{\PYZcb{}}
\end{Verbatim}

\begin{inout}
\tcblower
Hello World!
\end{inout}


\end{examples}

\verb+printf+ - funkcija, ki ji v dvojnih narekovajih damo besedilo ali števila, ki jih želimo izpisati\\
\verb+\n+ - znak za novo vrstico
\\\\
Stvari, ki jih mora vsebovati (skoraj) vsak program:
\begin{itemize}
	\item \verb+stdio.h+ - \emph{knjižnica} (datoteka), ki vsebuje funkcije, ki jih bomo uporabljali v programu (kot npr. \verb+printf+ in \verb+scanf+)
	\item \verb+#include<>+ - ukaz, s katerim našemu programu povemo, katere knjižnice potrebuje
	\item \verb+int main(){}+ - telo našega programa - večino kode v programu napišemo med zavite oklepaje
	\item \verb+return 0+ - zadnja vrstica v programu, ki sporoča računalniku, da se je pravilno zaključil
\end{itemize}

\noindent V program lahko dodamo \emph{komentarje}. To je takšno besedilo, ki je napisano v kodi, a vsebinsko ne vpliva na program.
\begin{itemize}
	\item \verb+//+ - s tem zakomentiramo vse od poševnic do konca vrstice
	\item \verb+/* */+ - s tem lahko zakomentiramo več vrstic ali del znotraj vrstice
\end{itemize}

\begin{examples}
\begin{Verbatim}[commandchars=\\\{\}]
\PY{c+cp}{\PYZsh{}}\PY{c+cp}{include}\PY{c+cpf}{\PYZlt{}stdio.h\PYZgt{}}

\PY{k+kt}{int}\PY{+w}{ }\PY{n+nf}{main}\PY{p}{(}\PY{p}{)}\PY{p}{\PYZob{}}\PY{c+c1}{ //komentar do konca vrstice}
\PY{+w}{	}\PY{n}{printf}\PY{+w}{ }\PY{c+cm}{/*komentar znotraj vrstice*/}\PY{p}{(}\PY{l+s}{\PYZdq{}}\PY{l+s}{Zivjo svet!}\PY{l+s+se}{\PYZbs{}n}\PY{l+s}{\PYZdq{}}\PY{p}{)}\PY{p}{;}
\PY{+w}{	}\PY{c+cm}{/*}
\PY{c+cm}{	komentar}
\PY{c+cm}{	čez}
\PY{c+cm}{	več}
\PY{c+cm}{	vrstic}
\PY{c+cm}{	*/}
\PY{+w}{	}\PY{k}{return}\PY{+w}{ }\PY{l+m+mi}{0}\PY{p}{;}\PY{+w}{ }
\PY{p}{\PYZcb{}}
\end{Verbatim}

\begin{inout}
\tcblower
Zivjo svet!
\end{inout}

\end{examples}

\begin{errors}
Zadnji znak, ki ga program izpiše, mora biti \verb+\n+.
\end{errors}

\begin{errors}
Večina vrstic v programu se konča s podpičjem (\verb+;+) (skoraj vse razen tistih, ki se končajo z oklepaji  ali zavitimi zaklepaji). Brez tega program ne bo delal.
\end{errors}

\pagebreak
\section{Branje}
Program za branje stvari, ki mu jih sporočamo, uporablja funkcijo \verb+scanf+. 

\begin{examples}
\begin{Verbatim}[commandchars=\\\{\}]
\PY{c+cp}{\PYZsh{}}\PY{c+cp}{include}\PY{c+cpf}{\PYZlt{}stdio.h\PYZgt{}}

\PY{k+kt}{int}\PY{+w}{ }\PY{n+nf}{main}\PY{p}{(}\PY{p}{)}\PY{p}{\PYZob{}}
\PY{+w}{	}\PY{k+kt}{char}\PY{+w}{ }\PY{n}{ime}\PY{p}{[}\PY{l+m+mi}{50}\PY{p}{]}\PY{p}{;}
\PY{+w}{	}\PY{n}{printf}\PY{p}{(}\PY{l+s}{\PYZdq{}}\PY{l+s}{Kako ti je ime?}\PY{l+s+se}{\PYZbs{}n}\PY{l+s}{\PYZdq{}}\PY{p}{)}\PY{p}{;}
\PY{+w}{	}\PY{n}{scanf}\PY{p}{(}\PY{l+s}{\PYZdq{}}\PY{l+s}{\PYZpc{}s}\PY{l+s}{\PYZdq{}}\PY{p}{,}\PY{+w}{ }\PY{n}{ime}\PY{p}{)}\PY{p}{;}
\PY{+w}{	}\PY{n}{printf}\PY{p}{(}\PY{l+s}{\PYZdq{}}\PY{l+s}{Zivjo, \PYZpc{}s!}\PY{l+s+se}{\PYZbs{}n}\PY{l+s}{\PYZdq{}}\PY{p}{,}\PY{+w}{ }\PY{n}{ime}\PY{p}{)}\PY{p}{;}
\PY{+w}{	}\PY{k}{return}\PY{+w}{ }\PY{l+m+mi}{0}\PY{p}{;}
\PY{p}{\PYZcb{}}
\end{Verbatim}

\begin{inout}
{\color{blue} \bf output:} Kako ti je ime?\\
{\color{blue} \bf input:} Tinka \\
{\color{blue} \bf output:} Zivjo, Tinka! 
\end{inout}

\end{examples}


Če želimo, da program lahko kaj počne s podatki, ki jih je prebral, moramo najprej to shraniti na neko mesto v spominu. Temu mestu rečemo \emph{spremenljivka}, saj lahko s programom spreminjamo, kaj je tam shranjeno. Spremenljivke v svojem programu poimenujemo, v našem primeru ji rečemo \verb+ime+. Vsaki spremenljivki moramo določiti \emph{podatkovni tip}, saj lahko beremo in pišemo več različnih vrst podatkov, npr. besede ali števila. \\\\
\verb+char+ - s tem povemo, da je naša spremenljivka besedilo oz. \emph{niz} \\\\
\verb+[...]+ - številka v oglatih oklepajih za besedo pove največjo dolžino niza, ki ga lahko program prebere.

\begin{errors} %CHECK
Ko določamo največjo dolžino niza, vedno vzamemo večjo številko, kot jo bomo potrebovali. O tem bomo več govorili v kasnejših poglavjih.
\end{errors}


Funkciji \verb+scanf+ podamo dva ali več \emph{parametrov}:
\begin{itemize}
	\item kaj naj prebere, torej kakšne tipe spremenljivk. To podamo s \emph{formatnikom} (v našem primeru \verb+%s+, ki pomeni niz (\emph{string}). \verb+%s+ prebere vse znake do prvega presledka ali nove vrstice)
	\item ostali parametri povejo, kam naj funkcija shrani stvari, ki jih je prebrala (torej v spremenljivke, ki smo jih naredili prej)
\end{itemize}


%CHECK
Do zdaj smo funkciji \verb+printf+ podali samo točno določeno besedilo, ki smo ga želeli izpisati. Izpisujemo pa lahko tudi spremenljivke, kot smo to naredili v tem zadnjem primeru. Znotraj besedila dodamo formatnike na mesta, kjer želimo, da so spremenljivke, potem pa izven narekovajev naštejemo imena spremenljivk, ki jih želimo izpisati (tako kot pri funkciji \verb+scanf+).

\pagebreak
\section{Branje števil}
%CHECK
Poleg nizov lahko programi delajo tudi s števili. Števila beremo in izpisujemo z istima funkcijama kot nize, vendar moramo uporabiti drug formatnik.

\begin{examples}
\begin{Verbatim}[commandchars=\\\{\}]
\PY{c+cp}{\PYZsh{}}\PY{c+cp}{include}\PY{c+cpf}{\PYZlt{}stdio.h\PYZgt{}}

\PY{k+kt}{int}\PY{+w}{ }\PY{n+nf}{main}\PY{p}{(}\PY{p}{)}\PY{p}{\PYZob{}}
\PY{+w}{	}\PY{k+kt}{int}\PY{+w}{ }\PY{n}{razred}\PY{p}{;}
\PY{+w}{	}\PY{n}{printf}\PY{p}{(}\PY{l+s}{\PYZdq{}}\PY{l+s}{Kateri razred si?}\PY{l+s+se}{\PYZbs{}n}\PY{l+s}{\PYZdq{}}\PY{p}{)}\PY{p}{;}
\PY{+w}{	}\PY{n}{scanf}\PY{p}{(}\PY{l+s}{\PYZdq{}}\PY{l+s}{\PYZpc{}d}\PY{l+s}{\PYZdq{}}\PY{p}{,}\PY{+w}{ }\PY{o}{\PYZam{}}\PY{n}{razred}\PY{p}{)}\PY{p}{;}
\PY{+w}{	}\PY{n}{printf}\PY{p}{(}\PY{l+s}{\PYZdq{}}\PY{l+s}{\PYZpc{}d. razred je najboljši.}\PY{l+s+se}{\PYZbs{}n}\PY{l+s}{\PYZdq{}}\PY{p}{,}\PY{+w}{ }\PY{n}{razred}\PY{p}{)}\PY{p}{;}
\PY{+w}{	}\PY{k}{return}\PY{+w}{ }\PY{l+m+mi}{0}\PY{p}{;}
\PY{p}{\PYZcb{}}
\end{Verbatim}

\begin{inout}
{\color{blue} \bf output:} Kateri razred si?\\
{\color{blue} \bf input :} 7\\
{\color{blue} \bf output:} 7. razred je najboljši.
\end{inout}

\end{examples}

%CHECK
\verb+&+ - znak, ki ga moramo dati pred ime spremenljivke vedno, kadar beremo števila.
\verb+int+ - podatkovni tip število \\
Pri številih za imenom spremenljivke ne povemo, kako velika so lahko. \\\\
Za branje in pisanje števil uporabimo formatnik \verb+%d+.

\begin{errors}
Ko beremo števila, moramo pred ime spremenljivke dati znak \verb+&+, česar pri branju nizov ne delamo. Prav tako tega ne delamo pri izpisovanju števil.
\end{errors}

\end{document}
