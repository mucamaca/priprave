\documentclass{article}
\usepackage{fancyvrb}
\usepackage{color}
\usepackage[utf8]{inputenc}
\usepackage[a4paper, total={6.2in, 9in}]{geometry}
\usepackage[many]{tcolorbox}
\usepackage[T1]{fontenc}
\usepackage[slovene]{babel}

\usetikzlibrary{calc}

\setlength\parindent{0pt}

\makeatletter
\def\PY@reset{\let\PY@it=\relax \let\PY@bf=\relax%
    \let\PY@ul=\relax \let\PY@tc=\relax%
    \let\PY@bc=\relax \let\PY@ff=\relax}
\def\PY@tok#1{\csname PY@tok@#1\endcsname}
\def\PY@toks#1+{\ifx\relax#1\empty\else%
    \PY@tok{#1}\expandafter\PY@toks\fi}
\def\PY@do#1{\PY@bc{\PY@tc{\PY@ul{%
    \PY@it{\PY@bf{\PY@ff{#1}}}}}}}
\def\PY#1#2{\PY@reset\PY@toks#1+\relax+\PY@do{#2}}

\@namedef{PY@tok@c}{\def\PY@tc##1{\textcolor[rgb]{0.63,0.63,0.88}{##1}}}
\@namedef{PY@tok@cp}{\def\PY@tc##1{\textcolor[rgb]{0.19,0.51,0.25}{##1}}}
\@namedef{PY@tok@k}{\let\PY@bf=\textbf\def\PY@tc##1{\textcolor[rgb]{0.00,0.00,0.67}{##1}}}
\@namedef{PY@tok@o}{\def\PY@tc##1{\textcolor[rgb]{1.00,0.02,0.02}{##1}}}
\@namedef{PY@tok@p}{\def\PY@tc##1{\textcolor[rgb]{1.00,0.02,0.02}{##1}}}
\@namedef{PY@tok@n}{\def\PY@tc##1{\textcolor[rgb]{0.05,0.05,0.05}{##1}}}
\@namedef{PY@tok@s}{\def\PY@tc##1{\textcolor[rgb]{0.16,0.16,1.00}{##1}}}
\@namedef{PY@tok@m}{\def\PY@tc##1{\textcolor[rgb]{0.94,0.03,0.94}{##1}}}
\@namedef{PY@tok@kc}{\let\PY@bf=\textbf\def\PY@tc##1{\textcolor[rgb]{0.00,0.00,0.67}{##1}}}
\@namedef{PY@tok@kd}{\let\PY@bf=\textbf\def\PY@tc##1{\textcolor[rgb]{0.00,0.00,0.67}{##1}}}
\@namedef{PY@tok@kn}{\let\PY@bf=\textbf\def\PY@tc##1{\textcolor[rgb]{0.00,0.00,0.67}{##1}}}
\@namedef{PY@tok@kp}{\let\PY@bf=\textbf\def\PY@tc##1{\textcolor[rgb]{0.00,0.00,0.67}{##1}}}
\@namedef{PY@tok@kr}{\let\PY@bf=\textbf\def\PY@tc##1{\textcolor[rgb]{0.00,0.00,0.67}{##1}}}
\@namedef{PY@tok@kt}{\let\PY@bf=\textbf\def\PY@tc##1{\textcolor[rgb]{0.00,0.00,0.67}{##1}}}
\@namedef{PY@tok@na}{\def\PY@tc##1{\textcolor[rgb]{0.05,0.05,0.05}{##1}}}
\@namedef{PY@tok@nb}{\def\PY@tc##1{\textcolor[rgb]{0.05,0.05,0.05}{##1}}}
\@namedef{PY@tok@bp}{\def\PY@tc##1{\textcolor[rgb]{0.05,0.05,0.05}{##1}}}
\@namedef{PY@tok@nc}{\def\PY@tc##1{\textcolor[rgb]{0.05,0.05,0.05}{##1}}}
\@namedef{PY@tok@no}{\def\PY@tc##1{\textcolor[rgb]{0.05,0.05,0.05}{##1}}}
\@namedef{PY@tok@nd}{\def\PY@tc##1{\textcolor[rgb]{0.05,0.05,0.05}{##1}}}
\@namedef{PY@tok@ni}{\def\PY@tc##1{\textcolor[rgb]{0.05,0.05,0.05}{##1}}}
\@namedef{PY@tok@ne}{\def\PY@tc##1{\textcolor[rgb]{0.05,0.05,0.05}{##1}}}
\@namedef{PY@tok@nf}{\def\PY@tc##1{\textcolor[rgb]{0.05,0.05,0.05}{##1}}}
\@namedef{PY@tok@fm}{\def\PY@tc##1{\textcolor[rgb]{0.05,0.05,0.05}{##1}}}
\@namedef{PY@tok@py}{\def\PY@tc##1{\textcolor[rgb]{0.05,0.05,0.05}{##1}}}
\@namedef{PY@tok@nl}{\def\PY@tc##1{\textcolor[rgb]{0.05,0.05,0.05}{##1}}}
\@namedef{PY@tok@nn}{\def\PY@tc##1{\textcolor[rgb]{0.05,0.05,0.05}{##1}}}
\@namedef{PY@tok@nx}{\def\PY@tc##1{\textcolor[rgb]{0.05,0.05,0.05}{##1}}}
\@namedef{PY@tok@nt}{\def\PY@tc##1{\textcolor[rgb]{0.05,0.05,0.05}{##1}}}
\@namedef{PY@tok@nv}{\def\PY@tc##1{\textcolor[rgb]{0.05,0.05,0.05}{##1}}}
\@namedef{PY@tok@vc}{\def\PY@tc##1{\textcolor[rgb]{0.05,0.05,0.05}{##1}}}
\@namedef{PY@tok@vg}{\def\PY@tc##1{\textcolor[rgb]{0.05,0.05,0.05}{##1}}}
\@namedef{PY@tok@vi}{\def\PY@tc##1{\textcolor[rgb]{0.05,0.05,0.05}{##1}}}
\@namedef{PY@tok@vm}{\def\PY@tc##1{\textcolor[rgb]{0.05,0.05,0.05}{##1}}}
\@namedef{PY@tok@sa}{\def\PY@tc##1{\textcolor[rgb]{0.16,0.16,1.00}{##1}}}
\@namedef{PY@tok@sb}{\def\PY@tc##1{\textcolor[rgb]{0.16,0.16,1.00}{##1}}}
\@namedef{PY@tok@sc}{\def\PY@tc##1{\textcolor[rgb]{0.16,0.16,1.00}{##1}}}
\@namedef{PY@tok@dl}{\def\PY@tc##1{\textcolor[rgb]{0.16,0.16,1.00}{##1}}}
\@namedef{PY@tok@sd}{\def\PY@tc##1{\textcolor[rgb]{0.16,0.16,1.00}{##1}}}
\@namedef{PY@tok@s2}{\def\PY@tc##1{\textcolor[rgb]{0.16,0.16,1.00}{##1}}}
\@namedef{PY@tok@se}{\def\PY@tc##1{\textcolor[rgb]{0.16,0.16,1.00}{##1}}}
\@namedef{PY@tok@sh}{\def\PY@tc##1{\textcolor[rgb]{0.16,0.16,1.00}{##1}}}
\@namedef{PY@tok@si}{\def\PY@tc##1{\textcolor[rgb]{0.16,0.16,1.00}{##1}}}
\@namedef{PY@tok@sx}{\def\PY@tc##1{\textcolor[rgb]{0.16,0.16,1.00}{##1}}}
\@namedef{PY@tok@sr}{\def\PY@tc##1{\textcolor[rgb]{0.16,0.16,1.00}{##1}}}
\@namedef{PY@tok@s1}{\def\PY@tc##1{\textcolor[rgb]{0.16,0.16,1.00}{##1}}}
\@namedef{PY@tok@ss}{\def\PY@tc##1{\textcolor[rgb]{0.16,0.16,1.00}{##1}}}
\@namedef{PY@tok@mb}{\def\PY@tc##1{\textcolor[rgb]{0.94,0.03,0.94}{##1}}}
\@namedef{PY@tok@mf}{\def\PY@tc##1{\textcolor[rgb]{0.94,0.03,0.94}{##1}}}
\@namedef{PY@tok@mh}{\def\PY@tc##1{\textcolor[rgb]{0.94,0.03,0.94}{##1}}}
\@namedef{PY@tok@mi}{\def\PY@tc##1{\textcolor[rgb]{0.94,0.03,0.94}{##1}}}
\@namedef{PY@tok@il}{\def\PY@tc##1{\textcolor[rgb]{0.94,0.03,0.94}{##1}}}
\@namedef{PY@tok@mo}{\def\PY@tc##1{\textcolor[rgb]{0.94,0.03,0.94}{##1}}}
\@namedef{PY@tok@ow}{\def\PY@tc##1{\textcolor[rgb]{1.00,0.02,0.02}{##1}}}
\@namedef{PY@tok@ch}{\def\PY@tc##1{\textcolor[rgb]{0.63,0.63,0.88}{##1}}}
\@namedef{PY@tok@cm}{\def\PY@tc##1{\textcolor[rgb]{0.63,0.63,0.88}{##1}}}
\@namedef{PY@tok@cpf}{\def\PY@tc##1{\textcolor[rgb]{0.63,0.63,0.88}{##1}}}
\@namedef{PY@tok@c1}{\def\PY@tc##1{\textcolor[rgb]{0.63,0.63,0.88}{##1}}}
\@namedef{PY@tok@cs}{\def\PY@tc##1{\textcolor[rgb]{0.63,0.63,0.88}{##1}}}

\def\PYZbs{\char`\\}
\def\PYZus{\char`\_}
\def\PYZob{\char`\{}
\def\PYZcb{\char`\}}
\def\PYZca{\char`\^}
\def\PYZam{\char`\&}
\def\PYZlt{\char`\<}
\def\PYZgt{\char`\>}
\def\PYZsh{\char`\#}
\def\PYZpc{\char`\%}
\def\PYZdl{\char`\$}
\def\PYZhy{\char`\-}
\def\PYZsq{\char`\'}
\def\PYZdq{\char`\"}
\def\PYZti{\char`\~}
% for compatibility with earlier versions
\def\PYZat{@}
\def\PYZlb{[}
\def\PYZrb{]}
\makeatother

\definecolor{myblue}{RGB}{0,163,243}
\definecolor{myred}{RGB}{243, 10, 25}
\definecolor{mygreen}{RGB}{50, 205, 50}

\newcommand{\fon}[1]{\fontfamily{#1}\selectfont}


\tcbset{examplestyle/.style={
  enhanced,
  outer arc=4pt,
  arc=4pt,
  colframe=myblue,
  colback=myblue!20,
  attach boxed title to top left,
  boxed title style={
    colback=myblue,
    outer arc=4pt,
    arc=4pt,
    top=3pt,
    bottom=3pt,
    },
  fonttitle=\sffamily
  }
}

\tcbset{inoutstyle/.style={
  enhanced,
  outer arc=4pt,
  arc=4pt,
  colframe=mygreen,
  colback=mygreen!20,
  attach boxed title to top left,
  boxed title style={
    colback=mygreen,
    outer arc=4pt,
    arc=4pt,
    top=3pt,
    bottom=3pt,
    },
  fonttitle=\sffamily,
  fontupper=\ttfamily,
  fontlower=\ttfamily,
  }
}

\tcbset{errorstyle/.style={
  enhanced,
  outer arc=4pt,
  arc=4pt,
  colframe=myred,
  colback=myred!20,
  attach boxed title to top left,
  boxed title style={
    colback=myred,
    outer arc=4pt,
    arc=4pt,
    top=3pt,
    bottom=3pt,
    },
  fonttitle=\sffamily
  }
}

\newtcolorbox[auto counter,number within=section]{examples}[1][]{
  examplestyle,
  colback=white,
  title=Primer,
  overlay unbroken and first={
      \path
        let
        \p1=(title.north east),
        \p2=(frame.north east)
        in
        node[anchor=west,font=\sffamily,color=myblue,text width=\x2-\x1]
        at (title.east) {#1};
  }
}
\newtcolorbox[auto counter]{errors}[1][]{
  errorstyle,
  colback=white,
  title=Pogoste napake,
  overlay unbroken and first={
      \path
        let
        \p1=(title.north east),
        \p2=(frame.north east)
        in
        node[anchor=west,font=\sffamily,color=myblue,text width=\x2-\x1]
        at (title.east) {#1};
  }
}

\newtcolorbox[auto counter]{inout}[1][]{
  inoutstyle,
  colback=white,
  title=Primer vhoda in izhoda,
  overlay unbroken and first={
      \path
        let
        \p1=(title.north east),
        \p2=(frame.north east)
        in
        node[anchor=west,font=\sffamily,color=myblue,text width=\x2-\x1]
        at (title.east) {#1};
  }
}
\title{For zanka}
\date{}

\begin{document}
\maketitle

\section{Kako napišemo zanko?}

V programiranju pogosto želimo nek del kode ponoviti, zato imamo \emph{zanke}.
Zanka, ki jo bomo najpogosteje uporabljali je \emph{for} zanka,
ki deluje na princip \textbf{začetka}, \textbf{pogoja} in \textbf{koraka}.
\begin{examples}
\verb+Najbolj osnoven program, ki uporablja for zanko.+

\begin{Verbatim}[commandchars=\\\{\}]
\PY{c+cp}{\PYZsh{}}\PY{c+cp}{include}\PY{+w}{ }\PY{c+cpf}{\PYZlt{}stdio.h\PYZgt{}}
	
\PY{k+kt}{int}\PY{+w}{ }\PY{n+nf}{main}\PY{p}{(}\PY{p}{)}\PY{+w}{ }\PY{p}{\PYZob{}}
	
\PY{+w}{    }\PY{c+c1}{//     začetek       pogoj       korak}
\PY{+w}{    }\PY{k}{for}\PY{+w}{ }\PY{p}{(}\PY{k+kt}{int}\PY{+w}{ }\PY{n}{stevec}\PY{o}{=}\PY{l+m+mi}{0}\PY{p}{;}\PY{+w}{ }\PY{n}{stevec}\PY{+w}{ }\PY{o}{\PYZlt{}}\PY{+w}{ }\PY{l+m+mi}{3}\PY{p}{;}\PY{+w}{ }\PY{n}{stevec}\PY{o}{+}\PY{o}{+}\PY{p}{)}\PY{+w}{ }\PY{p}{\PYZob{}}
\PY{+w}{        }\PY{c+c1}{// koda, ki se izvedce vsako zanko}
\PY{+w}{        }\PY{n}{printf}\PY{p}{(}\PY{l+s}{\PYZdq{}}\PY{l+s}{nekaj}\PY{l+s+se}{\PYZbs{}n}\PY{l+s}{\PYZdq{}}\PY{p}{)}\PY{p}{;}
\PY{+w}{    }\PY{p}{\PYZcb{}}

\PY{+w}{    }\PY{k}{return}\PY{+w}{ }\PY{l+m+mi}{0}\PY{p}{;}
\PY{p}{\PYZcb{}}
\end{Verbatim}

\begin{inout}
	nekaj \\
	nekaj \\
	nekaj
\end{inout}

\end{examples}

Poglejmo si, kako ta program deluje. Podpičja v vrstici \\
\texttt{for~({\textcolor{red}{int~stevec=0}};~\textcolor{blue}{stevec~<~3};~\textcolor{purple}{stevec++})~\{} \\
razdelijo okrogle oklepaje na tri dele; \textcolor{red}{začetek},
\textcolor{blue}{pogoj} in \textcolor{purple}{korak}.
Začetek se bo izvedel, ko se ta zanka začne. Vsakič preden se izvede koda
v notranjosti zanke se preveri pogoj, če drži,
se bo še enkrat izvedla koda v zanki, sicer se bo pa zanka končala.
Korak je podoben začetku, le da se izvede na koncu vsake ponovitve zanke.

Tu začetek naredi novo številko $stevec$ in jo nastavi na $0$.
Pogoj preveri, če je $stevec$ manjši od $3$. Korak \texttt{stevec++} je pa
okrajšava za \texttt{stevec = stevec + 1}, torej poveča $stevec$ za $1$.

\begin{itemize}
	\item Program se začne in pride do for zanke,
	najprej se izvede začetek \texttt{int~stevec=0}.
	\item Zdaj se je začela zanka, preveri se pogoj \texttt{stevec~<~3}.
	Ker je $stevec$ za zdaj še $0$, je pogoj izpolnjen. Izvede se vsebina zanke,
	torej program izpiše \emph{nekaj}. Zdaj smo prišli do konca zanke,
	izvede se korak, $stevec$ se poveča na $1$, program pa skoči nazaj na
	začetek zanke.
	\item Ker smo na začetku zanke se preveri pogoj, \texttt{stevec~<~3},
	ker je $stevec$ zdaj $1$ je pogoj še vedno izpolnjen, zato se izvede
	vsebina zanke. Ko program še enkrat izpiše \emph{nekaj} izvede korak,
	$stevec$ poveča na $2$ in skoči nazaj na začetek.
	\item Spet smo na začetku, zato se preveri pogoj, $stevec$ je zdaj $2$,
	kar je manjše od $3$, zato se \emph{nekaj} spet izpiše. Program poveča
	$stevec$ na $3$ in skoči nazaj na začetek.
	\item Ker smo spet na začetku se bo še enkrat preveril pogoj,
	a zdaj je $stevec$ enak $3$ in $3$ ni manjše od $3$,
	zato se for zanka konča. Ker je naslednji ukaz \texttt{return 0;} se bo
	program tam končal.
\end{itemize}

Vidimo, da bo res izpisal \emph{nekaj} trikrat. Vredno je omeniti,
da je naš števec zavzel vrednosti 0, 1, 2, kar se mogoče zdi čudno, glede na
to, da bi ponavadi šteli do tri kot 1, 2, 3, a je \emph{štetje od nič} zelo
pogosto v programiranju, zato smo se odločili, da vas že od začetka poskušamo
navaditi nanj.

\section{Razni primeri uporabe for zanke}

\subsection{Spreminjanje dolžine for zanke}
Zanka, ki smo jo napisali zgoraj, se bo vedno ponovila trikrat,
kaj pa če hočemo da se zanka ponovi glede na neko število na vhodu?
Seveda je tudi to mogoče in sicer tako, da vstavimo našo spremenljivko v pogoj
for zanke. Poglejmo si primer:

\begin{examples}
\verb+Program, ki dobi število in nariše puščico te dolžine.+

\verb+Še en trik tu je, da printf-ja v for zanki ne končamo z \n, kar doseže+

\verb+to, da so v izhodu pomišljaji eden zraven drugega v isti vrstici in ne+

\verb+vsak v svoji vrstici.+

\begin{Verbatim}[commandchars=\\\{\}]
	\PY{c+cp}{\PYZsh{}}\PY{c+cp}{include}\PY{+w}{ }\PY{c+cpf}{\PYZlt{}stdio.h\PYZgt{}}
	
	\PY{k+kt}{int}\PY{+w}{ }\PY{n+nf}{main}\PY{p}{(}\PY{p}{)}\PY{+w}{ }\PY{p}{\PYZob{}}
	
	\PY{+w}{    }\PY{k+kt}{int}\PY{+w}{ }\PY{n}{dolzina}\PY{p}{;}
	\PY{+w}{    }\PY{n}{scanf}\PY{p}{(}\PY{l+s}{\PYZdq{}}\PY{l+s}{\PYZpc{}d}\PY{l+s}{\PYZdq{}}\PY{p}{,}\PY{+w}{ }\PY{o}{\PYZam{}}\PY{n}{dolzina}\PY{p}{)}\PY{p}{;}
	
	\PY{+w}{    }\PY{k}{for}\PY{+w}{ }\PY{p}{(}\PY{k+kt}{int}\PY{+w}{ }\PY{n}{stevec}\PY{o}{=}\PY{l+m+mi}{0}\PY{p}{;}\PY{+w}{ }\PY{n}{stevec}\PY{+w}{ }\PY{o}{\PYZlt{}}\PY{+w}{ }\PY{n}{dolzina}\PY{p}{;}\PY{+w}{ }\PY{n}{stevec}\PY{o}{+}\PY{o}{+}\PY{p}{)}\PY{+w}{ }\PY{p}{\PYZob{}}
	\PY{+w}{        }\PY{n}{printf}\PY{p}{(}\PY{l+s}{\PYZdq{}}\PY{l+s}{\PYZhy{}}\PY{l+s}{\PYZdq{}}\PY{p}{)}\PY{p}{;}
	\PY{+w}{    }\PY{p}{\PYZcb{}}
	
	\PY{+w}{    }\PY{n}{printf}\PY{p}{(}\PY{l+s}{\PYZdq{}}\PY{l+s}{\PYZgt{}}\PY{l+s+se}{\PYZbs{}n}\PY{l+s}{\PYZdq{}}\PY{p}{)}\PY{p}{;}
	
	\PY{+w}{    }\PY{k}{return}\PY{+w}{ }\PY{l+m+mi}{0}\PY{p}{;}
	\PY{p}{\PYZcb{}}
\end{Verbatim}

\begin{inout}
	4
	\tcblower
	-{}-{}-{}->
\end{inout}

\end{examples}

\subsection{Branje števil v for zanki}

Ena od moči računalnikov je zelo hitra obdelava velike količine podatkov,
računalnik bo zlahka seštel 1000 števil, medtem ko bi bilo to početi na roko
precej zamudno. Poglejmo si, kako bi napisali program, ki bi nekaj izračunal z
več števili.

\begin{examples}
\verb+Najprej preberemo eno število, recimo mu n.+
	
\verb+Po njemu moramo prebrati še n števili in jih sešteti.+
	
\begin{Verbatim}[commandchars=\\\{\}]
	\PY{c+cp}{\PYZsh{}}\PY{c+cp}{include}\PY{+w}{ }\PY{c+cpf}{\PYZlt{}stdio.h\PYZgt{}}
		
	\PY{k+kt}{int}\PY{+w}{ }\PY{n+nf}{main}\PY{p}{(}\PY{p}{)}\PY{+w}{ }\PY{p}{\PYZob{}}
		
	\PY{+w}{    }\PY{k+kt}{int}\PY{+w}{ }\PY{n}{n}\PY{p}{;}
	\PY{+w}{    }\PY{n}{scanf}\PY{p}{(}\PY{l+s}{\PYZdq{}}\PY{l+s}{\PYZpc{}d}\PY{l+s}{\PYZdq{}}\PY{p}{,}\PY{+w}{ }\PY{o}{\PYZam{}}\PY{n}{n}\PY{p}{)}\PY{p}{;}
		
	\PY{+w}{    }\PY{k+kt}{int}\PY{+w}{ }\PY{n}{vsota}\PY{+w}{ }\PY{o}{=}\PY{+w}{ }\PY{l+m+mi}{0}\PY{p}{;}
		
	\PY{+w}{    }\PY{k}{for}\PY{+w}{ }\PY{p}{(}\PY{k+kt}{int}\PY{+w}{ }\PY{n}{i}\PY{o}{=}\PY{l+m+mi}{0}\PY{p}{;}\PY{+w}{ }\PY{n}{i}\PY{+w}{ }\PY{o}{\PYZlt{}}\PY{+w}{ }\PY{n}{n}\PY{p}{;}\PY{+w}{ }\PY{n}{i}\PY{o}{+}\PY{o}{+}\PY{p}{)}\PY{+w}{ }\PY{p}{\PYZob{}}
	\PY{+w}{        }\PY{k+kt}{int}\PY{+w}{ }\PY{n}{sestevanec}\PY{p}{;}
	\PY{+w}{        }\PY{n}{scanf}\PY{p}{(}\PY{l+s}{\PYZdq{}}\PY{l+s}{\PYZpc{}d}\PY{l+s}{\PYZdq{}}\PY{p}{,}\PY{+w}{ }\PY{o}{\PYZam{}}\PY{n}{sestevanec}\PY{p}{)}\PY{p}{;}
	\PY{+w}{        }\PY{n}{vsota}\PY{+w}{ }\PY{o}{+}\PY{o}{=}\PY{+w}{ }\PY{n}{sestevanec}\PY{p}{;}
	\PY{+w}{    }\PY{p}{\PYZcb{}}
		
	\PY{+w}{    }\PY{n}{printf}\PY{p}{(}\PY{l+s}{\PYZdq{}}\PY{l+s}{\PYZpc{}d}\PY{l+s+se}{\PYZbs{}n}\PY{l+s}{\PYZdq{}}\PY{p}{,}\PY{+w}{ }\PY{n}{vsota}\PY{p}{)}\PY{p}{;}
		
	\PY{+w}{    }\PY{k}{return}\PY{+w}{ }\PY{l+m+mi}{0}\PY{p}{;}
	\PY{p}{\PYZcb{}}
\end{Verbatim}
	
\begin{inout}
	4  \\
	12 \\
	13 \\
	8  \\
	1
	\tcblower
	34
\end{inout}
	
\end{examples}

V zgornjem primeru se je zgodilo precej novih stvari, poglejmo si vse po vrsti.

Najprej preberemo n iz navodil. Naredimo novo spremenljivko z imenom $vsota$, v njo
bomo sešteli vsa dana števila.

Opazimo, da je v for zanki namesto $stevec$ zdaj uporabljen $i$, tradicionalno
se namreč v for zankah uporablja $i$. 
V notranjosti zanke so zdaj trije ukazi. Najprej naredimo novo
spremenljivko, ki jo poimenujemo $sestevanec$ in preberemo naslednje
število iz vhoda. Nato pa z okrajšavo \texttt{vsota~+=~sestevanec;} prištejemo
spremenljivki $vsota$ spremenljivko $sestevanec$. Na daljše bi to lahko napisali
\texttt{vsota~=~vsota~+~sestevanec}.

\subsection{While zanka in branje neznano mnogo števil}

Naučili smo se, kako se prebere nekaj števil, ko nam je podano koliko števil
moramo prebrati. Kaj pa če tega ne vemo, če hočemo pač prebrati vsa števila,
ki so na vhodu? Zato lahko uporabimo \emph{while} zanko, ki deluje tako kot
for zanka, le da ima samo pogoj.

\begin{examples}
\verb+Preperimo vsa števila in jih zmnožimo.+
	
\begin{Verbatim}[commandchars=\\\{\}]
	\PY{c+cp}{\PYZsh{}}\PY{c+cp}{include}\PY{+w}{ }\PY{c+cpf}{\PYZlt{}stdio.h\PYZgt{}}
	
	\PY{k+kt}{int}\PY{+w}{ }\PY{n+nf}{main}\PY{p}{(}\PY{p}{)}\PY{+w}{ }\PY{p}{\PYZob{}}
	
	\PY{+w}{    }\PY{k+kt}{int}\PY{+w}{ }\PY{n}{produkt}\PY{+w}{ }\PY{o}{=}\PY{+w}{ }\PY{l+m+mi}{1}\PY{p}{;}
	
	\PY{+w}{    }\PY{k+kt}{int}\PY{+w}{ }\PY{n}{clen}\PY{p}{;}
	
	\PY{+w}{    }\PY{k}{while}\PY{+w}{ }\PY{p}{(}\PY{n}{scanf}\PY{p}{(}\PY{l+s}{\PYZdq{}}\PY{l+s}{\PYZpc{}d}\PY{l+s}{\PYZdq{}}\PY{p}{,}\PY{+w}{ }\PY{o}{\PYZam{}}\PY{n}{clen}\PY{p}{)}\PY{+w}{ }\PY{o}{=}\PY{o}{=}\PY{+w}{ }\PY{l+m+mi}{1}\PY{p}{)}\PY{+w}{ }\PY{p}{\PYZob{}}
	\PY{+w}{        }\PY{n}{produkt}\PY{+w}{ }\PY{o}{=}\PY{+w}{ }\PY{n}{produkt}\PY{+w}{ }\PY{o}{*}\PY{+w}{ }\PY{n}{clen}\PY{p}{;}
	\PY{+w}{    }\PY{p}{\PYZcb{}}
	
	\PY{+w}{    }\PY{n}{printf}\PY{p}{(}\PY{l+s}{\PYZdq{}}\PY{l+s}{\PYZpc{}d}\PY{l+s+se}{\PYZbs{}n}\PY{l+s}{\PYZdq{}}\PY{p}{,}\PY{+w}{ }\PY{n}{produkt}\PY{p}{)}\PY{p}{;}
	
	\PY{+w}{    }\PY{k}{return}\PY{+w}{ }\PY{l+m+mi}{0}\PY{p}{;}
	\PY{p}{\PYZcb{}}
\end{Verbatim}
	
\begin{inout}
	2  \\
	3 \\
	4 \\
	5
	\tcblower
	120
\end{inout}
	
\end{examples}

Kot že povedano, zgoraj napisani stavek while bi enako napisali kot
\texttt{for~(~;~scanf("\%d")~==~1~;~)}, torej for zanka, ki ima samo pogoj.
Mogoče se zdi čudno, da ukaz scanf \emph{primerjamo} s številko, a bomo kasneje
v letu, ko se bomo učili o funkcijah razumeli, kaj to pomeni. Za zdaj pa
bomo rekli le, da to, da je scanf \emph{enak} ena pomeni, da je uspešno prebral
in shranil eno številko. Če bi zgornji program prebral v človeških besedah, bi
bilo:
\begin{itemize}
	\item Nastavi $produkt$ na 1
	\item Naredi novo spremenljivko $clen$
	\item Preberi eno število in ga shrani v $clen$, dokler tega ne moreš
	narediti več.
	\item Nastavi $produkt$ na $produkt * clen$
	\item Ko končaš s množenjem vseh členov, izpiši rezultat.
\end{itemize}

\begin{errors}
\verb+Tu na začetku nastavimo produkt na 0, kar pa je napaka, saj je 0 krat+

\verb+karkoli še vedno 0. Naš program bo veno izpisal 0 ne glede na vhod.+

\begin{Verbatim}[commandchars=\\\{\}]
	\PY{c+cp}{\PYZsh{}}\PY{c+cp}{include}\PY{+w}{ }\PY{c+cpf}{\PYZlt{}stdio.h\PYZgt{}}
	
	\PY{k+kt}{int}\PY{+w}{ }\PY{n+nf}{main}\PY{p}{(}\PY{p}{)}\PY{+w}{ }\PY{p}{\PYZob{}}
	
	\PY{+w}{    }\PY{k+kt}{int}\PY{+w}{ }\PY{n}{produkt}\PY{+w}{ }\PY{o}{=}\PY{+w}{ }\PY{l+m+mi}{0}\PY{p}{;}
	
	\PY{+w}{    }\PY{k+kt}{int}\PY{+w}{ }\PY{n}{clen}\PY{p}{;}
	
	\PY{+w}{    }\PY{k}{while}\PY{+w}{ }\PY{p}{(}\PY{n}{scanf}\PY{p}{(}\PY{l+s}{\PYZdq{}}\PY{l+s}{\PYZpc{}d}\PY{l+s}{\PYZdq{}}\PY{p}{,}\PY{+w}{ }\PY{o}{\PYZam{}}\PY{n}{clen}\PY{p}{)}\PY{+w}{ }\PY{o}{=}\PY{o}{=}\PY{+w}{ }\PY{l+m+mi}{1}\PY{p}{)}\PY{+w}{ }\PY{p}{\PYZob{}}
	\PY{+w}{        }\PY{n}{produkt}\PY{+w}{ }\PY{o}{=}\PY{+w}{ }\PY{n}{produkt}\PY{+w}{ }\PY{o}{*}\PY{+w}{ }\PY{n}{clen}\PY{p}{;}
	\PY{+w}{    }\PY{p}{\PYZcb{}}
	
	\PY{+w}{    }\PY{n}{printf}\PY{p}{(}\PY{l+s}{\PYZdq{}}\PY{l+s}{\PYZpc{}d}\PY{l+s+se}{\PYZbs{}n}\PY{l+s}{\PYZdq{}}\PY{p}{,}\PY{+w}{ }\PY{n}{produkt}\PY{p}{)}\PY{p}{;}
	\PY{+w}{    }
	\PY{+w}{    }\PY{k}{return}\PY{+w}{ }\PY{l+m+mi}{0}\PY{p}{;}
	\PY{p}{\PYZcb{}}
\end{Verbatim}

\end{errors}

\subsection{For zanka z drugačnim korakom in začetkom}

Do zdaj so vse naše for zanke izgledale nekako tako
\texttt{for~(int~i=0;~i~<~10;~i++)}, torej so začele na nič in se nekajkrat
ponovile. Ampak zapis for zanke, ki ga imamo v c++ lahko naredi veliko več.
Poglejmo kot primer zanko, ki izpiše vsa soda števila med $1$ in $100$.

Pozorno poglejmo števila, ki jih moramo izpisati. Ker $1$ ni sodo, bo prvo
izpisano število $2$. Število $3$ prav tako ni sodo, tako da bomo izpisali $4$,
po tem pa $6$, $8$, $10$ in tako dalje. Vidimo, da vsak korak povečamo izpisano
število za $2$, napišimo torej program.

\begin{examples}
\verb+Izpišemo vsa soda števila med 1 in 100.+
	
\begin{Verbatim}[commandchars=\\\{\}]
	\PY{c+cp}{\PYZsh{}}\PY{c+cp}{include}\PY{+w}{ }\PY{c+cpf}{\PYZlt{}stdio.h\PYZgt{}}
	
	\PY{k+kt}{int}\PY{+w}{ }\PY{n+nf}{main}\PY{p}{(}\PY{p}{)}\PY{+w}{ }\PY{p}{\PYZob{}}
	
	\PY{+w}{    }\PY{k}{for}\PY{+w}{ }\PY{p}{(}\PY{k+kt}{int}\PY{+w}{ }\PY{n}{i}\PY{o}{=}\PY{l+m+mi}{2}\PY{p}{;}\PY{+w}{ }\PY{n}{i}\PY{o}{\PYZlt{}}\PY{o}{=}\PY{l+m+mi}{100}\PY{p}{;}\PY{+w}{ }\PY{n}{i}\PY{+w}{ }\PY{o}{+}\PY{o}{=}\PY{+w}{ }\PY{l+m+mi}{2}\PY{p}{)}\PY{+w}{ }\PY{p}{\PYZob{}}
	\PY{+w}{        }\PY{n}{printf}\PY{p}{(}\PY{l+s}{\PYZdq{}}\PY{l+s}{\PYZpc{}d}\PY{l+s+se}{\PYZbs{}n}\PY{l+s}{\PYZdq{}}\PY{p}{,}\PY{+w}{ }\PY{n}{i}\PY{p}{)}\PY{p}{;}
	\PY{+w}{    }\PY{p}{\PYZcb{}}
	
	\PY{+w}{    }\PY{k}{return}\PY{+w}{ }\PY{l+m+mi}{0}\PY{p}{;}
	\PY{p}{\PYZcb{}}
\end{Verbatim}
	
\begin{inout}
	2  \\
	4  \\
	6  \\
	8  \\
	10 \\
	12 \\
	$\vdots$ \\
	96 \\
	98 \\
	100
\end{inout}
	
\end{examples}

Ker se želena števila začnejo z dva, bomo v začetni del for zanke vpisali
\texttt{int i=2}. Ker hočemo izpisati števila med $1$ in $100$ in ne med $1$ in
$99$ bomo v pogojnem delu uporabili znak manjše ali enako \texttt{i<=100} 
(pogoj \texttt{i<100} ne bi veljal za število $100$). Ker želimo povečati naše 
število za $2$ vsak korak, smo v polje za korak napisali \texttt{i += 2}. 
Edina stvar, ki jo naredimo v notranjosti napisane zanke pa je, da izpišemo 
trenutno vrednost spremenljivke $i$.


\end{document}
