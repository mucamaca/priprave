\documentclass{article}
\usepackage{fancyvrb}
\usepackage{color}
\usepackage[utf8]{inputenc}
\usepackage[a4paper, total={6.2in, 9in}]{geometry}
\usepackage[many]{tcolorbox}
\usepackage[T1]{fontenc}
\usepackage[slovene]{babel}

\usetikzlibrary{calc}

\setlength\parindent{0pt}

\makeatletter
\def\PY@reset{\let\PY@it=\relax \let\PY@bf=\relax%
    \let\PY@ul=\relax \let\PY@tc=\relax%
    \let\PY@bc=\relax \let\PY@ff=\relax}
\def\PY@tok#1{\csname PY@tok@#1\endcsname}
\def\PY@toks#1+{\ifx\relax#1\empty\else%
    \PY@tok{#1}\expandafter\PY@toks\fi}
\def\PY@do#1{\PY@bc{\PY@tc{\PY@ul{%
    \PY@it{\PY@bf{\PY@ff{#1}}}}}}}
\def\PY#1#2{\PY@reset\PY@toks#1+\relax+\PY@do{#2}}

\@namedef{PY@tok@c}{\def\PY@tc##1{\textcolor[rgb]{0.63,0.63,0.88}{##1}}}
\@namedef{PY@tok@cp}{\def\PY@tc##1{\textcolor[rgb]{0.19,0.51,0.25}{##1}}}
\@namedef{PY@tok@k}{\let\PY@bf=\textbf\def\PY@tc##1{\textcolor[rgb]{0.00,0.00,0.67}{##1}}}
\@namedef{PY@tok@o}{\def\PY@tc##1{\textcolor[rgb]{1.00,0.02,0.02}{##1}}}
\@namedef{PY@tok@p}{\def\PY@tc##1{\textcolor[rgb]{1.00,0.02,0.02}{##1}}}
\@namedef{PY@tok@n}{\def\PY@tc##1{\textcolor[rgb]{0.05,0.05,0.05}{##1}}}
\@namedef{PY@tok@s}{\def\PY@tc##1{\textcolor[rgb]{0.16,0.16,1.00}{##1}}}
\@namedef{PY@tok@m}{\def\PY@tc##1{\textcolor[rgb]{0.94,0.03,0.94}{##1}}}
\@namedef{PY@tok@kc}{\let\PY@bf=\textbf\def\PY@tc##1{\textcolor[rgb]{0.00,0.00,0.67}{##1}}}
\@namedef{PY@tok@kd}{\let\PY@bf=\textbf\def\PY@tc##1{\textcolor[rgb]{0.00,0.00,0.67}{##1}}}
\@namedef{PY@tok@kn}{\let\PY@bf=\textbf\def\PY@tc##1{\textcolor[rgb]{0.00,0.00,0.67}{##1}}}
\@namedef{PY@tok@kp}{\let\PY@bf=\textbf\def\PY@tc##1{\textcolor[rgb]{0.00,0.00,0.67}{##1}}}
\@namedef{PY@tok@kr}{\let\PY@bf=\textbf\def\PY@tc##1{\textcolor[rgb]{0.00,0.00,0.67}{##1}}}
\@namedef{PY@tok@kt}{\let\PY@bf=\textbf\def\PY@tc##1{\textcolor[rgb]{0.00,0.00,0.67}{##1}}}
\@namedef{PY@tok@na}{\def\PY@tc##1{\textcolor[rgb]{0.05,0.05,0.05}{##1}}}
\@namedef{PY@tok@nb}{\def\PY@tc##1{\textcolor[rgb]{0.05,0.05,0.05}{##1}}}
\@namedef{PY@tok@bp}{\def\PY@tc##1{\textcolor[rgb]{0.05,0.05,0.05}{##1}}}
\@namedef{PY@tok@nc}{\def\PY@tc##1{\textcolor[rgb]{0.05,0.05,0.05}{##1}}}
\@namedef{PY@tok@no}{\def\PY@tc##1{\textcolor[rgb]{0.05,0.05,0.05}{##1}}}
\@namedef{PY@tok@nd}{\def\PY@tc##1{\textcolor[rgb]{0.05,0.05,0.05}{##1}}}
\@namedef{PY@tok@ni}{\def\PY@tc##1{\textcolor[rgb]{0.05,0.05,0.05}{##1}}}
\@namedef{PY@tok@ne}{\def\PY@tc##1{\textcolor[rgb]{0.05,0.05,0.05}{##1}}}
\@namedef{PY@tok@nf}{\def\PY@tc##1{\textcolor[rgb]{0.05,0.05,0.05}{##1}}}
\@namedef{PY@tok@fm}{\def\PY@tc##1{\textcolor[rgb]{0.05,0.05,0.05}{##1}}}
\@namedef{PY@tok@py}{\def\PY@tc##1{\textcolor[rgb]{0.05,0.05,0.05}{##1}}}
\@namedef{PY@tok@nl}{\def\PY@tc##1{\textcolor[rgb]{0.05,0.05,0.05}{##1}}}
\@namedef{PY@tok@nn}{\def\PY@tc##1{\textcolor[rgb]{0.05,0.05,0.05}{##1}}}
\@namedef{PY@tok@nx}{\def\PY@tc##1{\textcolor[rgb]{0.05,0.05,0.05}{##1}}}
\@namedef{PY@tok@nt}{\def\PY@tc##1{\textcolor[rgb]{0.05,0.05,0.05}{##1}}}
\@namedef{PY@tok@nv}{\def\PY@tc##1{\textcolor[rgb]{0.05,0.05,0.05}{##1}}}
\@namedef{PY@tok@vc}{\def\PY@tc##1{\textcolor[rgb]{0.05,0.05,0.05}{##1}}}
\@namedef{PY@tok@vg}{\def\PY@tc##1{\textcolor[rgb]{0.05,0.05,0.05}{##1}}}
\@namedef{PY@tok@vi}{\def\PY@tc##1{\textcolor[rgb]{0.05,0.05,0.05}{##1}}}
\@namedef{PY@tok@vm}{\def\PY@tc##1{\textcolor[rgb]{0.05,0.05,0.05}{##1}}}
\@namedef{PY@tok@sa}{\def\PY@tc##1{\textcolor[rgb]{0.16,0.16,1.00}{##1}}}
\@namedef{PY@tok@sb}{\def\PY@tc##1{\textcolor[rgb]{0.16,0.16,1.00}{##1}}}
\@namedef{PY@tok@sc}{\def\PY@tc##1{\textcolor[rgb]{0.16,0.16,1.00}{##1}}}
\@namedef{PY@tok@dl}{\def\PY@tc##1{\textcolor[rgb]{0.16,0.16,1.00}{##1}}}
\@namedef{PY@tok@sd}{\def\PY@tc##1{\textcolor[rgb]{0.16,0.16,1.00}{##1}}}
\@namedef{PY@tok@s2}{\def\PY@tc##1{\textcolor[rgb]{0.16,0.16,1.00}{##1}}}
\@namedef{PY@tok@se}{\def\PY@tc##1{\textcolor[rgb]{0.16,0.16,1.00}{##1}}}
\@namedef{PY@tok@sh}{\def\PY@tc##1{\textcolor[rgb]{0.16,0.16,1.00}{##1}}}
\@namedef{PY@tok@si}{\def\PY@tc##1{\textcolor[rgb]{0.16,0.16,1.00}{##1}}}
\@namedef{PY@tok@sx}{\def\PY@tc##1{\textcolor[rgb]{0.16,0.16,1.00}{##1}}}
\@namedef{PY@tok@sr}{\def\PY@tc##1{\textcolor[rgb]{0.16,0.16,1.00}{##1}}}
\@namedef{PY@tok@s1}{\def\PY@tc##1{\textcolor[rgb]{0.16,0.16,1.00}{##1}}}
\@namedef{PY@tok@ss}{\def\PY@tc##1{\textcolor[rgb]{0.16,0.16,1.00}{##1}}}
\@namedef{PY@tok@mb}{\def\PY@tc##1{\textcolor[rgb]{0.94,0.03,0.94}{##1}}}
\@namedef{PY@tok@mf}{\def\PY@tc##1{\textcolor[rgb]{0.94,0.03,0.94}{##1}}}
\@namedef{PY@tok@mh}{\def\PY@tc##1{\textcolor[rgb]{0.94,0.03,0.94}{##1}}}
\@namedef{PY@tok@mi}{\def\PY@tc##1{\textcolor[rgb]{0.94,0.03,0.94}{##1}}}
\@namedef{PY@tok@il}{\def\PY@tc##1{\textcolor[rgb]{0.94,0.03,0.94}{##1}}}
\@namedef{PY@tok@mo}{\def\PY@tc##1{\textcolor[rgb]{0.94,0.03,0.94}{##1}}}
\@namedef{PY@tok@ow}{\def\PY@tc##1{\textcolor[rgb]{1.00,0.02,0.02}{##1}}}
\@namedef{PY@tok@ch}{\def\PY@tc##1{\textcolor[rgb]{0.63,0.63,0.88}{##1}}}
\@namedef{PY@tok@cm}{\def\PY@tc##1{\textcolor[rgb]{0.63,0.63,0.88}{##1}}}
\@namedef{PY@tok@cpf}{\def\PY@tc##1{\textcolor[rgb]{0.63,0.63,0.88}{##1}}}
\@namedef{PY@tok@c1}{\def\PY@tc##1{\textcolor[rgb]{0.63,0.63,0.88}{##1}}}
\@namedef{PY@tok@cs}{\def\PY@tc##1{\textcolor[rgb]{0.63,0.63,0.88}{##1}}}

\def\PYZbs{\char`\\}
\def\PYZus{\char`\_}
\def\PYZob{\char`\{}
\def\PYZcb{\char`\}}
\def\PYZca{\char`\^}
\def\PYZam{\char`\&}
\def\PYZlt{\char`\<}
\def\PYZgt{\char`\>}
\def\PYZsh{\char`\#}
\def\PYZpc{\char`\%}
\def\PYZdl{\char`\$}
\def\PYZhy{\char`\-}
\def\PYZsq{\char`\'}
\def\PYZdq{\char`\"}
\def\PYZti{\char`\~}
% for compatibility with earlier versions
\def\PYZat{@}
\def\PYZlb{[}
\def\PYZrb{]}
\makeatother

\definecolor{myblue}{RGB}{0,163,243}
\definecolor{myred}{RGB}{243, 10, 25}
\definecolor{mygreen}{RGB}{50, 205, 50}

\newcommand{\fon}[1]{\fontfamily{#1}\selectfont}


\tcbset{examplestyle/.style={
  enhanced,
  outer arc=4pt,
  arc=4pt,
  colframe=myblue,
  colback=myblue!20,
  attach boxed title to top left,
  boxed title style={
    colback=myblue,
    outer arc=4pt,
    arc=4pt,
    top=3pt,
    bottom=3pt,
    },
  fonttitle=\sffamily
  }
}

\tcbset{inoutstyle/.style={
  enhanced,
  outer arc=4pt,
  arc=4pt,
  colframe=mygreen,
  colback=mygreen!20,
  attach boxed title to top left,
  boxed title style={
    colback=mygreen,
    outer arc=4pt,
    arc=4pt,
    top=3pt,
    bottom=3pt,
    },
  fonttitle=\sffamily,
  fontupper=\ttfamily,
  fontlower=\ttfamily,
  }
}

\tcbset{errorstyle/.style={
  enhanced,
  outer arc=4pt,
  arc=4pt,
  colframe=myred,
  colback=myred!20,
  attach boxed title to top left,
  boxed title style={
    colback=myred,
    outer arc=4pt,
    arc=4pt,
    top=3pt,
    bottom=3pt,
    },
  fonttitle=\sffamily
  }
}

\newtcolorbox[auto counter,number within=section]{examples}[1][]{
  examplestyle,
  colback=white,
  title=Primer,
  overlay unbroken and first={
      \path
        let
        \p1=(title.north east),
        \p2=(frame.north east)
        in
        node[anchor=west,font=\sffamily,color=myblue,text width=\x2-\x1]
        at (title.east) {#1};
  }
}
\newtcolorbox[auto counter]{errors}[1][]{
  errorstyle,
  colback=white,
  title=Pogoste napake,
  overlay unbroken and first={
      \path
        let
        \p1=(title.north east),
        \p2=(frame.north east)
        in
        node[anchor=west,font=\sffamily,color=myblue,text width=\x2-\x1]
        at (title.east) {#1};
  }
}

\newtcolorbox[auto counter]{inout}[1][]{
  inoutstyle,
  colback=white,
  title=Primer vhoda in izhoda,
  overlay unbroken and first={
      \path
        let
        \p1=(title.north east),
        \p2=(frame.north east)
        in
        node[anchor=west,font=\sffamily,color=myblue,text width=\x2-\x1]
        at (title.east) {#1};
  }
}
\title{Efektivnost programov in asimptotična notacija}
\date{}

\begin{document}
\maketitle

\section{Merjenje efektivnosti programa}

Pogosto obstaja več možnosti, kako se lahko lotimo reševanja danega problema.
Če želimo najti najmanjši element v seznamu, lahko pregledamo celoten seznam in
si beležimo najmanjšega, ki smo ga našli do sedaj, lahko pa celoten seznam
uredimo po vrsti in nato izberemo prvi element, na primer.
Pričakujemo lahko, da se bodo različni algoritmi za reševanje istega problema
razlikovali tudi po tem, kako hitro problem rešijo.
Kako pa v računalništvu izmerimo hitrost? Če delamo samo na enem računalniku,
lahko izmerimo, konkretno koliko časa je program potreboval, da je zaključil
z delovanjem. Na ta način lahko na primerih demonstriramo, da je nek algoritem
boljši od drugega; ko pa želimo naše rezultate deliti in primerjati z drugimi,
pa se ne moramo zanašati, da bodo imeli enako močen računalnik kot mi, in da
bodo njihovi testni primeri primerljivo zahtevni z našimi. Dejansko so težave
pri tem še hujše; na hitrost delovanja našega programa ne vpliva samo strojna
oprema računalnika (torej, kakšen procesor ima, koliko ima spomina itd.), temveč
tudi ostali programi, ki jih imamo hkrati odprte.
Če se želimo pogovarjati o hitrosti algoritmov, potrebujemo bolj abstraktno
orodje. Na pomoč pride asimptotična zahtevnost.

Da določimo hitrost našega programa, moramo prvo določiti, katere spremenljivke
vplivajo na čas delovanja, ter kako je čas od njih odvisen.
Rezultat take analize zapišemo kot izraz v oklepaje, pred katere zapišemo
veliko črko O: \(O(\ldots)\)
Poglejmo si primer.

\begin{Verbatim}[commandchars=\\\{\}]
\PY{k+kt}{int} \PY{n}{arr}\PY{p}{[}\PY{l+m+mi}{100002}\PY{p}{]}\PY{p}{;}

\PY{k+kt}{int} \PY{n+nf}{poisci\PYZus{}najmanjsega}\PY{p}{(}\PY{k+kt}{int} \PY{n}{n}\PY{p}{)} \PY{p}{\PYZob{}}
    \PY{c+c1}{// Poišči najmanjše število v seznamu, dolgemu n}
    \PY{k+kt}{int} \PY{n}{min\PYZus{}idx} \PY{o}{=} \PY{l+m+mi}{0}\PY{p}{;}
    \PY{k}{for} \PY{p}{(}\PY{k+kt}{int} \PY{n}{i} \PY{o}{=} \PY{l+m+mi}{0}\PY{p}{;} \PY{n}{i} \PY{o}{\PYZlt{}} \PY{n}{n}\PY{p}{;} \PY{n}{i}\PY{o}{+}\PY{o}{+}\PY{p}{)} \PY{p}{\PYZob{}}
        \PY{k}{if} \PY{p}{(}\PY{n}{arr}\PY{p}{[}\PY{n}{min\PYZus{}idx}\PY{p}{]} \PY{o}{\PYZgt{}} \PY{n}{arr}\PY{p}{[}\PY{n}{i}\PY{p}{]}\PY{p}{)} \PY{p}{\PYZob{}}
            \PY{n}{min\PYZus{}idx} \PY{o}{=} \PY{n}{i}\PY{p}{;}
        \PY{p}{\PYZcb{}}
    \PY{p}{\PYZcb{}}
    \PY{k}{return} \PY{n}{arr}\PY{p}{[}\PY{n}{min\PYZus{}idx}\PY{p}{]}\PY{p}{;}
\PY{p}{\PYZcb{}}
\end{Verbatim}

Funkcija \verb+poisci_najmanjsega+ sprejme število \(n\), ki pove dolžino seznama
\verb+arr+. Po seznamu se nato enkrat sprehodi, in si ob tem beleži indeks
najmanjšega elementa, ki ga je do sedaj našla.

Razmislimo, katere vse različne operacije program opravi.
\begin{itemize}
  \item 
	Večkrat med programom nastavimo neki spremenljivki novo vrednost.
  \item
	V vsaki iteraciji zanke prištejemo 1 spremenljivki \verb+i+.
  \item
	Poleg tega v vsaki iteraciji zanke tudi primerjamo \verb+i+ z \verb+n+,
  \item
	dvakrat dostopamo do nekega elementa v seznamu,
  \item
	ter ju primerjamo.
  \item
	Na koncu še enkrat dostopamo do elementa v seznamu, ter ga vrnemo.
\end{itemize}

Vse naštete operacije same po sebi \emph{trajajo} \(O(1)\) časa. To pomeni, da
se vedno izvajajo enako hitro, neodvisno od parametrov, ki jim podamo. Rečemo
tudi, da porabijo \emph{konstantno mnogo} časa.

Kolikokrat pa izvedemo te operacije? Analizirajmo najslabši primer za naš
program; če je seznam \verb+arr+ padajoče urejen. Tedaj bomo v vsaki iteraciji
zanke enkrat primerjali \verb+i < n+, dvakrat dostopali do elementov seznama,
enkrat primerjali \verb+arr[min_idx] > arr[i]+, enkrat nastavili \verb+min_idx+,
ter enkrat povečali \verb+i+. Zunaj zanke bomo nastavili \verb+min_idx+ ter
\verb+i+ na začetni vrednosti, ter še enkrat dostopali do elementa v seznamu.
Zanka se vedno izvaja za natanko \verb+n+ iteracij; vedno vsak element pregledamo
enkrat. Torej je celotna časovna zahtevnost našega programa \(O(3 + 6n)\).
Ker pa za velike \(n\) del zunaj zanke hitro postane nepomemben, ga ignoriramo.
Poleg tega ignoriramo tudi faktor pred členom \(n\) -- ker tako in tako ne moramo
vedeti, kako hitre so operacije v zanki v primerjavi druga z drugo, te konstante
ne moramo natančno določiti. Končna časovna zahtevnost našega programa je torej
\(O(n)\).

To je tudi najboljši možni algoritem za iskanje najmanjšega elementa v seznamu.
Če bi nek algoritem namreč deloval v hitrejšem času kot \(O(n)\), bi moral
nekatera mesta v seznamu izpustiti; če tedaj algoritmu podamo seznam, ki ima
najmanjši element ravno na takem mestu, ga algoritem ne bo našel, in bo podal
napačen odgovor.

Pomembna opazka je, da hitrost našega algoritma ni odvisna od velikosti števil
v seznamu, temveč le od velikosti seznama. Naslednji program prav tako poišče
najmanjše število v seznamu, vendar je konkretno počasnejši od zgornjega:

\begin{Verbatim}[commandchars=\\\{\}]
\PY{k+kt}{int} \PY{n}{arr}\PY{p}{[}\PY{l+m+mi}{100002}\PY{p}{]}\PY{p}{;}

\PY{k+kt}{int} \PY{n+nf}{poisci\PYZus{}najmanjsega\PYZus{}slabsi}\PY{p}{(}\PY{k+kt}{int} \PY{n}{n}\PY{p}{,} \PY{k+kt}{int} \PY{n}{m}\PY{p}{)} \PY{p}{\PYZob{}}
    \PY{c+c1}{// Poišči najmanjše število v seznamu, dolgemu n,}
    \PY{c+c1}{// kjer je največje število veliko največ m}

    \PY{k}{for} \PY{p}{(}\PY{k+kt}{int} \PY{n}{zelja} \PY{o}{=} \PY{l+m+mi}{0}\PY{p}{;} \PY{n}{zelja} \PY{o}{\PYZlt{}}\PY{o}{=} \PY{n}{m}\PY{p}{;} \PY{n}{zelja}\PY{o}{+}\PY{o}{+}\PY{p}{)} \PY{p}{\PYZob{}}
        \PY{c+c1}{// zelja nam pove, kateri element si v tej iteraciji želimo}
        \PY{c+c1}{// najti. Pogledati moramo še, da ta element dejansko je v seznamu;}
        \PY{c+c1}{// ko pa najdemo enega, bo to najmanjši (ker zelja v vsaki iteraciji}
        \PY{c+c1}{// narasca)}

        \PY{k+kt}{bool} \PY{n}{je\PYZus{}v\PYZus{}seznamu} \PY{o}{=} \PY{n+nb}{false}\PY{p}{;}
        \PY{k}{for} \PY{p}{(}\PY{k+kt}{int} \PY{n}{i} \PY{o}{=} \PY{l+m+mi}{0}\PY{p}{;} \PY{n}{i} \PY{o}{\PYZlt{}} \PY{n}{n}\PY{p}{;} \PY{n}{i}\PY{o}{+}\PY{o}{+}\PY{p}{)} \PY{p}{\PYZob{}}
            \PY{k}{if} \PY{p}{(}\PY{n}{arr}\PY{p}{[}\PY{n}{i}\PY{p}{]} \PY{o}{=}\PY{o}{=} \PY{n}{zelja}\PY{p}{)}
                \PY{n}{je\PYZus{}v\PYZus{}seznamu} \PY{o}{=} \PY{n+nb}{true}\PY{p}{;}
        \PY{p}{\PYZcb{}}

        \PY{k}{if} \PY{p}{(}\PY{n}{je\PYZus{}v\PYZus{}seznamu}\PY{p}{)}
            \PY{k}{return} \PY{n}{zelja}\PY{p}{;}
    \PY{p}{\PYZcb{}}
\PY{p}{\PYZcb{}}
\end{Verbatim}

V tem programu imamo dve zanki; ena se sprehaja po vseh možnih vrednosti števil
v seznamu, druga pa preverja, če je ta element dejansko v seznamu. Vsakič, ko se
zunanja zanka izvede enkrat, se notranja izvede \verb+n+-krat (v najslabšem
primeru), zunanja zanka pa se izvede \verb|m+1|-krat. Torej je zahtevnost
\(O((m+1) \cdot n)\), oziroma \(O(mn + n)\). Člen \(n\) v vsoti pa je v vseh
primerih manjši od člena \(mn\) ali njemu enako velik, zato ga izpustimo.
Končna časovna zahtevnost drugega algoritma je torej \(O(mn)\).

\verb+int+ lahko hrani števila, velika do približno dve milijardi -- najslabšem
primeru je \(m\) torej približno \(2 \cdot 10^9\). Če prvi algoritem na nekem
računalniku potrebuje eno sekundo, da se konča, bi drugi algoritem v najslabšem
primeru na istem računalniku potreboval več kot šestdeset let.

\begin{examples}

Naslednji program za vsako število v seznamu \verb+arr+ poišče število števil desno od njega, ki so večja.

\begin{Verbatim}[commandchars=\\\{\}]
\PY{k}{for} \PY{p}{(}\PY{k+kt}{int} \PY{n}{i} \PY{o}{=} \PY{l+m+mi}{0}\PY{p}{;} \PY{n}{i} \PY{o}{\PYZlt{}} \PY{n}{n}\PY{p}{;} \PY{n}{i}\PY{o}{+}\PY{o}{+}\PY{p}{)} \PY{p}{\PYZob{}}
    \PY{k+kt}{int} \PY{n}{stevilo} \PY{o}{=} \PY{l+m+mi}{0}\PY{p}{;}
    \PY{k}{for} \PY{p}{(}\PY{k+kt}{int} \PY{n}{j} \PY{o}{=} \PY{n}{i}\PY{o}{+}\PY{l+m+mi}{1}\PY{p}{;} \PY{n}{j} \PY{o}{\PYZlt{}} \PY{n}{n}\PY{p}{;} \PY{n}{j}\PY{o}{+}\PY{o}{+}\PY{p}{)} \PY{p}{\PYZob{}}
        \PY{k}{if} \PY{p}{(}\PY{n}{arr}\PY{p}{[}\PY{n}{j}\PY{p}{]} \PY{o}{\PYZgt{}} \PY{n}{arr}\PY{p}{[}\PY{n}{i}\PY{p}{]}\PY{p}{)} \PY{p}{\PYZob{}}
            \PY{n}{stevilo}\PY{o}{+}\PY{o}{+}\PY{p}{;}
        \PY{p}{\PYZcb{}}
    \PY{p}{\PYZcb{}}
    \PY{n}{printf}\PY{p}{(}\PY{l+s}{\PYZdq{}}\PY{l+s}{\PYZpc{}d}\PY{l+s+se}{\PYZbs{}n}\PY{l+s}{\PYZdq{}}\PY{p}{,} \PY{n}{stevilo}\PY{p}{)}\PY{p}{;}
\PY{p}{\PYZcb{}}
\end{Verbatim}


Notranja zanka se v prvi iteraciji izvede \((n-1)\)-krat, v drugi iteraciji
\((n-2)\)-krat, v tretji \((n-3)\)-krat, itd. V zadnji iteraciji se sploh ne
izvede. Skupaj se koda znotraj druge zanke torej izvede
\((n-1) + (n-2) + \ldots + 1 + 0 = \frac{n(n-1)}{2}\)-krat. Spet ignoriramo
konstanto \(\frac{1}{2}\) ter člen samo z \(n\), in pridemo do zahtevnosti
\(O(n^2)\).

\end{examples}

\begin{examples}

Naslednji program preveri, če je število \(n\) praštevilo.

\begin{Verbatim}[commandchars=\\\{\}]
\PY{k+kt}{bool} \PY{n}{preklicano} \PY{o}{=} \PY{n+nb}{false}\PY{p}{;}
\PY{k}{for} \PY{p}{(}\PY{k+kt}{int} \PY{n}{i} \PY{o}{=} \PY{l+m+mi}{2}\PY{p}{;} \PY{n}{i} \PY{o}{*} \PY{n}{i} \PY{o}{\PYZlt{}} \PY{n}{n}\PY{p}{;} \PY{n}{i}\PY{o}{+}\PY{o}{+}\PY{p}{)} \PY{p}{\PYZob{}}
    \PY{k}{if} \PY{p}{(}\PY{n}{n} \PY{o}{\PYZpc{}} \PY{n}{i} \PY{o}{=}\PY{o}{=} \PY{l+m+mi}{0}\PY{p}{)} \PY{p}{\PYZob{}}
        \PY{n}{printf}\PY{p}{(}\PY{l+s}{\PYZdq{}}\PY{l+s}{\PYZpc{}d ni prastevilo}\PY{l+s+se}{\PYZbs{}n}\PY{l+s}{\PYZdq{}}\PY{p}{,} \PY{n}{n}\PY{p}{)}\PY{p}{;}
        \PY{n}{preklicano} \PY{o}{=} \PY{n+nb}{true}\PY{p}{;}
        \PY{k}{break}\PY{p}{;}
    \PY{p}{\PYZcb{}}
\PY{p}{\PYZcb{}}
\PY{k}{if} \PY{p}{(}\PY{o}{!}\PY{n}{preklicano}\PY{p}{)}
    \PY{n}{printf}\PY{p}{(}\PY{l+s}{\PYZdq{}}\PY{l+s}{\PYZpc{}d je prastevilo}\PY{l+s+se}{\PYZbs{}n}\PY{l+s}{\PYZdq{}}\PY{p}{,} \PY{n}{n}\PY{p}{)}\PY{p}{;}
\end{Verbatim}

Program ima eno zanko, ki se sprehaja toliko časa, da kvadrat spremenljivke
\(i\) postane večji kot \(n\), oz.~dokler je \(i < \sqrt{n}\). Zanka se torej
izvede v \(O(\sqrt{n})\).

\end{examples}

\pagebreak
\section{Klasifikacija}

Računalnik lahko v eni sekundi opravi približno \(10^7\) operacij. Da določimo,
kako dober algoritem potrebujemo za rešitev neke naloge, lahko preverimo
omejitve vhodnih podatkov. Spodnja tabela prikazuje nekaj pogostih časovnih
zahtevnosti, ter pripadajoče največje omejitve. Z uporabo te tabele lahko
vnaprej določimo, kakšno največjo časovno zahtevnost mora imeti naš program,
da reši določeno nalogo.

\begin{table}[h!]
  \centering
  \begin{tabular}{|c|c|c|}
	\hline
	Zahtevnost & Omejitev za \(n\) & Ime zahtevnosti \\
	\hline
	\(O(1)\) & brez & \emph{konstantna} \\
	\(O(\log n)\) & zelo visoka & \emph{logaritemska} \\
	\(O(\sqrt{n})\) & \(10^{14}\) & \emph{korenska} \\
	\(O(n)\) & \(10^7\) & \emph{linearna} \\
	\(O(n \log n)\) & \(10^6\) & \\
	\(O(n^2)\) & \(10^4\) & \emph{kvadratna} \\
	\(O(n^3)\) & \(300\) & \emph{kubična} \\
	\(O(2^n)\) & \(20\) & \emph{eksponentna} \\
	\hline
  \end{tabular}
\end{table}

\end{document}
