\documentclass{article}
\usepackage{fancyvrb}
\usepackage{color}
\usepackage[utf8]{inputenc}
\usepackage[a4paper, total={6.2in, 9in}]{geometry}
\usepackage[many]{tcolorbox}
\usepackage[T1]{fontenc}
\usepackage[slovene]{babel}

\usetikzlibrary{calc}

\setlength\parindent{0pt}

\makeatletter
\def\PY@reset{\let\PY@it=\relax \let\PY@bf=\relax%
    \let\PY@ul=\relax \let\PY@tc=\relax%
    \let\PY@bc=\relax \let\PY@ff=\relax}
\def\PY@tok#1{\csname PY@tok@#1\endcsname}
\def\PY@toks#1+{\ifx\relax#1\empty\else%
    \PY@tok{#1}\expandafter\PY@toks\fi}
\def\PY@do#1{\PY@bc{\PY@tc{\PY@ul{%
    \PY@it{\PY@bf{\PY@ff{#1}}}}}}}
\def\PY#1#2{\PY@reset\PY@toks#1+\relax+\PY@do{#2}}

\@namedef{PY@tok@c}{\def\PY@tc##1{\textcolor[rgb]{0.63,0.63,0.88}{##1}}}
\@namedef{PY@tok@cp}{\def\PY@tc##1{\textcolor[rgb]{0.19,0.51,0.25}{##1}}}
\@namedef{PY@tok@k}{\let\PY@bf=\textbf\def\PY@tc##1{\textcolor[rgb]{0.00,0.00,0.67}{##1}}}
\@namedef{PY@tok@o}{\def\PY@tc##1{\textcolor[rgb]{1.00,0.02,0.02}{##1}}}
\@namedef{PY@tok@p}{\def\PY@tc##1{\textcolor[rgb]{1.00,0.02,0.02}{##1}}}
\@namedef{PY@tok@n}{\def\PY@tc##1{\textcolor[rgb]{0.05,0.05,0.05}{##1}}}
\@namedef{PY@tok@s}{\def\PY@tc##1{\textcolor[rgb]{0.16,0.16,1.00}{##1}}}
\@namedef{PY@tok@m}{\def\PY@tc##1{\textcolor[rgb]{0.94,0.03,0.94}{##1}}}
\@namedef{PY@tok@kc}{\let\PY@bf=\textbf\def\PY@tc##1{\textcolor[rgb]{0.00,0.00,0.67}{##1}}}
\@namedef{PY@tok@kd}{\let\PY@bf=\textbf\def\PY@tc##1{\textcolor[rgb]{0.00,0.00,0.67}{##1}}}
\@namedef{PY@tok@kn}{\let\PY@bf=\textbf\def\PY@tc##1{\textcolor[rgb]{0.00,0.00,0.67}{##1}}}
\@namedef{PY@tok@kp}{\let\PY@bf=\textbf\def\PY@tc##1{\textcolor[rgb]{0.00,0.00,0.67}{##1}}}
\@namedef{PY@tok@kr}{\let\PY@bf=\textbf\def\PY@tc##1{\textcolor[rgb]{0.00,0.00,0.67}{##1}}}
\@namedef{PY@tok@kt}{\let\PY@bf=\textbf\def\PY@tc##1{\textcolor[rgb]{0.00,0.00,0.67}{##1}}}
\@namedef{PY@tok@na}{\def\PY@tc##1{\textcolor[rgb]{0.05,0.05,0.05}{##1}}}
\@namedef{PY@tok@nb}{\def\PY@tc##1{\textcolor[rgb]{0.05,0.05,0.05}{##1}}}
\@namedef{PY@tok@bp}{\def\PY@tc##1{\textcolor[rgb]{0.05,0.05,0.05}{##1}}}
\@namedef{PY@tok@nc}{\def\PY@tc##1{\textcolor[rgb]{0.05,0.05,0.05}{##1}}}
\@namedef{PY@tok@no}{\def\PY@tc##1{\textcolor[rgb]{0.05,0.05,0.05}{##1}}}
\@namedef{PY@tok@nd}{\def\PY@tc##1{\textcolor[rgb]{0.05,0.05,0.05}{##1}}}
\@namedef{PY@tok@ni}{\def\PY@tc##1{\textcolor[rgb]{0.05,0.05,0.05}{##1}}}
\@namedef{PY@tok@ne}{\def\PY@tc##1{\textcolor[rgb]{0.05,0.05,0.05}{##1}}}
\@namedef{PY@tok@nf}{\def\PY@tc##1{\textcolor[rgb]{0.05,0.05,0.05}{##1}}}
\@namedef{PY@tok@fm}{\def\PY@tc##1{\textcolor[rgb]{0.05,0.05,0.05}{##1}}}
\@namedef{PY@tok@py}{\def\PY@tc##1{\textcolor[rgb]{0.05,0.05,0.05}{##1}}}
\@namedef{PY@tok@nl}{\def\PY@tc##1{\textcolor[rgb]{0.05,0.05,0.05}{##1}}}
\@namedef{PY@tok@nn}{\def\PY@tc##1{\textcolor[rgb]{0.05,0.05,0.05}{##1}}}
\@namedef{PY@tok@nx}{\def\PY@tc##1{\textcolor[rgb]{0.05,0.05,0.05}{##1}}}
\@namedef{PY@tok@nt}{\def\PY@tc##1{\textcolor[rgb]{0.05,0.05,0.05}{##1}}}
\@namedef{PY@tok@nv}{\def\PY@tc##1{\textcolor[rgb]{0.05,0.05,0.05}{##1}}}
\@namedef{PY@tok@vc}{\def\PY@tc##1{\textcolor[rgb]{0.05,0.05,0.05}{##1}}}
\@namedef{PY@tok@vg}{\def\PY@tc##1{\textcolor[rgb]{0.05,0.05,0.05}{##1}}}
\@namedef{PY@tok@vi}{\def\PY@tc##1{\textcolor[rgb]{0.05,0.05,0.05}{##1}}}
\@namedef{PY@tok@vm}{\def\PY@tc##1{\textcolor[rgb]{0.05,0.05,0.05}{##1}}}
\@namedef{PY@tok@sa}{\def\PY@tc##1{\textcolor[rgb]{0.16,0.16,1.00}{##1}}}
\@namedef{PY@tok@sb}{\def\PY@tc##1{\textcolor[rgb]{0.16,0.16,1.00}{##1}}}
\@namedef{PY@tok@sc}{\def\PY@tc##1{\textcolor[rgb]{0.16,0.16,1.00}{##1}}}
\@namedef{PY@tok@dl}{\def\PY@tc##1{\textcolor[rgb]{0.16,0.16,1.00}{##1}}}
\@namedef{PY@tok@sd}{\def\PY@tc##1{\textcolor[rgb]{0.16,0.16,1.00}{##1}}}
\@namedef{PY@tok@s2}{\def\PY@tc##1{\textcolor[rgb]{0.16,0.16,1.00}{##1}}}
\@namedef{PY@tok@se}{\def\PY@tc##1{\textcolor[rgb]{0.16,0.16,1.00}{##1}}}
\@namedef{PY@tok@sh}{\def\PY@tc##1{\textcolor[rgb]{0.16,0.16,1.00}{##1}}}
\@namedef{PY@tok@si}{\def\PY@tc##1{\textcolor[rgb]{0.16,0.16,1.00}{##1}}}
\@namedef{PY@tok@sx}{\def\PY@tc##1{\textcolor[rgb]{0.16,0.16,1.00}{##1}}}
\@namedef{PY@tok@sr}{\def\PY@tc##1{\textcolor[rgb]{0.16,0.16,1.00}{##1}}}
\@namedef{PY@tok@s1}{\def\PY@tc##1{\textcolor[rgb]{0.16,0.16,1.00}{##1}}}
\@namedef{PY@tok@ss}{\def\PY@tc##1{\textcolor[rgb]{0.16,0.16,1.00}{##1}}}
\@namedef{PY@tok@mb}{\def\PY@tc##1{\textcolor[rgb]{0.94,0.03,0.94}{##1}}}
\@namedef{PY@tok@mf}{\def\PY@tc##1{\textcolor[rgb]{0.94,0.03,0.94}{##1}}}
\@namedef{PY@tok@mh}{\def\PY@tc##1{\textcolor[rgb]{0.94,0.03,0.94}{##1}}}
\@namedef{PY@tok@mi}{\def\PY@tc##1{\textcolor[rgb]{0.94,0.03,0.94}{##1}}}
\@namedef{PY@tok@il}{\def\PY@tc##1{\textcolor[rgb]{0.94,0.03,0.94}{##1}}}
\@namedef{PY@tok@mo}{\def\PY@tc##1{\textcolor[rgb]{0.94,0.03,0.94}{##1}}}
\@namedef{PY@tok@ow}{\def\PY@tc##1{\textcolor[rgb]{1.00,0.02,0.02}{##1}}}
\@namedef{PY@tok@ch}{\def\PY@tc##1{\textcolor[rgb]{0.63,0.63,0.88}{##1}}}
\@namedef{PY@tok@cm}{\def\PY@tc##1{\textcolor[rgb]{0.63,0.63,0.88}{##1}}}
\@namedef{PY@tok@cpf}{\def\PY@tc##1{\textcolor[rgb]{0.63,0.63,0.88}{##1}}}
\@namedef{PY@tok@c1}{\def\PY@tc##1{\textcolor[rgb]{0.63,0.63,0.88}{##1}}}
\@namedef{PY@tok@cs}{\def\PY@tc##1{\textcolor[rgb]{0.63,0.63,0.88}{##1}}}

\def\PYZbs{\char`\\}
\def\PYZus{\char`\_}
\def\PYZob{\char`\{}
\def\PYZcb{\char`\}}
\def\PYZca{\char`\^}
\def\PYZam{\char`\&}
\def\PYZlt{\char`\<}
\def\PYZgt{\char`\>}
\def\PYZsh{\char`\#}
\def\PYZpc{\char`\%}
\def\PYZdl{\char`\$}
\def\PYZhy{\char`\-}
\def\PYZsq{\char`\'}
\def\PYZdq{\char`\"}
\def\PYZti{\char`\~}
% for compatibility with earlier versions
\def\PYZat{@}
\def\PYZlb{[}
\def\PYZrb{]}
\makeatother

\definecolor{myblue}{RGB}{0,163,243}
\definecolor{myred}{RGB}{243, 10, 25}
\definecolor{mygreen}{RGB}{50, 205, 50}

\newcommand{\fon}[1]{\fontfamily{#1}\selectfont}


\tcbset{examplestyle/.style={
  enhanced,
  outer arc=4pt,
  arc=4pt,
  colframe=myblue,
  colback=myblue!20,
  attach boxed title to top left,
  boxed title style={
    colback=myblue,
    outer arc=4pt,
    arc=4pt,
    top=3pt,
    bottom=3pt,
    },
  fonttitle=\sffamily
  }
}

\tcbset{inoutstyle/.style={
  enhanced,
  outer arc=4pt,
  arc=4pt,
  colframe=mygreen,
  colback=mygreen!20,
  attach boxed title to top left,
  boxed title style={
    colback=mygreen,
    outer arc=4pt,
    arc=4pt,
    top=3pt,
    bottom=3pt,
    },
  fonttitle=\sffamily,
  fontupper=\ttfamily,
  fontlower=\ttfamily,
  }
}

\tcbset{errorstyle/.style={
  enhanced,
  outer arc=4pt,
  arc=4pt,
  colframe=myred,
  colback=myred!20,
  attach boxed title to top left,
  boxed title style={
    colback=myred,
    outer arc=4pt,
    arc=4pt,
    top=3pt,
    bottom=3pt,
    },
  fonttitle=\sffamily
  }
}

\newtcolorbox[auto counter,number within=section]{examples}[1][]{
  examplestyle,
  colback=white,
  title=Primer,
  overlay unbroken and first={
      \path
        let
        \p1=(title.north east),
        \p2=(frame.north east)
        in
        node[anchor=west,font=\sffamily,color=myblue,text width=\x2-\x1]
        at (title.east) {#1};
  }
}
\newtcolorbox[auto counter]{errors}[1][]{
  errorstyle,
  colback=white,
  title=Pogoste napake,
  overlay unbroken and first={
      \path
        let
        \p1=(title.north east),
        \p2=(frame.north east)
        in
        node[anchor=west,font=\sffamily,color=myblue,text width=\x2-\x1]
        at (title.east) {#1};
  }
}

\newtcolorbox[auto counter]{inout}[1][]{
  inoutstyle,
  colback=white,
  title=Primer vhoda in izhoda,
  overlay unbroken and first={
      \path
        let
        \p1=(title.north east),
        \p2=(frame.north east)
        in
        node[anchor=west,font=\sffamily,color=myblue,text width=\x2-\x1]
        at (title.east) {#1};
  }
}

\title{Računske operacije}
\date{}

\begin{document}
\maketitle

\section{Seštevanje, odštevanje, množenje}
Računalniki lahko s spremenljivkami počnejo veliko stvari. Najpreprostejše so  operacije na številih, kot so seštevanje, odštevanje in množenje. Račune zapisujemo tako kot v šoli, z \emph{operatorji}.
\begin{itemize}
	\item \verb-+- za seštevanje
	\item \verb+-+ za odštevanje
	\item \verb+*+ za množenje
\end{itemize}

Rezultat lahko izračunamo kar znotraj funkcije \verb+printf+:


\begin{examples}
\begin{Verbatim}[commandchars=\\\{\}]
\PY{c+cp}{\PYZsh{}}\PY{c+cp}{include}\PY{c+cpf}{\PYZlt{}stdio.h\PYZgt{}}

\PY{k+kt}{int}\PY{+w}{ }\PY{n+nf}{main}\PY{p}{(}\PY{p}{)}\PY{p}{\PYZob{}}
\PY{+w}{	}\PY{k+kt}{int}\PY{+w}{ }\PY{n}{a}\PY{+w}{ }\PY{o}{=}\PY{+w}{ }\PY{l+m+mi}{5}\PY{p}{,}\PY{+w}{ }\PY{n}{b}\PY{+w}{ }\PY{o}{=}\PY{+w}{ }\PY{l+m+mi}{7}\PY{p}{;}
\PY{+w}{	}\PY{n}{printf}\PY{p}{(}\PY{l+s}{\PYZdq{}}\PY{l+s}{\PYZpc{}d}\PY{l+s+se}{\PYZbs{}n}\PY{l+s}{\PYZdq{}}\PY{p}{,}\PY{+w}{ }\PY{n}{a}\PY{o}{+}\PY{n}{b}\PY{p}{)}\PY{p}{;}
\PY{+w}{	}\PY{k}{return}\PY{+w}{ }\PY{l+m+mi}{0}\PY{p}{;}
\PY{p}{\PYZcb{}}
\end{Verbatim}


\begin{inout}

\tcblower
12
\end{inout}


\end{examples}

\pagebreak
Rezultat lahko tudi shranimo v novo spremenljivko:
\begin{examples}
\begin{Verbatim}[commandchars=\\\{\}]
\PY{c+cp}{\PYZsh{}}\PY{c+cp}{include}\PY{c+cpf}{\PYZlt{}stdio.h\PYZgt{}}

\PY{k+kt}{int}\PY{+w}{ }\PY{n+nf}{main}\PY{p}{(}\PY{p}{)}\PY{p}{\PYZob{}}
\PY{+w}{	}\PY{k+kt}{int}\PY{+w}{ }\PY{n}{a}\PY{p}{,}\PY{+w}{ }\PY{n}{b}\PY{p}{,}\PY{+w}{ }\PY{n}{vsota}\PY{p}{,}\PY{+w}{ }\PY{n}{razlika}\PY{p}{,}\PY{+w}{ }\PY{n}{produkt}\PY{p}{;}
\PY{+w}{	}\PY{n}{scanf}\PY{p}{(}\PY{l+s}{\PYZdq{}}\PY{l+s}{\PYZpc{}d\PYZpc{}d}\PY{l+s}{\PYZdq{}}\PY{p}{,}\PY{+w}{ }\PY{o}{\PYZam{}}\PY{n}{a}\PY{p}{,}\PY{+w}{ }\PY{o}{\PYZam{}}\PY{n}{b}\PY{p}{)}\PY{p}{;}
\PY{+w}{	}\PY{n}{vsota}\PY{+w}{ }\PY{o}{=}\PY{+w}{ }\PY{n}{a}\PY{o}{+}\PY{n}{b}\PY{p}{;}
\PY{+w}{	}\PY{n}{razlika}\PY{+w}{ }\PY{o}{=}\PY{+w}{ }\PY{n}{a}\PY{o}{\PYZhy{}}\PY{n}{b}\PY{p}{;}
\PY{+w}{	}\PY{n}{produkt}\PY{+w}{ }\PY{o}{=}\PY{+w}{ }\PY{n}{a}\PY{o}{*}\PY{n}{b}\PY{p}{;}
\PY{+w}{	}\PY{n}{printf}\PY{p}{(}\PY{l+s}{\PYZdq{}}\PY{l+s}{\PYZpc{}d}\PY{l+s+se}{\PYZbs{}n}\PY{l+s}{\PYZpc{}d}\PY{l+s+se}{\PYZbs{}n}\PY{l+s}{\PYZpc{}d}\PY{l+s+se}{\PYZbs{}n}\PY{l+s}{\PYZdq{}}\PY{p}{,}\PY{+w}{ }\PY{n}{vsota}\PY{p}{,}\PY{+w}{ }\PY{n}{razlika}\PY{p}{,}\PY{+w}{ }\PY{n}{produkt}\PY{p}{)}\PY{p}{;}
\PY{+w}{	}\PY{k}{return}\PY{+w}{ }\PY{l+m+mi}{0}\PY{p}{;}
\PY{p}{\PYZcb{}}
\end{Verbatim}

\begin{inout}
3 7
\tcblower
12\\
-4\\
21
\end{inout}

\end{examples}

\begin{errors}
Spremenljivko \emph{inicializiramo} tako, da notri nekaj shranimo, bodisi kot \verb+a=5+, bodisi s tem, da vanjo nekaj napišemo s funkcijo \verb+scanf+.
Če je ne inicializiramo, pozneje pa jo poskusimo uporabiti za izpisovanje, računanje ali kaj drugega, lahko dobimo zelo čudne rezultate, naš program se lahko celo sesuje.
\end{errors}

%CHECK
V zgornjem programu tudi vidimo, da lahko z enim klicem funkcije \verb+scanf+ preberemo več spremenljivk.

\subsection*{Negativna števila}
Na meteorološki postaji Kredarica so leta 2014 izmerili povprečno januarsko temperaturo približno -5 °C, povprečno avgustovsko pa približno 6 °C.
Med tema meritvama je 11 °C razlike. \\
Pozimi lahko izmerimo temperature manjše od 0. Takšnim številom, kot je -5, rečemo \emph{negativna števila}. Lahko jih uporabimo tudi drugje, ne samo pri merjenju temperature. \\
S pozitivnimi števili lahko štejemo od 0 do neskončno (1, 2, 3, ...), z negativnimi pa do - neskončno. (-1, -2, -3, ...). Tako kot pozitivna števila jih lahko seštevamo in odštevamo:

\begin{examples}
5 - 11 = -6 \\
-6 + 11 = 5 \\
5 -(-6) = 11 \\
-2 - 1 = -3 \\ 
-2 - (-1) = -1
-2 + (-1) = -3 \\\\
\emph{Pravila:}\\
-(-6) = +6\\
+(-1) = -1 \\
-(-(-(-1))) = -(-(+1)) = -(-1) = 1
\end{examples}

Lahko jih tudi množimo:

\begin{examples}
2 * (-5) = -10 \\
(-5) * (-5) = 25 \\

\emph{Pravila:}
Če množimo dve pozitivni števili, je produkt pozitiven. \\
Če množimo eno pozitivno in eno negativno število, je produkt negativen. \\
Če množimo dve negativni števili, je produkt spet negativen. \\
O negativnih številih lahko razmišljamo kot: -3 = (-1) * 3
\end{examples}

Računalniki z negativnimi števili računajo enako kot s pozitivnimi:

\begin{examples}
\begin{Verbatim}[commandchars=\\\{\}]
\PY{c+cp}{\PYZsh{}}\PY{c+cp}{include}\PY{c+cpf}{\PYZlt{}stdio.h\PYZgt{}}

\PY{k+kt}{int}\PY{+w}{ }\PY{n+nf}{main}\PY{p}{(}\PY{p}{)}\PY{p}{\PYZob{}}
\PY{+w}{	}\PY{k+kt}{int}\PY{+w}{ }\PY{n}{a}\PY{p}{,}\PY{+w}{ }\PY{n}{b}\PY{p}{;}
\PY{+w}{	}\PY{n}{scanf}\PY{p}{(}\PY{l+s}{\PYZdq{}}\PY{l+s}{\PYZpc{}d\PYZpc{}d}\PY{l+s}{\PYZdq{}}\PY{p}{,}\PY{+w}{ }\PY{o}{\PYZam{}}\PY{n}{a}\PY{p}{,}\PY{+w}{ }\PY{o}{\PYZam{}}\PY{n}{b}\PY{p}{)}\PY{p}{;}
\PY{+w}{	}\PY{n}{printf}\PY{p}{(}\PY{l+s}{\PYZdq{}}\PY{l+s}{Vsota: \PYZpc{}d}\PY{l+s+se}{\PYZbs{}n}\PY{l+s}{Razlika: \PYZpc{}d}\PY{l+s+se}{\PYZbs{}n}\PY{l+s}{Produkt: \PYZpc{}d}\PY{l+s+se}{\PYZbs{}n}\PY{l+s}{\PYZdq{}}\PY{p}{,}\PY{+w}{ }\PY{n}{a}\PY{+w}{ }\PY{o}{+}\PY{+w}{ }\PY{n}{b}\PY{p}{,}\PY{+w}{ }\PY{n}{a}\PY{+w}{ }\PY{o}{\PYZhy{}}\PY{+w}{ }\PY{n}{b}\PY{p}{,}\PY{+w}{ }\PY{n}{a}\PY{+w}{ }\PY{o}{*}\PY{+w}{ }\PY{n}{b}\PY{p}{)}\PY{p}{;}
\PY{+w}{	}\PY{k}{return}\PY{+w}{ }\PY{l+m+mi}{0}\PY{p}{;}
\PY{p}{\PYZcb{}}
\end{Verbatim}


\begin{inout}
3 7
\tcblower
Vsota: 10 \\
Razlika: -4 \\
Produkt: 21
\end{inout}

\end{examples}

\pagebreak
\section{Deljenje}
Števila lahko tudi delimo. Za deljenje uporabljamo znak \verb+/+.
Za razumevanje poglavja si bomo pomagali s formulo $a = k*b + o$, ki ponazarja deljenje z ostankom.

\begin{examples}
\begin{Verbatim}[commandchars=\\\{\}]
\PY{c+cp}{\PYZsh{}}\PY{c+cp}{include}\PY{c+cpf}{\PYZlt{}stdio.h\PYZgt{}}

\PY{k+kt}{int}\PY{+w}{ }\PY{n+nf}{main}\PY{p}{(}\PY{p}{)}\PY{p}{\PYZob{}}
\PY{+w}{	}\PY{k+kt}{int}\PY{+w}{ }\PY{n}{a}\PY{p}{,}\PY{+w}{ }\PY{n}{b}\PY{p}{;}
\PY{+w}{	}\PY{n}{scanf}\PY{p}{(}\PY{l+s}{\PYZdq{}}\PY{l+s}{\PYZpc{}d\PYZpc{}d}\PY{l+s}{\PYZdq{}}\PY{p}{,}\PY{+w}{ }\PY{o}{\PYZam{}}\PY{n}{a}\PY{p}{,}\PY{+w}{ }\PY{o}{\PYZam{}}\PY{n}{b}\PY{p}{)}\PY{p}{;}
\PY{+w}{	}\PY{n}{printf}\PY{p}{(}\PY{l+s}{\PYZdq{}}\PY{l+s}{\PYZpc{}d}\PY{l+s+se}{\PYZbs{}n}\PY{l+s}{\PYZdq{}}\PY{p}{,}\PY{+w}{ }\PY{n}{a}\PY{o}{/}\PY{n}{b}\PY{p}{)}\PY{p}{;}
\PY{+w}{	}\PY{k}{return}\PY{+w}{ }\PY{l+m+mi}{0}\PY{p}{;}
\PY{p}{\PYZcb{}}
\end{Verbatim}


\begin{inout}
16 8
\tcblower
2
\end{inout}

\end{examples}

\begin{errors}
Deljenje v jezikih C in C++ je celoštevilsko. Deljenje z ostankom lahko zapišemo po formuli: $a = k*b + o$.

16/8: 16 = 2*8 + 0 \\
Ostanek je 0, ker je 16 deljivo z 8. \\\\

16/5: 15 = 3*5 + 1 \\
1 je ostanek. \\\\

Celoštevilsko deljenje pomeni, da nam program vnre samo $k$. Primer vhoda in izhoda za zgornji program, kjer števili nista deljivi:

\begin{inout}
16 3
\tcblower
5
\end{inout}

Tudi, če bi bil rezultat deljenja 7.9, tega program ne zaokroži na 8, temveč nam vrne 7. \\
Če manjše število delimo z večjim, zato vedno dobimo 0.

\end{errors}

\pagebreak
Operator \verb+/+ nam torej iz pri deljenju \verb+a/b+ vrne \verb+k+. Lahko pa dobimo tudi ostanek \verb+o+. Do njega pridemo z operatorjem \verb+%+ (\emph{modulo}).

\begin{examples}
\begin{Verbatim}[commandchars=\\\{\}]
\PY{c+cp}{\PYZsh{}}\PY{c+cp}{include}\PY{c+cpf}{\PYZlt{}stdio.h\PYZgt{}}

\PY{k+kt}{int}\PY{+w}{ }\PY{n+nf}{main}\PY{p}{(}\PY{p}{)}\PY{p}{\PYZob{}}
\PY{+w}{	}\PY{k+kt}{int}\PY{+w}{ }\PY{n}{a}\PY{p}{,}\PY{+w}{ }\PY{n}{b}\PY{p}{;}
\PY{+w}{	}\PY{n}{scanf}\PY{p}{(}\PY{l+s}{\PYZdq{}}\PY{l+s}{\PYZpc{}d\PYZpc{}d}\PY{l+s}{\PYZdq{}}\PY{p}{,}\PY{+w}{ }\PY{o}{\PYZam{}}\PY{n}{a}\PY{p}{,}\PY{+w}{ }\PY{o}{\PYZam{}}\PY{n}{b}\PY{p}{)}\PY{p}{;}
\PY{+w}{	}\PY{n}{printf}\PY{p}{(}\PY{l+s}{\PYZdq{}}\PY{l+s}{\PYZpc{}d = \PYZpc{}d*\PYZpc{}d + \PYZpc{}d}\PY{l+s+se}{\PYZbs{}n}\PY{l+s}{\PYZdq{}}\PY{p}{,}\PY{+w}{ }\PY{n}{a}\PY{p}{,}\PY{+w}{ }\PY{n}{a}\PY{o}{/}\PY{n}{b}\PY{p}{,}\PY{+w}{ }\PY{n}{b}\PY{p}{,}\PY{+w}{ }\PY{n}{a}\PY{o}{\PYZpc{}}\PY{n}{b}\PY{p}{)}\PY{p}{;}
\PY{+w}{	}\PY{n}{printf}\PY{p}{(}\PY{l+s}{\PYZdq{}}\PY{l+s}{Količnik: \PYZpc{}d}\PY{l+s+se}{\PYZbs{}n}\PY{l+s}{Ostanek: \PYZpc{}d}\PY{l+s+se}{\PYZbs{}n}\PY{l+s}{\PYZdq{}}\PY{p}{,}\PY{+w}{ }\PY{n}{a}\PY{o}{/}\PY{n}{b}\PY{p}{,}\PY{+w}{ }\PY{n}{a}\PY{o}{\PYZpc{}}\PY{n}{b}\PY{p}{)}\PY{p}{;}
\PY{+w}{	}\PY{k}{return}\PY{+w}{ }\PY{l+m+mi}{0}\PY{p}{;}
\PY{p}{\PYZcb{}}
\end{Verbatim}

\begin{inout}
25 7
\tcblower
25 = 3*7 + 4 \\
Količnik: 3 \\
Ostanek: 4
\end{inout}

\end{examples}

\begin{errors}
Deljenje z 0 v matematiki ni definirano in prav tako ne v programiranju. Če neko število delimo z 0, se nam bo program sesul. Prav tako, če poskušamo izračunati ostanek pri deljenju z 0. \\
Ta napaka se pogosto zgodi v programih, kjer delimo z več števili, zato moramo biti na to pozorni.
\end{errors}

\end{document}
