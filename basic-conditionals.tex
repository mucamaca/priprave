\documentclass{article}
\usepackage{fancyvrb}
\usepackage{color}
\usepackage[utf8]{inputenc}
\usepackage[a4paper, total={6.2in, 9in}]{geometry}
\usepackage[many]{tcolorbox}
\usepackage[T1]{fontenc}
\usepackage[slovene]{babel}

\usetikzlibrary{calc}

\setlength\parindent{0pt}

\makeatletter
\def\PY@reset{\let\PY@it=\relax \let\PY@bf=\relax%
    \let\PY@ul=\relax \let\PY@tc=\relax%
    \let\PY@bc=\relax \let\PY@ff=\relax}
\def\PY@tok#1{\csname PY@tok@#1\endcsname}
\def\PY@toks#1+{\ifx\relax#1\empty\else%
    \PY@tok{#1}\expandafter\PY@toks\fi}
\def\PY@do#1{\PY@bc{\PY@tc{\PY@ul{%
    \PY@it{\PY@bf{\PY@ff{#1}}}}}}}
\def\PY#1#2{\PY@reset\PY@toks#1+\relax+\PY@do{#2}}

\@namedef{PY@tok@c}{\def\PY@tc##1{\textcolor[rgb]{0.63,0.63,0.88}{##1}}}
\@namedef{PY@tok@cp}{\def\PY@tc##1{\textcolor[rgb]{0.19,0.51,0.25}{##1}}}
\@namedef{PY@tok@k}{\let\PY@bf=\textbf\def\PY@tc##1{\textcolor[rgb]{0.00,0.00,0.67}{##1}}}
\@namedef{PY@tok@o}{\def\PY@tc##1{\textcolor[rgb]{1.00,0.02,0.02}{##1}}}
\@namedef{PY@tok@p}{\def\PY@tc##1{\textcolor[rgb]{1.00,0.02,0.02}{##1}}}
\@namedef{PY@tok@n}{\def\PY@tc##1{\textcolor[rgb]{0.05,0.05,0.05}{##1}}}
\@namedef{PY@tok@s}{\def\PY@tc##1{\textcolor[rgb]{0.16,0.16,1.00}{##1}}}
\@namedef{PY@tok@m}{\def\PY@tc##1{\textcolor[rgb]{0.94,0.03,0.94}{##1}}}
\@namedef{PY@tok@kc}{\let\PY@bf=\textbf\def\PY@tc##1{\textcolor[rgb]{0.00,0.00,0.67}{##1}}}
\@namedef{PY@tok@kd}{\let\PY@bf=\textbf\def\PY@tc##1{\textcolor[rgb]{0.00,0.00,0.67}{##1}}}
\@namedef{PY@tok@kn}{\let\PY@bf=\textbf\def\PY@tc##1{\textcolor[rgb]{0.00,0.00,0.67}{##1}}}
\@namedef{PY@tok@kp}{\let\PY@bf=\textbf\def\PY@tc##1{\textcolor[rgb]{0.00,0.00,0.67}{##1}}}
\@namedef{PY@tok@kr}{\let\PY@bf=\textbf\def\PY@tc##1{\textcolor[rgb]{0.00,0.00,0.67}{##1}}}
\@namedef{PY@tok@kt}{\let\PY@bf=\textbf\def\PY@tc##1{\textcolor[rgb]{0.00,0.00,0.67}{##1}}}
\@namedef{PY@tok@na}{\def\PY@tc##1{\textcolor[rgb]{0.05,0.05,0.05}{##1}}}
\@namedef{PY@tok@nb}{\def\PY@tc##1{\textcolor[rgb]{0.05,0.05,0.05}{##1}}}
\@namedef{PY@tok@bp}{\def\PY@tc##1{\textcolor[rgb]{0.05,0.05,0.05}{##1}}}
\@namedef{PY@tok@nc}{\def\PY@tc##1{\textcolor[rgb]{0.05,0.05,0.05}{##1}}}
\@namedef{PY@tok@no}{\def\PY@tc##1{\textcolor[rgb]{0.05,0.05,0.05}{##1}}}
\@namedef{PY@tok@nd}{\def\PY@tc##1{\textcolor[rgb]{0.05,0.05,0.05}{##1}}}
\@namedef{PY@tok@ni}{\def\PY@tc##1{\textcolor[rgb]{0.05,0.05,0.05}{##1}}}
\@namedef{PY@tok@ne}{\def\PY@tc##1{\textcolor[rgb]{0.05,0.05,0.05}{##1}}}
\@namedef{PY@tok@nf}{\def\PY@tc##1{\textcolor[rgb]{0.05,0.05,0.05}{##1}}}
\@namedef{PY@tok@fm}{\def\PY@tc##1{\textcolor[rgb]{0.05,0.05,0.05}{##1}}}
\@namedef{PY@tok@py}{\def\PY@tc##1{\textcolor[rgb]{0.05,0.05,0.05}{##1}}}
\@namedef{PY@tok@nl}{\def\PY@tc##1{\textcolor[rgb]{0.05,0.05,0.05}{##1}}}
\@namedef{PY@tok@nn}{\def\PY@tc##1{\textcolor[rgb]{0.05,0.05,0.05}{##1}}}
\@namedef{PY@tok@nx}{\def\PY@tc##1{\textcolor[rgb]{0.05,0.05,0.05}{##1}}}
\@namedef{PY@tok@nt}{\def\PY@tc##1{\textcolor[rgb]{0.05,0.05,0.05}{##1}}}
\@namedef{PY@tok@nv}{\def\PY@tc##1{\textcolor[rgb]{0.05,0.05,0.05}{##1}}}
\@namedef{PY@tok@vc}{\def\PY@tc##1{\textcolor[rgb]{0.05,0.05,0.05}{##1}}}
\@namedef{PY@tok@vg}{\def\PY@tc##1{\textcolor[rgb]{0.05,0.05,0.05}{##1}}}
\@namedef{PY@tok@vi}{\def\PY@tc##1{\textcolor[rgb]{0.05,0.05,0.05}{##1}}}
\@namedef{PY@tok@vm}{\def\PY@tc##1{\textcolor[rgb]{0.05,0.05,0.05}{##1}}}
\@namedef{PY@tok@sa}{\def\PY@tc##1{\textcolor[rgb]{0.16,0.16,1.00}{##1}}}
\@namedef{PY@tok@sb}{\def\PY@tc##1{\textcolor[rgb]{0.16,0.16,1.00}{##1}}}
\@namedef{PY@tok@sc}{\def\PY@tc##1{\textcolor[rgb]{0.16,0.16,1.00}{##1}}}
\@namedef{PY@tok@dl}{\def\PY@tc##1{\textcolor[rgb]{0.16,0.16,1.00}{##1}}}
\@namedef{PY@tok@sd}{\def\PY@tc##1{\textcolor[rgb]{0.16,0.16,1.00}{##1}}}
\@namedef{PY@tok@s2}{\def\PY@tc##1{\textcolor[rgb]{0.16,0.16,1.00}{##1}}}
\@namedef{PY@tok@se}{\def\PY@tc##1{\textcolor[rgb]{0.16,0.16,1.00}{##1}}}
\@namedef{PY@tok@sh}{\def\PY@tc##1{\textcolor[rgb]{0.16,0.16,1.00}{##1}}}
\@namedef{PY@tok@si}{\def\PY@tc##1{\textcolor[rgb]{0.16,0.16,1.00}{##1}}}
\@namedef{PY@tok@sx}{\def\PY@tc##1{\textcolor[rgb]{0.16,0.16,1.00}{##1}}}
\@namedef{PY@tok@sr}{\def\PY@tc##1{\textcolor[rgb]{0.16,0.16,1.00}{##1}}}
\@namedef{PY@tok@s1}{\def\PY@tc##1{\textcolor[rgb]{0.16,0.16,1.00}{##1}}}
\@namedef{PY@tok@ss}{\def\PY@tc##1{\textcolor[rgb]{0.16,0.16,1.00}{##1}}}
\@namedef{PY@tok@mb}{\def\PY@tc##1{\textcolor[rgb]{0.94,0.03,0.94}{##1}}}
\@namedef{PY@tok@mf}{\def\PY@tc##1{\textcolor[rgb]{0.94,0.03,0.94}{##1}}}
\@namedef{PY@tok@mh}{\def\PY@tc##1{\textcolor[rgb]{0.94,0.03,0.94}{##1}}}
\@namedef{PY@tok@mi}{\def\PY@tc##1{\textcolor[rgb]{0.94,0.03,0.94}{##1}}}
\@namedef{PY@tok@il}{\def\PY@tc##1{\textcolor[rgb]{0.94,0.03,0.94}{##1}}}
\@namedef{PY@tok@mo}{\def\PY@tc##1{\textcolor[rgb]{0.94,0.03,0.94}{##1}}}
\@namedef{PY@tok@ow}{\def\PY@tc##1{\textcolor[rgb]{1.00,0.02,0.02}{##1}}}
\@namedef{PY@tok@ch}{\def\PY@tc##1{\textcolor[rgb]{0.63,0.63,0.88}{##1}}}
\@namedef{PY@tok@cm}{\def\PY@tc##1{\textcolor[rgb]{0.63,0.63,0.88}{##1}}}
\@namedef{PY@tok@cpf}{\def\PY@tc##1{\textcolor[rgb]{0.63,0.63,0.88}{##1}}}
\@namedef{PY@tok@c1}{\def\PY@tc##1{\textcolor[rgb]{0.63,0.63,0.88}{##1}}}
\@namedef{PY@tok@cs}{\def\PY@tc##1{\textcolor[rgb]{0.63,0.63,0.88}{##1}}}

\def\PYZbs{\char`\\}
\def\PYZus{\char`\_}
\def\PYZob{\char`\{}
\def\PYZcb{\char`\}}
\def\PYZca{\char`\^}
\def\PYZam{\char`\&}
\def\PYZlt{\char`\<}
\def\PYZgt{\char`\>}
\def\PYZsh{\char`\#}
\def\PYZpc{\char`\%}
\def\PYZdl{\char`\$}
\def\PYZhy{\char`\-}
\def\PYZsq{\char`\'}
\def\PYZdq{\char`\"}
\def\PYZti{\char`\~}
% for compatibility with earlier versions
\def\PYZat{@}
\def\PYZlb{[}
\def\PYZrb{]}
\makeatother

\definecolor{myblue}{RGB}{0,163,243}
\definecolor{myred}{RGB}{243, 10, 25}
\definecolor{mygreen}{RGB}{50, 205, 50}

\newcommand{\fon}[1]{\fontfamily{#1}\selectfont}


\tcbset{examplestyle/.style={
  enhanced,
  outer arc=4pt,
  arc=4pt,
  colframe=myblue,
  colback=myblue!20,
  attach boxed title to top left,
  boxed title style={
    colback=myblue,
    outer arc=4pt,
    arc=4pt,
    top=3pt,
    bottom=3pt,
    },
  fonttitle=\sffamily
  }
}

\tcbset{inoutstyle/.style={
  enhanced,
  outer arc=4pt,
  arc=4pt,
  colframe=mygreen,
  colback=mygreen!20,
  attach boxed title to top left,
  boxed title style={
    colback=mygreen,
    outer arc=4pt,
    arc=4pt,
    top=3pt,
    bottom=3pt,
    },
  fonttitle=\sffamily,
  fontupper=\ttfamily,
  fontlower=\ttfamily,
  }
}

\tcbset{errorstyle/.style={
  enhanced,
  outer arc=4pt,
  arc=4pt,
  colframe=myred,
  colback=myred!20,
  attach boxed title to top left,
  boxed title style={
    colback=myred,
    outer arc=4pt,
    arc=4pt,
    top=3pt,
    bottom=3pt,
    },
  fonttitle=\sffamily
  }
}

\newtcolorbox[auto counter,number within=section]{examples}[1][]{
  examplestyle,
  colback=white,
  title=Primer,
  overlay unbroken and first={
      \path
        let
        \p1=(title.north east),
        \p2=(frame.north east)
        in
        node[anchor=west,font=\sffamily,color=myblue,text width=\x2-\x1]
        at (title.east) {#1};
  }
}
\newtcolorbox[auto counter]{errors}[1][]{
  errorstyle,
  colback=white,
  title=Pogoste napake,
  overlay unbroken and first={
      \path
        let
        \p1=(title.north east),
        \p2=(frame.north east)
        in
        node[anchor=west,font=\sffamily,color=myblue,text width=\x2-\x1]
        at (title.east) {#1};
  }
}

\newtcolorbox[auto counter]{inout}[1][]{
  inoutstyle,
  colback=white,
  title=Primer vhoda in izhoda,
  overlay unbroken and first={
      \path
        let
        \p1=(title.north east),
        \p2=(frame.north east)
        in
        node[anchor=west,font=\sffamily,color=myblue,text width=\x2-\x1]
        at (title.east) {#1};
  }
}
\title{Pogojni stavki}
\date{}

\begin{document}
\maketitle

Pogosto želimo, da računalnik izvaja drugačno kodo glede na vrednost ene ali
večih spremenljivk, npr. da nam pokaže drugačno vsebino, če smo napisali
pravilno ali napačno geslo, da računalo sešteva, če smo stisnili gumb za
seštevanje, oz. odšteva, če smo stisnili gumb za odštevanje, ipd. Z drugimi
besedami, želimo upravljati potek programa (torej izbrati, katera koda naj
se izvede) glede na vrednosti spremenljivk. Angleško takemu upravljanju
pravimo \emph{control flow}, najpogosteje pa ga izvajamo s t.i. \emph{pogojnimi}
ali \emph{if stavki}. Osnovna sintaksa je sledeča:

\begin{examples}
\begin{Verbatim}[commandchars=\\\{\}]
\PY{k}{if} \PY{p}{(}\PY{n}{pogoj}\PY{p}{)} \PY{p}{\PYZob{}}
    \PY{c+c1}{// koda, ki se izvede, ce pogoj velja}
\PY{p}{\PYZcb{}} \PY{k}{else} \PY{p}{\PYZob{}}
    \PY{c+c1}{// koda, ki se izvede, ce pogoj ne velja}
\PY{p}{\PYZcb{}}
\end{Verbatim}
\end{examples}

\emph{Pogoj} je nov pojem. Označuje neke vrste račun, katerega rezultat ni
število, vendar \emph{logična vrednost}. Tu sta možni vrednosti le dve:
pravilno (angl.~\verb+true+) in napačno (angl.~\verb+false+). Če bo rezultat
računa, navedenega v običajnih oklepajih v zgornjem \verb+if+ stavku, pravilno,
se bo izvedla koda znotraj prvih zavitih oklepajev, če pa je rezultat računa
napačno, pa se bo izvedla koda v drugih zavitih oklepajih (tistih za besedo
\verb+else+). Drugi del, t.j. \verb+else+ in oklepaji za njim, ni potrebno
pisati, če ga ne želimo.

Kako pa zapišemo pogoj? Za to uporabimo posebne logične operatorje. Pri delu s
številkami so nam na voljo naslednji:
\begin{itemize}
  \item \verb+==+: primerja dve številski vrednosti.
	Rezultat je pravilen, če sta vrednosti enaki.
  \item \verb+!=+: primerja dve številski vrednosti.
	Rezultat je pravilen, če sta vrednosti različni.
  \item \verb+<+: primerja dve številski vrednosti.
	Rezultat je pravilen, če je vrednost na levi manjša od vrednosti na desni.
  \item \verb+>+: deluje podobno kot \verb+<+, le da v drugo smer;
	rezultat je pravilen, če je vrednost na desni manjša od vrednosti na levi.
  \item \verb+<=+: primerja dve številski vrednosti.
	Rezultat je pravilen, če sta vrednosti enaki, ali če je vrednost
	na levi manjša od vrednosti na desni.
  \item \verb+>=+ deluje podobno kot \verb+<=+, le da v drugo smer.
\end{itemize}

\begin{examples}

Poglejmo si primer uporabe pogojnega stavka.

\begin{Verbatim}[commandchars=\\\{\}]
\PY{c+cp}{\PYZsh{}}\PY{c+cp}{include} \PY{c+cpf}{\PYZlt{}stdio.h\PYZgt{}}

\PY{k+kt}{int} \PY{n+nf}{main}\PY{p}{(}\PY{p}{)} \PY{p}{\PYZob{}}
    \PY{k+kt}{int} \PY{n}{a}\PY{p}{,} \PY{n}{b}\PY{p}{;}
    \PY{n}{scanf}\PY{p}{(}\PY{l+s}{\PYZdq{}}\PY{l+s}{\PYZpc{}d\PYZpc{}d}\PY{l+s}{\PYZdq{}}\PY{p}{,} \PY{o}{\PYZam{}}\PY{n}{a}\PY{p}{,} \PY{o}{\PYZam{}}\PY{n}{b}\PY{p}{)}\PY{p}{;}
        
    \PY{k}{if} \PY{p}{(}\PY{n}{a} \PY{o}{=}\PY{o}{=} \PY{n}{b}\PY{p}{)} \PY{p}{\PYZob{}}
        \PY{n}{printf}\PY{p}{(}\PY{l+s}{\PYZdq{}}\PY{l+s}{Stevili sta enaki.}\PY{l+s+se}{\PYZbs{}n}\PY{l+s}{\PYZdq{}}\PY{p}{)}\PY{p}{;}
    \PY{p}{\PYZcb{}}
    \PY{k}{if} \PY{p}{(}\PY{n}{a} \PY{o}{!}\PY{o}{=} \PY{n}{b}\PY{p}{)} \PY{p}{\PYZob{}}
        \PY{n}{printf}\PY{p}{(}\PY{l+s}{\PYZdq{}}\PY{l+s}{Stevili sta razlicni.}\PY{l+s+se}{\PYZbs{}n}\PY{l+s}{\PYZdq{}}\PY{p}{)}\PY{p}{;}
    \PY{p}{\PYZcb{}}
    \PY{k}{if} \PY{p}{(}\PY{n}{a} \PY{o}{\PYZlt{}} \PY{n}{b}\PY{p}{)} \PY{p}{\PYZob{}}
        \PY{n}{printf}\PY{p}{(}\PY{l+s}{\PYZdq{}}\PY{l+s}{Prvo stevilo je manjse od drugega.}\PY{l+s+se}{\PYZbs{}n}\PY{l+s}{\PYZdq{}}\PY{p}{)}\PY{p}{;}
    \PY{p}{\PYZcb{}}
    \PY{k}{if} \PY{p}{(}\PY{n}{a} \PY{o}{\PYZgt{}} \PY{n}{b}\PY{p}{)} \PY{p}{\PYZob{}}
        \PY{n}{printf}\PY{p}{(}\PY{l+s}{\PYZdq{}}\PY{l+s}{Prvo stevilo je vecje od drugega.}\PY{l+s+se}{\PYZbs{}n}\PY{l+s}{\PYZdq{}}\PY{p}{)}\PY{p}{;}
    \PY{p}{\PYZcb{}}
    \PY{k}{if} \PY{p}{(}\PY{n}{a} \PY{o}{\PYZlt{}}\PY{o}{=} \PY{n}{b}\PY{p}{)} \PY{p}{\PYZob{}}
        \PY{n}{printf}\PY{p}{(}\PY{l+s}{\PYZdq{}}\PY{l+s}{Prvo stevilo je manjse ali enako drugemu.}\PY{l+s+se}{\PYZbs{}n}\PY{l+s}{\PYZdq{}}\PY{p}{)}\PY{p}{;}
    \PY{p}{\PYZcb{}}
    \PY{k}{if} \PY{p}{(}\PY{n}{a} \PY{o}{\PYZgt{}}\PY{o}{=} \PY{n}{b}\PY{p}{)} \PY{p}{\PYZob{}}
        \PY{n}{printf}\PY{p}{(}\PY{l+s}{\PYZdq{}}\PY{l+s}{Prvo stevilo je vecje ali enako drugemu.}\PY{l+s+se}{\PYZbs{}n}\PY{l+s}{\PYZdq{}}\PY{p}{)}\PY{p}{;}
    \PY{p}{\PYZcb{}}
    \PY{k}{return} \PY{l+m+mi}{0}\PY{p}{;}
\PY{p}{\PYZcb{}}
\end{Verbatim}

\begin{inout}
12 12
\tcblower
Stevili sta enaki.\\
Prvo stevilo je manjse ali enako drugemu.\\
Prvo stevilo je vecje ali enako drugemu.
\end{inout}

\begin{inout}
3 7
\tcblower
Stevili sta razlicni.\\
Prvo stevilo je manjse od drugega.\\
Prvo stevilo je majnse ali enako drugemu.
\end{inout}

\end{examples}


\begin{errors}
  Pri primerjavi moramo uporabiti dva enačaja (\verb+==+). Če uporabimo samo
  en enačaj, kot v matematiki (torej \verb+=+), se bo program sicer zagnal,
  vendar ne bo deloval pravilno.
  Enojni enačaj uporabljamo samo pri nastavljanju vrednosti neki spremenljivki,
  in nikoli v običajnih oklepajih v \verb+if+ stavku.
\end{errors}

\begin{errors}
  % Pri predelavi v učbenik: pogovor, če je smiselno že tu omenjati strcmp
  Taka primerjava deluje samo na številkah. Če želimo primerjati med dvema
  besediloma, za to potrebujemo posebno funkcijo \verb+strcmp+, ki jo bomo
  podrobneje spoznali, ko bomo govorili o besedilu.
\end{errors}

\begin{examples}

Primer uporabe stavka \verb+else+. Program bo uporabnika vprašal za PIN,
in mu napisal, če je bil PIN pravilen oziroma napačen.

\begin{Verbatim}[commandchars=\\\{\}]
\PY{c+cp}{\PYZsh{}}\PY{c+cp}{include} \PY{c+cpf}{\PYZlt{}stdio.h\PYZgt{}}

\PY{k+kt}{int} \PY{n+nf}{main}\PY{p}{(}\PY{p}{)} \PY{p}{\PYZob{}}
    \PY{c+c1}{// Vprasaj uporabnika za PIN, in preveri, ce je pravilen}
    \PY{k+kt}{int} \PY{n}{pin}\PY{p}{;}
    \PY{n}{scanf}\PY{p}{(}\PY{l+s}{\PYZdq{}}\PY{l+s}{\PYZpc{}d}\PY{l+s}{\PYZdq{}}\PY{p}{,} \PY{o}{\PYZam{}}\PY{n}{pin}\PY{p}{)}\PY{p}{;}
    \PY{k}{if} \PY{p}{(}\PY{n}{pin} \PY{o}{=}\PY{o}{=} \PY{l+m+mi}{42}\PY{p}{)} \PY{p}{\PYZob{}}
        \PY{n}{printf}\PY{p}{(}\PY{l+s}{\PYZdq{}}\PY{l+s}{PIN je pravilen!}\PY{l+s}{\PYZdq{}}\PY{p}{)}\PY{p}{;}
    \PY{p}{\PYZcb{}} \PY{k}{else} \PY{p}{\PYZob{}}
        \PY{n}{printf}\PY{p}{(}\PY{l+s}{\PYZdq{}}\PY{l+s}{PIN je napacen!}\PY{l+s}{\PYZdq{}}\PY{p}{)}\PY{p}{;}
    \PY{p}{\PYZcb{}}
    \PY{k}{return} \PY{l+m+mi}{0}\PY{p}{;}
\PY{p}{\PYZcb{}}
\end{Verbatim}

\begin{inout}
42
\tcblower
PIN je pravilen!
\end{inout}

\begin{inout}
7
\tcblower
PIN je napacen!
\end{inout}

\end{examples}

\begin{examples}
\verb+if+ stavke lahko tudi gnezdimo; torej vstavimo enega v drugega.
Spodnji program od uporabnika sprejme naročilo v restavraciji, kjer ponujajo
dve vrsti hrane; juhe in sendviče. Na voljo sta dve vrsti juhe, in dve vrsti
sendvičev.

\begin{Verbatim}[commandchars=\\\{\}]
\PY{c+cp}{\PYZsh{}}\PY{c+cp}{include} \PY{c+cpf}{\PYZlt{}stdio.h\PYZgt{}}

\PY{k+kt}{int} \PY{n+nf}{main}\PY{p}{(}\PY{p}{)} \PY{p}{\PYZob{}}
    \PY{c+c1}{// Uporabnik naj napise 1, ce zeli juho, in 2, ce zeli sendvic.}
    \PY{k+kt}{int} \PY{n}{zelja}\PY{p}{;}
    \PY{n}{scanf}\PY{p}{(}\PY{l+s}{\PYZdq{}}\PY{l+s}{\PYZpc{}d}\PY{l+s}{\PYZdq{}}\PY{p}{,} \PY{o}{\PYZam{}}\PY{n}{zelja}\PY{p}{)}\PY{p}{;}

    \PY{k}{if} \PY{p}{(}\PY{n}{zelja} \PY{o}{=}\PY{o}{=} \PY{l+m+mi}{1}\PY{p}{)} \PY{p}{\PYZob{}}
        \PY{c+c1}{// Uporabnik naj napise 1, ce zeli govejo juho,}
        \PY{c+c1}{// in 2, ce zeli paradiznikovo.}
        \PY{n}{scanf}\PY{p}{(}\PY{l+s}{\PYZdq{}}\PY{l+s}{\PYZpc{}d}\PY{l+s}{\PYZdq{}}\PY{p}{,} \PY{n}{zelja}\PY{p}{)}\PY{p}{;}
        \PY{k}{if} \PY{p}{(}\PY{n}{zelja} \PY{o}{=}\PY{o}{=} \PY{l+m+mi}{1}\PY{p}{)} \PY{p}{\PYZob{}}
            \PY{n}{printf}\PY{p}{(}\PY{l+s}{\PYZdq{}}\PY{l+s}{Ena goveja juha. Dober tek!}\PY{l+s+se}{\PYZbs{}n}\PY{l+s}{\PYZdq{}}\PY{p}{)}\PY{p}{;}
        \PY{p}{\PYZcb{}} \PY{k}{else} \PY{p}{\PYZob{}}
            \PY{n}{printf}\PY{p}{(}\PY{l+s}{\PYZdq{}}\PY{l+s}{Ena paradiznikova juha. Dober tek!}\PY{l+s+se}{\PYZbs{}n}\PY{l+s}{\PYZdq{}}\PY{p}{)}\PY{p}{;}
        \PY{p}{\PYZcb{}}
    \PY{p}{\PYZcb{}} \PY{k}{else} \PY{p}{\PYZob{}}
        \PY{c+c1}{// Uporabnik naj napise 1, ce zeli sendvic s sunko,}
        \PY{c+c1}{// in 2, ce zeli vegeterjanski sendvic.}
        \PY{n}{scanf}\PY{p}{(}\PY{l+s}{\PYZdq{}}\PY{l+s}{\PYZpc{}d}\PY{l+s}{\PYZdq{}}\PY{p}{,} \PY{o}{\PYZam{}}\PY{n}{zelja}\PY{p}{)}\PY{p}{;}
        \PY{k}{if} \PY{p}{(}\PY{n}{zelja} \PY{o}{=}\PY{o}{=} \PY{l+m+mi}{1}\PY{p}{)} \PY{p}{\PYZob{}}
            \PY{n}{printf}\PY{p}{(}\PY{l+s}{\PYZdq{}}\PY{l+s}{En sendvic s sunko. Dober tek!}\PY{l+s+se}{\PYZbs{}n}\PY{l+s}{\PYZdq{}}\PY{p}{)}\PY{p}{;}
        \PY{p}{\PYZcb{}} \PY{k}{else} \PY{p}{\PYZob{}}
            \PY{n}{printf}\PY{p}{(}\PY{l+s}{\PYZdq{}}\PY{l+s}{En vegeterjanski sendvic. Dober tek!}\PY{l+s+se}{\PYZbs{}n}\PY{l+s}{\PYZdq{}}\PY{p}{)}\PY{p}{;}
        \PY{p}{\PYZcb{}}
    \PY{p}{\PYZcb{}}
    \PY{k}{return} \PY{l+m+mi}{0}\PY{p}{;}
\PY{p}{\PYZcb{}}
\end{Verbatim}

\begin{inout}
2\\
1
\tcblower
En sendvic s sunko. Dober tek!
\end{inout}
\end{examples}


\begin{errors}

Pozorni bodimo tudi na postavitev kode. Običajno kodo znotraj zavitih oklepajev
\verb+if+ stavka pišemo tako, da je poravnana štiri presledke bolj desno od
kode zunaj \verb+if+ stavka. V nekaterih programskih jezikih je taka poravnava
obvezna, v C/C++ pa ne, vendar nezamaknjena koda že v zelo majhnih programih
postane popolnoma nepregledna. Branje in popravljanje kode je veliko lažje, če
del kode znotraj zavitih oklepajev zamaknemo za štiri presledke. Za to lahko
uporabimo tudi tipko Tab, ki se na tipkovnici nahaja levo od tipke Q.

\vspace{0.3cm}
Tako \textbf{nikoli} ne pišemo:

\begin{Verbatim}[commandchars=\\\{\}]
\PY{c+cp}{\PYZsh{}}\PY{c+cp}{include} \PY{c+cpf}{\PYZlt{}stdio.h\PYZgt{}}
\PY{k+kt}{int} \PY{n+nf}{main}\PY{p}{(}\PY{p}{)} \PY{p}{\PYZob{}}
\PY{k}{if} \PY{p}{(}\PY{l+m+mi}{3} \PY{o}{\PYZgt{}} \PY{l+m+mi}{2}\PY{p}{)} \PY{p}{\PYZob{}}
\PY{n}{printf}\PY{p}{(}\PY{l+s}{\PYZdq{}}\PY{l+s}{Velja}\PY{l+s+se}{\PYZbs{}n}\PY{l+s}{\PYZdq{}}\PY{p}{)}\PY{p}{;}
\PY{p}{\PYZcb{}} \PY{k}{else} \PY{p}{\PYZob{}}
\PY{n}{printf}\PY{p}{(}\PY{l+s}{\PYZdq{}}\PY{l+s}{Ne velja}\PY{l+s+se}{\PYZbs{}n}\PY{l+s}{\PYZdq{}}\PY{p}{)}\PY{p}{;}
\PY{p}{\PYZcb{}}
\PY{k}{return} \PY{l+m+mi}{0}\PY{p}{;}
\PY{p}{\PYZcb{}}
\end{Verbatim}

\end{errors}

\end{document}
