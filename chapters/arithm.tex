%#template templates/template.tex

%#block title
Računske operacije
%#endblock

%#block content
\section{Seštevanje, odštevanje, množenje}
Računalniki lahko s spremenljivkami počnejo veliko stvari. Najpreprostejše so  operacije na številih, kot so seštevanje, odštevanje in množenje. Račune zapisujemo tako kot v šoli, z \emph{operatorji}.
\begin{itemize}
	\item \verb-+- za seštevanje
	\item \verb+-+ za odštevanje
	\item \verb+*+ za množenje
\end{itemize}

Rezultat lahko izračunamo kar znotraj funkcije \verb+printf+:


\begin{examples}

%#insert python3 style.py < snippets/arithm/sestevanje.cpp

\begin{inout}

\tcblower
12
\end{inout}


\end{examples}

\pagebreak
Rezultat lahko tudi shranimo v novo spremenljivko:
\begin{examples}

%#insert python3 style.py < snippets/arithm/nova-spremenljivka.cpp

\begin{inout}
3 7
\tcblower
12\\
-4\\
21
\end{inout}

\end{examples}

\begin{errors}
Spremenljivko \emph{inicializiramo} tako, da notri nekaj shranimo, bodisi kot \verb+a=5+, bodisi s tem, da vanjo nekaj napišemo s funkcijo \verb+scanf+.
Če je ne inicializiramo, pozneje pa jo poskusimo uporabiti za izpisovanje, računanje ali kaj drugega, lahko dobimo zelo čudne rezultate, naš program se lahko celo sesuje.
\end{errors}

%CHECK
V zgornjem programu tudi vidimo, da lahko z enim klicem funkcije \verb+scanf+ preberemo več spremenljivk.

\subsection*{Negativna števila}
Na meteorološki postaji Kredarica so leta 2014 izmerili povprečno januarsko temperaturo približno -5 °C, povprečno avgustovsko pa približno 6 °C.
Med tema meritvama je 11 °C razlike. \\
Pozimi lahko izmerimo temperature manjše od 0. Takšnim številom, kot je -5, rečemo \emph{negativna števila}. Lahko jih uporabimo tudi drugje, ne samo pri merjenju temperature. \\
S pozitivnimi števili lahko štejemo od 0 do neskončno (1, 2, 3, ...), z negativnimi pa do - neskončno. (-1, -2, -3, ...). Tako kot pozitivna števila jih lahko seštevamo in odštevamo:

\begin{examples}
5 - 11 = -6 \\
-6 + 11 = 5 \\
5 -(-6) = 11 \\
-2 - 1 = -3 \\ 
-2 - (-1) = -1
-2 + (-1) = -3 \\\\
\emph{Pravila:}\\
-(-6) = +6\\
+(-1) = -1 \\
-(-(-(-1))) = -(-(+1)) = -(-1) = 1
\end{examples}

Lahko jih tudi množimo:

\begin{examples}
2 * (-5) = -10 \\
(-5) * (-5) = 25 \\

\emph{Pravila:}
Če množimo dve pozitivni števili, je produkt pozitiven. \\
Če množimo eno pozitivno in eno negativno število, je produkt negativen. \\
Če množimo dve negativni števili, je produkt spet negativen. \\
O negativnih številih lahko razmišljamo kot: -3 = (-1) * 3
\end{examples}

Računalniki z negativnimi števili računajo enako kot s pozitivnimi:

\begin{examples}

%#insert python3 style.py < snippets/arithm/negativna-stevila.cpp

\begin{inout}
3 7
\tcblower
Vsota: 10 \\
Razlika: -4 \\
Produkt: 21
\end{inout}

\end{examples}

\pagebreak
\section{Deljenje}
Števila lahko tudi delimo. Za deljenje uporabljamo znak \verb+/+.
Za razumevanje poglavja si bomo pomagali s formulo $a = k*b + o$, ki ponazarja deljenje z ostankom.

\begin{examples}

%#insert python3 style.py < snippets/arithm/deljenje.cpp

\begin{inout}
16 8
\tcblower
2
\end{inout}

\end{examples}

\begin{errors}
Deljenje v jezikih C in C++ je celoštevilsko. Deljenje z ostankom lahko zapišemo po formuli: $a = k*b + o$.

16/8: 16 = 2*8 + 0 \\
Ostanek je 0, ker je 16 deljivo z 8. \\\\

16/5: 15 = 3*5 + 1 \\
1 je ostanek. \\\\

Celoštevilsko deljenje pomeni, da nam program vnre samo $k$. Primer vhoda in izhoda za zgornji program, kjer števili nista deljivi:

\begin{inout}
16 3
\tcblower
5
\end{inout}

Tudi, če bi bil rezultat deljenja 7.9, tega program ne zaokroži na 8, temveč nam vrne 7. \\
Če manjše število delimo z večjim, zato vedno dobimo 0.

\end{errors}

\pagebreak
Operator \verb+/+ nam torej iz pri deljenju \verb+a/b+ vrne \verb+k+. Lahko pa dobimo tudi ostanek \verb+o+. Do njega pridemo z operatorjem \verb+%+ (\emph{modulo}).

\begin{examples}

%#insert python3 style.py < snippets/arithm/modulo.cpp

\begin{inout}
25 7
\tcblower
25 = 3*7 + 4 \\
Količnik: 3 \\
Ostanek: 4
\end{inout}

\end{examples}

\begin{errors}
Deljenje z 0 v matematiki ni definirano in prav tako ne v programiranju. Če neko število delimo z 0, se nam bo program sesul. Prav tako, če poskušamo izračunati ostanek pri deljenju z 0. \\
Ta napaka se pogosto zgodi v programih, kjer delimo z več števili, zato moramo biti na to pozorni.
\end{errors}
%#endblock
