%#template templates/template.tex

%#block title
Urejanje
%#endblock

%#block content

\section{Osnovno o urejanju}

V programih pogosto želimo nek seznam števil urediti po vrsti.
V ta namen lahko napišemo svojo funkcijo, ki implementira enega od znanih
algoritmov za urejanje; npr.~\textit{bubble sort}, \textit{insertion sort},
\textit{quick sort}, ipd. Ker pa so učinkovite implementacije pogosto komplicirane
in se pri pisanju hitro zmotimo, je bolje, da uporabimo funkcije, ravno v ta
namen vključene v standardno knjižnjico. Za to bomo potrebovali na začetek
programa dodati še dve vrstici:

%#insert python3 style.py < chapters/urejanje/algorithm.cpp

Ukaz \verb+#include+ že poznamo, opazimo pa, da tokrat za spremembo nima končnice
\verb+.h+. To je zato, ker funkcije, ki smo jih uporabljali do sedaj, izvirajo
iz jezika C, tokrat pa potrebujemo funkcijo, napisano posebaj za C++. To razloži
tudi drugo vrstico; vse funkcije v standardni knjižnjici v C++ so vključene
v imenski prostor \verb+std+. Če jih želimo klicati, moramo pred ime funkcije
vedno napisati \verb+std::+, ali pa na začetek programa vključiti vrstico
\verb+using namespace std+.

Sedaj lahko uporabimo funkcijo \verb+sort+, ki sprejme dva argumenta;
začetek in konec predela spomina, ki ga želimo urediti. Poglejmo si enostavni
primer.

%#insert python3 style.py < chapters/urejanje/uporaba.cpp


Program prebere število \(n\), za njim pa še \(n\) števil, jih uredi naraščajoče,
in jih izpiše. Funkcijo \verb+sort+ smo poklicali tako, da smo kot prvi argument
podali seznam \verb+arr+, kot drugi argument pa konec predela seznama, ki ga
želimo urediti; to zapišemo kot \verb|arr+n|. Točen pomen tega izraza bomo
spoznali v prihodnje, za sedaj pa bomo kot drugi argument vedno podali seznam
plus njegovo dolžino.

Optimalna časovna zahtevnost algoritma za urejanje je \(O(n \log n)\). To v praksi
pomeni, da bo urejanje delovalo dovolj hitro za \(n \le 10^6\). Če imamo več
podatkov kot toliko, bo urejaje trajalo predolgo in naša rešitev ne bo sprejeta.

\section{Primerjalna funkcija}

Če želimo urediti seznam padajoče namesto naraščajoče, lahko seznam prvo uredimo
naraščajoče, in ga nato obrnemo. Ker s tem dobimo veliko dodatnega dela, je
bolje, da funkciji \verb+sort+ podamo lastno primerjalno funkcijo. Le-ta mora
sprejeti dva argumenta ter vrniti \verb+bool+, in sicer; če mora biti prvi
argument v urejenem seznamu levo od drugega, mora funkcija vrniti \verb+true+,
sicer pa \verb+false+.

Če ne podamo tretjega argumenta, se \verb+sort+ obnaša tako, kot da bi podali
naslednjo funkcijo:

%#insert python3 style.py < chapters/urejanje/compare.cpp

Če želimo urediti seznam padajoče, moramo le podati nasprotno funkcijo:

%#insert python3 style.py < chapters/urejanje/nasprotno.cpp

\subsection{Urejanje sestavljenih podatkov}

Recimo, da imamo v nalogi dana imena tekmovalcev ter točke, ki so jih ti
tekmovalci dosegli na tekmovanju, naš cilj pa je, da izpišemo imena tekmovalcev
po vrsti glede na doseženo število točk. Če bomo prebrali točke in imena v
različna seznama, ter uredili seznam točk, bo seznam imen ostal nespremenjen in
ne bomo več vedeli, katero ime pripada katerim točkam.

Kako uredimo oba seznama hkrati? Bolj enostavna možnost je uporaba \verb+struct+,
ki pa ga še ne poznamo. Namesto tega si lahko pripravimo seznam indeksov, ki
na začetku na \verb+i+-tem mestu hrani številko \verb+i+. Če sestavimo funkcijo
\verb+compare+ tako, da sprejme dva indeksa, ter ju uredi glede na vrednosti
v tabeli s točkami na pripadajočih indeksih. Urejamo pa ne tabele s točkami,
temveč novo tabelo indeksov.
Na ta način se tabeli s točkami in z imeni ne bosta spreminjali,
in bodo točke pripadale imenu na istem indeksu.

\begin{examples}

Primer implementacije opisane rešitve:

%#insert python3 style.py < chapters/urejanje/kombinirani.cpp

\begin{inout}
  5 \\
  France 37 \\
  Gregor 34 \\
  Julija 38 \\
  Matija 29 \\
  Urska 8 \\
  \tcblower
  Julija
  France
  Gregor
  Matija
  Urska
\end{inout}

\end{examples}

%#endblock