%#template templates/template.tex

%#block title
Seznami
%#endblock

%#block content

Ko si želimo podatke shraniti tako, da jih bomo lahko spreminjali in na koncu
nekaj z njimi naredili (npr.~da z njimi računamo, jih izpišemo, itd.), za to
uporabimo spremenljivke.
Vsaka spremenljivka hrani en podatek -- eno številko, en znak, \ldots, razen
nizov, ki lahko hranijo več zaporednih znakov.
Nizi so posebni na zanimiv način.
Med pisanjem programa nismo vedeli, točno kako dolg bo niz, ki ga bo uporabnik
vnesel; rekli smo le, da mora biti krajši od neke največje dolžine niza, ki smo
jo vnesli v program.
Niz s kapaciteto 201 je tako lahko shranil do 200 znakov; namesto dvestotih
spremenljivk smo vse te znake shranili v eno.
Pogosto si želimo tudi števila shraniti tako, da bomo kasneje lahko dostopali do
njih, ampak med pisanjem programa ne vemo točno, s koliko števili bo program
moral delati.
Problem rešimo s seznami.

\section{Sintaksa}

Osnovna sintaksa seznamov (angl.~\textit{array}) je zelo podobna sintaksi nizov.
Da ustvarimo nov seznam, napišemo ime tipa (\verb+int+, \verb+long long+, itd.),
ime spremenljivke in nato največjo možno dolžino seznama:
%#insert python3 style.py < chapters/arrayi/sintaksa.cpp
Prav tako kot pri nizih tudi do posamičnih elementov seznama dostopamo z
oglatimi oklepaji:
%#insert python3 style.py < chapters/arrayi/dostop.cpp
Opazimo da, prav tako kot pri nizih, tudi pri seznamih šteti začnemo pri 0.
Spomnimo se, da smo pri nizih imeli posebni znak NULL, ki je programu sporočal,
da se naš niz na tistem mestu konča.
Za sezname števil takega znaka ne poznamo; ko delamo s števili, moramo drugače
vedeti, kako dolg naš seznam dejansko je.
Naloge so običajno narejene tako, da prvo število na vhodu pove, koliko števil
(ki jih želimo shraniti v seznam) bo sledilo.
V večini primerov je to ravno dolžina našega seznama, če pa delamo kakšne
posebne trike, pa moramo za dolžino skrbeti sami.

\begin{examples}
  Poglejmo si primer preproste naloge:
  Na vhodu je podano število $N$, ki mu sledi $N$ števil.
  Program naj izpiše vsoto teh $N$-tih števil.
  Velja $N \le 10^5$.
  %#insert python3 style.py < chapters/arrayi/branje.cpp
  \begin{inout}
	5
	1 2 3 4 5
	\tcblower
	15
  \end{inout}

  Pri programu opazimo nekaj posebnosti.
  Seznam števil smo postavili \emph{zunaj} funkcije \verb+main+.
  To je zmeraj dobro narediti, če uporabljamo zelo dolge sezname, kakor zgornji
  seznam je.
  Poleg tega smo največjo dolžino seznama nastavili na $10^5+2$.
  Lahko bi nastavili dolžino na točno $10^5$, vendar je dobra ideja, da si
  pustimo malce praznega prostora, če se kje v programu zmotimo pri indeksiranju.
  Ta prazen prostor nam lahko v nekaterih primerih prepreči, da se program sesuje.
\end{examples}

\begin{examples}
  Poglejmo si malo bolj zanimiv primer.
  Naša naloga sedaj je, da uredimo seznam $N$ števil ($N \le 10^6$) po velikosti
  od najmanjšega do največjega.
  Podano imamo tudi informacijo, da bodo ta števila velika med vključno $0$ in
  $100$.
  Naloge se lotimo tako, da preštejemo, kolikokrat se neko število pojavi v
  danem seznamu, nato pa bomo seznam rekonstruirali tako, da bo urejen.
  Da preštejemo, kolikokrat se kakšno število pojavi, uporabimo nov seznam, kjer
  indeks pomeni številko, ki jo štejemo, shranjena vrednost pa kolikokrat smo to
  številko že prešteli.
  %#insert python3 style.py < chapters/arrayi/counting.cpp
  Ta algoritem za urejanje je zelo znan; imenuje se urejanje s preštevanjem
  (angl.~\textit{counting sort}).
  Primeren je, kadar imamo zelo majhen razpon možnih vrednosti števil, kakor smo
  imeli tu ($0 - 100$).
\end{examples}

\section{Večdimenzionalni seznami}

Videli smo, kako ustvariti seznam števil, kaj pa seznam seznamov?
Takemu seznamu pravimo \textit{dvodimenzionalen seznam}, ustvarimo pa ga tako,
da napišemo dva zaporedna oglata oklepaja z velikostjo;
%#insert python3 style.py < chapters/arrayi/sintaksa-2d.cpp
Dvodimenzionalni seznami so uporabni, kadar moramo podatke predstaviti v tabeli.
Do posamičnih elementov dostopamo z dvojnimi oglatimi oklepaji, tako kot pri
inicializaciji spremenljivke; \verb+tabela[i][j]+.
Tabele si običajno predstavljamo tako, da nam prvi indeks pove zaporedno
številko vrstice, drugi pa zaporedno številko stolpca.

Druga možna uporaba dvodimenzionalnih seznamov je ustvarjanje seznamov nizov.
Sezname smo obravnavali kakor nekaj sorodnega nizom, dejansko pa je pravilna
smer razmišljanja tu ravno obratna; nizi so pravzaprav le seznami znakov.
Če želimo narediti seznam nizov, ustvarimo dvodimenzionalni seznam znakov, v
katerem nam bo prvi indeks predstavljal indeks niza, drugi indeks seznama pa nam
bo predstavljal indeks znaka v nizu.

%#endblock