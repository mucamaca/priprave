%#template templates/template.tex

%#block title
Pogojni stavki
%#endblock

%#block content

Pogosto želimo, da računalnik izvaja drugačno kodo glede na vrednost ene ali
večih spremenljivk, npr. da nam pokaže drugačno vsebino, če smo napisali
pravilno ali napačno geslo, da računalo sešteva, če smo pritisnili gumb za
seštevanje, oz. odšteva, če smo pritisnili gumb za odštevanje, ipd. Z drugimi
besedami, želimo upravljati potek programa (torej izbrati, katera koda naj
se izvede) glede na vrednosti spremenljivk. Angleško takemu upravljanju
pravimo \emph{control flow}, najpogosteje pa ga izvajamo s t.i. \emph{pogojnimi}
ali \emph{if stavki}. Osnovna struktura je sledeča:

\begin{examples}

%#insert python3 style.py < snippets/basic_conditionals/sintaksa.cpp

\end{examples}

\emph{Pogoj} je nov pojem. Označuje neke vrste račun, katerega rezultat ni
število, vendar \emph{logična vrednost}. Tu sta možni vrednosti le dve:
pravilno (angl.~\verb+true+) in napačno (angl.~\verb+false+). Če bo rezultat
računa, navedenega v običajnih oklepajih v zgornjem \verb+if+ stavku,
\verb+true+,
se bo izvedla koda znotraj prvih zavitih oklepajev, če pa je rezultat računa
\verb+false+, pa se bo izvedla koda v drugih zavitih oklepajih (tistih za besedo
\verb+else+). Drugega dela, t.j. \verb+else+ in oklepaji za njim, ni treba
pisati, če ga ne želimo.

Kako pa zapišemo pogoj? Za to uporabimo posebne logične operatorje. Pri delu s
številkami so nam na voljo naslednji:
\begin{itemize}
  \item \verb+==+: primerja dve številski vrednosti.
	Rezultat je \verb+true+, če sta vrednosti enaki.
  \item \verb+!=+: primerja dve številski vrednosti.
	Rezultat je \verb+true+, če sta vrednosti različni.
  \item \verb+<+: primerja dve številski vrednosti.
	Rezultat je \verb+true+, če je vrednost na levi manjša od vrednosti na desni.
  \item \verb+>+: deluje podobno kot \verb+<+, le da v drugo smer;
	rezultat je \verb+true+, če je vrednost na desni manjša od vrednosti na levi.
  \item \verb+<=+: primerja dve številski vrednosti.
	Rezultat je \verb+true+, če sta vrednosti enaki, ali če je vrednost
	na levi manjša od vrednosti na desni.
  \item \verb+>=+ deluje podobno kot \verb+<=+, le da v drugo smer.
\end{itemize}

\begin{examples}

Poglejmo si primer uporabe pogojnega stavka.

%#insert python3 style.py < snippets/basic_conditionals/stevila.cpp

\begin{inout}
12 12
\tcblower
Stevili sta enaki.\\
Prvo stevilo je manjse ali enako drugemu.\\
Prvo stevilo je vecje ali enako drugemu.
\end{inout}

\begin{inout}
3 7
\tcblower
Stevili sta razlicni.\\
Prvo stevilo je manjse od drugega.\\
Prvo stevilo je majnse ali enako drugemu.
\end{inout}

\end{examples}


\begin{errors}
  Pri primerjavi moramo uporabiti dva enačaja (\verb+==+). Če uporabimo samo
  en enačaj, kot v matematiki (torej \verb+=+), se bo program sicer zagnal,
  vendar ne bo deloval pravilno.
  Enojnega enačaja nikoli ne uporabljamo v pogoju \verb+if+ stavka.
\end{errors}

\begin{errors}
  % Pri predelavi v učbenik: pogovor, če je smiselno že tu omenjati strcmp
  Taka primerjava deluje samo na številih. Če želimo primerjati dve besedili,
  za to potrebujemo posebno funkcijo \verb+strcmp+, ki jo bomo
  podrobneje spoznali, ko bomo govorili o besedilu.
\end{errors}

\begin{examples}

Primer uporabe stavka \verb+else+. Program bo uporabnika vprašal za PIN,
in mu napisal, če je bil PIN pravilen oziroma napačen.

%#insert python3 style.py < snippets/basic_conditionals/pin.cpp

\begin{inout}
42
\tcblower
PIN je pravilen!
\end{inout}

\begin{inout}
7
\tcblower
PIN je napacen!
\end{inout}

\end{examples}

\begin{examples}
\verb+if+ stavke lahko tudi gnezdimo; torej vstavimo enega v drugega.
Spodnji program od uporabnika sprejme naročilo v restavraciji, kjer ponujajo
dve vrsti hrane; juhe in sendviče. Na voljo sta dve vrsti juhe, in dve vrsti
sendvičev.

%#insert python3 style.py < snippets/basic_conditionals/nesting.cpp

\begin{inout}
2\\
1
\tcblower
En sendvic s sunko. Dober tek!
\end{inout}
\end{examples}


\begin{errors}

Pozorni bodimo tudi na postavitev kode. Običajno kodo znotraj zavitih oklepajev
\verb+if+ stavka pišemo tako, da je poravnana štiri presledke bolj desno od
kode zunaj \verb+if+ stavka. V nekaterih programskih jezikih je taka poravnava
obvezna, v C/C++ pa ne, vendar nezamaknjena koda že v zelo majhnih programih
postane popolnoma nepregledna. Branje in popravljanje kode je veliko lažje, če
del kode znotraj zavitih oklepajev zamaknemo za štiri presledke. Za to lahko
uporabimo tudi tipko Tab, ki se na tipkovnici nahaja levo od tipke Q.

\vspace{0.3cm}
Tako \textbf{nikoli} ne pišemo:

%#insert python3 style.py < snippets/basic_conditionals/unindented.cpp

\end{errors}

%#endblock
