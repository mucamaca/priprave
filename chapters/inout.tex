%#template templates/template.tex

%#block title
Branje in pisanje
%#endblock

%#block content
\section{Vhod in izhod}
Programi za svoje delovanje potrebujejo način za komunikacijo z
uporabnikom. Kompleksnejši programi v ta namen uporabljajo ekran, miško
in tipkovnico. Pri tekmovalnem programiranju pa najpogosteje uporabljamo najpreprostejši način za komunikacijo: pisanje in branje s \emph{standardnega vhoda in izhoda}. Običajno to pomeni, da se nam ob zagonu programa odpre okno, kamor lahko pišemo programu in kamor program izpisuje stvari.\\
Ko želimo, da naš program kaj izpiše, uporabimo \emph{funkcijo} \verb+printf+.
\begin{examples}

%#insert python3 style.py < snippets/inout/helloworld.cpp

\begin{inout}
\tcblower
Hello World!
\end{inout}

\end{examples}

\verb+printf+ - funkcija, ki ji v dvojnih narekovajih damo besedilo ali števila, ki jih želimo izpisati\\
\verb+\n+ - znak za novo vrstico
\\\\
Stvari, ki jih mora vsebovati (skoraj) vsak program:
\begin{itemize}
	\item \verb+stdio.h+ - \emph{knjižnica} (datoteka), ki vsebuje funkcije, ki jih bomo uporabljali v programu (kot npr. \verb+printf+ in \verb+scanf+)
	\item \verb+#include<>+ - ukaz, s katerim našemu programu povemo, katere knjižnice potrebuje
	\item \verb+int main(){}+ - telo našega programa - večino kode v programu napišemo med zavite oklepaje
	\item \verb+return 0+ - zadnja vrstica v programu, ki sporoča računalniku, da se je pravilno zaključil
\end{itemize}

\noindent V program lahko dodamo \emph{komentarje}. To je takšno besedilo, ki je napisano v kodi, a vsebinsko ne vpliva na program.
\begin{itemize}
	\item \verb+//+ - s tem zakomentiramo vse od poševnic do konca vrstice
	\item \verb+/* */+ - s tem lahko zakomentiramo več vrstic ali del znotraj vrstice
\end{itemize}

\begin{examples}

%#insert python3 style.py < snippets/inout/komentarji.cpp

\begin{inout}
\tcblower
Zivjo svet!
\end{inout}

\end{examples}

\begin{errors}
Zadnji znak, ki ga program izpiše, mora biti \verb+\n+.
\end{errors}

\begin{errors}
Večina vrstic v programu se konča s podpičjem (\verb+;+) (skoraj vse razen tistih, ki se končajo z oklepaji  ali zavitimi zaklepaji). Brez tega program ne bo delal.
\end{errors}

\pagebreak
\section{Branje}
Program za branje stvari, ki mu jih sporočamo, uporablja funkcijo \verb+scanf+. 

\begin{examples}

%#insert python3 style.py < snippets/inout/branje.cpp

\begin{inout}
{\color{blue} \bf output:} Kako ti je ime?\\
{\color{blue} \bf input:} Tinka \\
{\color{blue} \bf output:} Zivjo, Tinka! 
\end{inout}

\end{examples}


Če želimo, da program lahko kaj počne s podatki, ki jih je prebral, moramo najprej to shraniti na neko mesto v spominu. Temu mestu rečemo \emph{spremenljivka}, saj lahko s programom spreminjamo, kaj je tam shranjeno. Spremenljivke v svojem programu poimenujemo, v našem primeru ji rečemo \verb+ime+. Vsaki spremenljivki moramo določiti \emph{podatkovni tip}, saj lahko beremo in pišemo več različnih vrst podatkov, npr. besede ali števila. \\\\
\verb+char+ - s tem povemo, da je naša spremenljivka besedilo oz. \emph{niz} \\\\
\verb+[...]+ - številka v oglatih oklepajih za besedo pove največjo dolžino niza, ki ga lahko program prebere.

\begin{errors} %CHECK
Ko določamo največjo dolžino niza, vedno vzamemo večjo številko, kot jo bomo potrebovali. O tem bomo več govorili v kasnejših poglavjih.
\end{errors}


Funkciji \verb+scanf+ podamo dva ali več \emph{parametrov}:
\begin{itemize}
	\item kaj naj prebere, torej kakšne tipe spremenljivk. To podamo s \emph{formatnikom} (v našem primeru \verb+%s+, ki pomeni niz (\emph{string}). \verb+%s+ prebere vse znake do prvega presledka ali nove vrstice)
	\item ostali parametri povejo, kam naj funkcija shrani stvari, ki jih je prebrala (torej v spremenljivke, ki smo jih naredili prej)
\end{itemize}


%CHECK
Do zdaj smo funkciji \verb+printf+ podali samo točno določeno besedilo, ki smo ga želeli izpisati. Izpisujemo pa lahko tudi spremenljivke, kot smo to naredili v tem zadnjem primeru. Znotraj besedila dodamo formatnike na mesta, kjer želimo, da so spremenljivke, potem pa izven narekovajev naštejemo imena spremenljivk, ki jih želimo izpisati (tako kot pri funkciji \verb+scanf+).

\pagebreak
\section{Branje števil}
%CHECK
Poleg nizov lahko programi delajo tudi s števili. Števila beremo in izpisujemo z istima funkcijama kot nize, vendar moramo uporabiti drug formatnik.

\begin{examples}

%#insert python3 style.py < snippets/inout/branje_stevil.cpp

\begin{inout}
{\color{blue} \bf output:} Kateri razred si?\\
{\color{blue} \bf input :} 7\\
{\color{blue} \bf output:} 7. razred je najboljši.
\end{inout}

\end{examples}

%CHECK
\verb+&+ - znak, ki ga moramo dati pred ime spremenljivke vedno, kadar beremo števila.
\verb+int+ - podatkovni tip število \\
Pri številih za imenom spremenljivke ne povemo, kako velika so lahko. \\\\
Za branje in pisanje števil uporabimo formatnik \verb+%d+.

\begin{errors}
Ko beremo števila, moramo pred ime spremenljivke dati znak \verb+&+, česar pri branju nizov ne delamo. Prav tako tega ne delamo pri izpisovanju števil.
\end{errors}
%#endblock
