%#template templates/template.tex

%#block title
Pogojni stavki: 2. del
%#endblock

%#block content

\section{Nizanje pogojev}

Pogosto se srečamo s problemi, kjer je za rešitev potrebno upoštevati več kot
en pogoj. V takih primerih želimo združiti več pogojnih stavkov tako, da se
nek del kode izvede, če velja prvi pogoj, drugi del kode pa, če prvi pogoj
ne velja, velja pa drugi pogoj. Z gnezdenjem stavkov lahko to v kodo vključimo
na naslednji način:

%#insert python3 style.py < chapters/more-conditionals/nizanje-narobe.cpp

V takem primeru nam je na voljo bližnjica \verb+else if+. Zgornja koda deluje
popolnoma enako kot spodnja:

%#insert python3 style.py < chapters/more-conditionals/nizanje-pravilno.cpp

Prednost te bližnjice je, da je naša koda krajša in bolj razumljiva.
Stavke \verb+else if+ lahko tudi verižimo; enemu \verb+if+ stavku lahko sledi
poljubno mnogo stavkov \verb+else if+. Pri tem bo računalnik pogoje preverjal po
vrsti. Pri prvem veljavnem pogoju se bo ustavil in izvedel kodo v pripadajočih
zavitih oklepajih, za čimer ne bo več preverjal pogojev, temveč bo izvajanje
nadaljeval za zaključkom vseh nanizanih stavkov.

Kakor nam v osnovnem \verb+if+ stavku ni bilo treba pisati dela z \verb+else+,
če ga nismo potrebovali, nam ga tudi pri uporabi \verb+else if+ ni treba.

\pagebreak
\begin{examples}
Naslednji program prebere število in pove, če je večje, manjše ali enako 0.

%#insert python3 style.py < chapters/more-conditionals/vecje-manjse-enako.cpp

\begin{inout}
3
\tcblower
Stevilo je vecje od 0.
\end{inout}

\end{examples}

\begin{examples}
Pogoji se preverjajo po vrsti; izvede se samo koda pri prvem veljavnem pogoju,
ne glede na to, koliko pogojev za njim je tudi veljavnih.

%#insert python3 style.py < chapters/more-conditionals/nizanje-elseif.cpp

Program izpiše le eno stvar - \verb+a je 7.+, kljub temu, da velja tudi
\verb+a > 1+.

\end{examples}

\section{Logični vezniki}

Osnovni operatorji za primerjavo pogosto niso dovolj, da izrazimo željen pogoj.
Če na primer želimo pogledati, ali je neko število med dvema drugima, tega ne
moramo narediti samo z eno primerjavo.

Pogoje združujemo s t.i. \emph{logičnimi vezniki}, ki jim včasih pravimo tudi
\emph{logični operatorji}. Poznamo tri osnovne veznike:
\begin{itemize}
  \item \verb+and+ oz.~\verb+&&+ združi dva pogoja tako, da združeni pogoj velja
	samo v primeru, da veljata oba hkrati.
  \item \verb+or+ oz.~\verb+||+ združi dva pogoja tako, da združeni pogoj velja
	v primeru, da velja katerikoli od dveh, ali da veljata oba
  \item \verb+not+ oz.~\verb+!+ sprejme samo en pogoj. Nov pogoj velja samo
	takrat, ko originalni pogoj ne velja.
\end{itemize}

Logična veznika \verb+&&+ (\emph{in}) ter \verb+||+ (\emph{ali}) lahko
predstavimo z resničnostno tabelo.

\begin{table}[h!]
  \centering
  \caption{Resničnostna tabela za in}
  \vspace{0.1cm}
  \begin{tabular}{c|c|c}
	Leva vrednost & Desna vrednost & Vrednost veznika \\
	\hline
	resnica & resnica & resnica \\
	resnica & neresnica & neresnica \\
	neresnica & resnica & neresnica \\
	neresnica & neresnica & neresnica
  \end{tabular}
\end{table}

\begin{table}[h!]
  \centering
  \caption{Resničnostna tabela za ali}
  \vspace{0.1cm}
  \begin{tabular}{c|c|c}
	Leva vrednost & Desna vrednost & Vrednost veznika \\
	\hline
	resnica & resnica & resnica \\
	resnica & neresnica & resnica \\
	neresnica & resnica & resnica \\
	neresnica & neresnica & neresnica
  \end{tabular}
\end{table}

\begin{examples}

Če želimo preveriti, ali je dano število med dvema drugima, uporabimo
logični veznik \verb+&&+.

%#insert python3 style.py < chapters/more-conditionals/med-steviloma.cpp

\begin{inout}
5
\tcblower
n je med 3 in 9.
\end{inout}

\end{examples}

\begin{errors}
  V matematiki pogosto napišemo dvojno primerjavo: \verb+a < b < c+.
  Če nekaj podobnega napišemo v C++ program, se bo le-ta sicer zagnal, vendar
  ne bo deloval pravilno. Kaj pričakujemo, da se zgodi, če v tako primerjavo
  zapišemo \verb+3 < 2 < 1+? Kaj pa se dejansko zgodi?
\end{errors}

Pri kombiniranju pogojev bodimo previdni glede pravil prednosti. Zanikanje
(veznik \verb+not+ oz.~\verb+!+) ima namreč prednost pred primerjalnimi
vezniki (\verb+<+, \verb+==+, \ldots), ter drugima ločnima veznikoma.
Če v takem primeru uporabljamo zanikanje, moramo zanikan izraz postaviti
v oklepaje.

\begin{examples}
  Zanikanje se lahko v večini primerov zapiše tudi na krajši način.

  \begin{tabular}{|c|c|c|}
	\hline
	Izraz & Ekvivalenten izraz & Komentar \\
	\hline
	\verb+!(a == b)+ & \verb+a != b+ & \\
	\verb+!(a < b)+ & \verb+a >= b + & Če sta \verb+a+ in \verb+b+ enaka, bo prvi izraz veljaven. \\
	\verb+!(a <= b)+ & \verb+a > b+ & \\
	\hline
  \end{tabular}
\end{examples}

\begin{examples}
Napišimo program, ki preveri, ali je uporabnik vpisal prestopno leto.
Leto je prestopno, če je deljivo s 4; razen, če je hkrati deljivo s 100. Izjema so leta, deljiva
s 400, ki so prestopna kljub temu, da so deljiva s 100.

Kako te pogoje zapišemo v program? Opazimo, da so leta, deljiva s 400,
prestopna ne glede na druga pogoja. Če leto ni deljivo s 400, potem mora biti
deljivo s 4 in ne sme biti deljivo s 100. Povedano krajše; leto mora biti
deljivo s 400, ali pa s 4 in ne hkrati s 100. Tak pogoj lahko zapišemo z
logičnimi vezniki.

%#insert python3 style.py < chapters/more-conditionals/prestopno-leto.cpp

\end{examples}

%#endblock
