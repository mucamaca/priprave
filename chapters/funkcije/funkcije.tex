%#template templates/template.tex

%#block title
Funkcije
%#endblock

%#block content
Ko pišemo daljše programe, se pogosto zgodi, da večrat uporabimo nek podoben
kos kode. Da si v takih situacijah olajšamo življenje, imamo \emph{funkcije}.
Že v zgodnjih sedemdesetih, ko se je začel razvijati jezik C so vedeli, da so
funkcije zelo uporabne, zato je osnovna oblika tega jezika sestavljena iz njih.
Res, funckijam se v C-ju ne da izogniti, do zdaj smo brali in pisali podatke s
funkcijama \verb+scanf+ in \verb+printf+, preverjali dolžino nizov s
\verb+strlen+ in vse smo napisali v funkciji \verb+main+.

Funkcijo si lahko predstavljamo kot neko napravo. Napravi podamo nekaj
\emph{parametrov} (včasih se uporabi tudi beseda \emph{argumentov}), ona nekaj
melje in nam potem \emph{vrne} neko vrednost. Poleg tega, da samo vrne neko
vrednost, ima lahko funkcija tudi kakšne stranske učinke. Lahko kaj izpiše z
uporabo \verb+printf+, ali pa spremeni kakšno spremenljivko. Včasih so
parametri ali vrnjena vrednost nepotrebni, zato lahko napišemo tudi funkcije
brez parametrov in funkcije, ki ne vrnejo ničesar.


\section{Kako napišemo svojo funkcijo}

Najbolj osnovna oblika funkcije je takšna, ki ne sprejme nobenega parametra in
ne vrne ničesar. Če hočemo, da ta funkcija ni neuporabna, bo morala imeti
kakšen stranski učinek, funkcija spodaj bo izpisala \verb+Zdravo+.

\begin{examples}
    %#insert python3 style.py < chapters/funkcije/osnovna_funkcija.cpp
\end{examples}

\begin{itemize}
    \item \verb+void+ - Posebna beseda, ki pove, da funkcija ne vrne ničesar.
    \item \verb+funkcija+ - Ime za funkcijo, kot pri spremenljivkah bi lahko tu
        napisali karkoli.
    \item \verb+()+ - Ime funkcije se more končati z (), v te oklepaje
        postavljamo parametre.
    \item \verb+{ ... }+ - Telo funkcije, v katerem je koda, ki se bo izvedla.
\end{itemize}

Kot napovedano, imajo lahko funkcije tudi kakšne parametre:

\begin{examples}
    %#insert python3 style.py < chapters/funkcije/funkcija_parametri.cpp
\end{examples}

Zdaj smo v oklepaje za imenom spremenljivke postavili \verb+int n+. Beseda
\verb+int+ pove, da je ta parameter celo število, \verb+n+ je pa ime za
parameter. Ime, ki smo si ga izbrali za parameter, uporabljamo v telesu
funkcije, da dostopamo do vrednosti v parametru.

Če je naš parameter seznam, to napišemo tako:

\begin{examples}
    %#insert python3 style.py < chapters/funkcije/funkcije_arr_parameter.cpp
\end{examples}

Zdaj pa še funkcije, ki vrnejo kakšno vrednost:

\begin{examples}
    %#insert python3 style.py < chapters/funkcije/funkcija_return.cpp
\end{examples}

Zdaj smo v funkcijo dodali ukaz \verb+return+, ki pove programu, da želimo
vrniti spremenljivko, ki se pojavi takoj za njim. Spodaj pa vidimo, da moramo
za dostop do vrnjene vrednosti nastaviti spremenljivko. Izraz
\verb+funkcija(3, 4)+ lahko uporabimo kjerkoli, kjer bi lahko uporabili neko
spremenljivko. Prav tako lahko za \verb+return+-om seštevamo števila. Zgornji
program bi lahko krajše napisali kot:

\begin{examples}
    %#insert python3 style.py < chapters/funkcije/funkcija_return_kratko.cpp
\end{examples}


\section{Potek funkcije}

Kot smo navajeni, se ukazi v telesu funkcije izvajajo po vrsti od zgoraj
navzdol. V njej lahko uporabimo vse, kar smo se do zdaj naučili (lahko
ustvarjamo nove spremenljivke, if stavke, zanke, \ldots ). Vredno omembe je, da
se funkcija konča, ko se izvede prvi \verb+return+.

\begin{examples}
    %#insert python3 style.py < chapters/funkcije/return_konca_funkcijo.cpp
\end{examples}

Tu funkcija vrača logično vrednost \verb+bool+, ki je ena izmed \verb+true+ ali
\verb+false+, uporablja se jo lahko v if stavku. Deluje torej pravilno, saj če
bo \verb+crka+ enaka a, e, i, o, u, se bo funkcija končala pri \verb+return+-u
v if stavku, sicer se bo pa nadaljevala do \verb+return false;+.


\section{Spremenljivke in funkcije}

\subsection{Globalne in lokalne spremenljivke}

Spomnimo se, da lahko spremenljivke definiramo na dva načina. Tistim, ki so
definirane na vrhu, izven funkcije rečemo \emph{globalne spremenljivke},
tistim, ki so pa definirane v funkciji, pa \emph{lokalne spremenljivke}. Zdaj,
ko vemo, kaj so funkcije, lahko povemo razliko med temi tipi spremenljivk.

Globalne spremenljivke so vidne povsod, torej v vseh funkcijah, medtem ko lahko
lokalne spremenljivke uporabljamo samo v funkciji, v kateri smo jih definirali.

\begin{examples}
    %#insert python3 style.py < chapters/funkcije/globalne_lokalne_spremenljivke.cpp

    \begin{inout}
        \tcblower
        12 \\
        15, 200
    \end{inout}

\end{examples}

Seveda, če globalno spremenljivko nekje spremenimo, se bo spremenila tudi za
vse druge funkcije. V zgornjem primeru je \verb+funkcija+ najprej izpisala
\verb+g+ (12) in ga nato povečala na 15. Ko se je g izpisal drugič, je bil zato
15.

\subsection{Spreminjanje parametrov}

Če funkciji podamo neko spremenljivko kot
parameter in poskušamo to spremenljivko v funkciji spreminjati, se obnaša na
dva različna načina, odvisno od tega, če je spremenljivka seznam.

Če spremenljivka ni seznam (npr. \verb+int+, \verb+char+) se za funkcijo
ustvari kopija te spremenljivke. Če jo v funkciji spremenimo, to ne bo
spremenilo izvirne spremenljivke, saj v funkciji delamo s kopijo. Če je seznam,
bodo pa spremembe na seznam v funkciji vplivale na izvirni seznam, saj se ta ne
bo prekopiral. V spodnjem primeru sta prikazani obe možnosti.

\begin{examples}
    %#insert python3 style.py < chapters/funkcije/seznam_se_ne_kopira.cpp
\end{examples}

\newpage

Sprva ne vidimo smisla, za drugačno obravnavo seznamov, a se izkaže, da je
lahko ta lastnost tudi uporabna. Če bi želeli napisati funkcijo, ki sprejeme
seznam in njegovo dolžino, potem pa obrne vrstni red elementov tega seznama,
lahko to napišemo takole:

\begin{examples}

    %#insert python3 style.py < chapters/funkcije/obrni_seznam.cpp

    \begin{inout}
        5 \\
        10 20 30 -3 7 1
        \tcblower
        1 \\ 7 \\ -3 \\ 30 \\ 20 \\ 10 \\
    \end{inout}

\end{examples}

Da je postopek za obračanje seznamov res pravilen, lahko razmislite sami, namen
tega primera je, da vidimo, kako se pri funkcijah dela s seznami. Seznam, ki ga
želimo obdelati, prejmemo kot parameter in ga potem spremenimo, ničesar pa ne
vrnemo.


\section{Rekurzija}

\emph{Rekurzija} je način pisanja programa na način, da funkcija kliče samo
sebe. Pogosto lahko zanko nadomestimo z rekurzijo, če nam je tak način
programiranja ljubši.

Recimo, da želimo sešteti vsa števila med 1 in nekim \verb+n+. Z zanko bi to
naredili tako, da si pripravimo neko spremenljivko \verb+vsota+ in potem
napišemo for zanko od 1 do \verb+n+ in v \verb+vsota+ seštejemo vsa ta števila. Pri
rekurziji je pa pristop drugačen.

\begin{examples}
    %#insert python3 style.py < chapters/funkcije/vsota_rekurzivno.cpp
\end{examples}

Najprej napišemo \emph{osnovni pogoj}, ki za n-je manjše ali enake 0 vrne 0. Če
ne bi napisali osnovnega pogoja bi, šli rekurzivni klici funkcije v neskončnost,
\verb+vsota(3)+ bi poklicala \verb+vsota(2)+, ki pokliče \verb+vsota(1)+, potem
\verb+vsota(0)+, \verb+vsota(-1)+, \verb+vsota(-2)+, \ldots \ Po osnovnem
pogoju moramo narediti \emph{korak}, ki izračuna \verb+vsota(n)+ s pomočjo
vrednosti funkcije v nekem manjšem številu, v našem primeru kar
\verb+vsota(n - 1)+.


\section{Funkcije v C knjižnici}

\subsection{Funkcije iz \texttt{<stdio.h>}}

\begin{itemize}
    \item \verb+sprintf+ deluje podobno kot \verb+printf+, le da sprejme niz
        kot prvi parameter, naslednji parametri so pa enaki kot v
        \verb+printf+. Namesto da bi izpisala na standardni izhod, ta funkcija
        izpiše v niz, ki je bil dan kot prvi parameter.
    \item \verb+sscanf+ deluje podobno kot \verb+scanf+, le da sprejme niz kot
        prvi parameter, naslednji parametri so pa enaki kot v \verb+scanf+.
        Namesto da bi prebrala iz standardnega vhoda, ta funkcija bere iz niza,
        ki je bil dan kot prvi parameter.
\end{itemize}

\begin{examples}
    %#insert python3 style.py < chapters/funkcije/pretvorba_int_str.cpp
\end{examples}

\subsection{Funkcije iz <\texttt{string.h>}}

\begin{itemize}
    \item \verb+strcpy+ kopira nize. Sprejme dva niza in drugega prepiše v
        prvega.
    \item \verb+strcat+ stika nize, enega za drugim. Sprejme dva niza in drugi
        niz prepiše na konec prvega tako, da prvi postane oba niza staknjena
        skupaj.
    \item \verb+strcmp+ primerja nize. Sprejme dva niza, če sta niza enaka vrne
        število 0, če sta pa različna pa vrne število različno od 0. Nize lahko
        tako kot v slovarju ali telefonskem imenuku uredimo po abecedi, če
        pride prvi niz pred drugim, bo vrnjeno število negativno, v obratnem
        primeru pa pozitivno.
    \item \verb+strlen+ že poznamo, sprejme en niz in vrne njegovo dolžino.
\end{itemize}

\begin{examples}
    %#insert python3 style.py < chapters/funkcije/str_funkcije.cpp
\end{examples}

%#endblock

